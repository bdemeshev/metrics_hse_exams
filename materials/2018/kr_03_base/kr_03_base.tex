\documentclass[12pt]{article}

\usepackage{tikz} % картинки в tikz
\usepackage{microtype} % свешивание пунктуации

\usepackage{array} % для столбцов фиксированной ширины


\usepackage{comment} % для комментирования целых окружений

\usepackage{indentfirst} % отступ в первом параграфе

\usepackage{sectsty} % для центрирования названий частей
\allsectionsfont{\centering}

\usepackage{amsmath, amssymb, amsthm} % куча стандартных математических плюшек

\usepackage{amsfonts}

\usepackage[top=2cm, left=1cm, right=1cm, bottom=2cm]{geometry} % размер текста на странице

\usepackage{lastpage} % чтобы узнать номер последней страницы

\usepackage{enumitem} % дополнительные плюшки для списков
%  например \begin{enumerate}[resume] позволяет продолжить нумерацию в новом списке
\usepackage{caption}

\usepackage{hyperref} % гиперссылки

\usepackage{multicol} % текст в несколько столбцов


\usepackage{fancyhdr} % весёлые колонтитулы
\pagestyle{fancy}
\lhead{Эконометрика, ВШЭ}
\chead{Контрольная 3}
\rhead{2019-03-30}
\lfoot{Вариант $\mu$}
\cfoot{Паниковать запрещается!}
\rfoot{Тест}
\renewcommand{\headrulewidth}{0.4pt}
\renewcommand{\footrulewidth}{0.4pt}

\usepackage{ifthen} % для написания условий

\usepackage{todonotes} % для вставки в документ заметок о том, что осталось сделать
% \todo{Здесь надо коэффициенты исправить}
% \missingfigure{Здесь будет Последний день Помпеи}
% \listoftodos --- печатает все поставленные \todo'шки


% более красивые таблицы
\usepackage{booktabs}
% заповеди из докупентации:
% 1. Не используйте вертикальные линни
% 2. Не используйте двойные линии
% 3. Единицы измерения - в шапку таблицы
% 4. Не сокращайте .1 вместо 0.1
% 5. Повторяющееся значение повторяйте, а не говорите "то же"


\usepackage{fontspec}
\usepackage{polyglossia}

\setmainlanguage{russian}
\setotherlanguages{english}

% download "Linux Libertine" fonts:
% http://www.linuxlibertine.org/index.php?id=91&L=1
\setmainfont{Linux Libertine O} % or Helvetica, Arial, Cambria
% why do we need \newfontfamily:
% http://tex.stackexchange.com/questions/91507/
\newfontfamily{\cyrillicfonttt}{Linux Libertine O}

\AddEnumerateCounter{\asbuk}{\russian@alph}{щ} % для списков с русскими буквами
\setlist[enumerate, 2]{label=\asbuk*),ref=\asbuk*}

%% эконометрические сокращения
\DeclareMathOperator{\Cov}{Cov}
\DeclareMathOperator{\Corr}{Corr}
\DeclareMathOperator{\Var}{Var}
\DeclareMathOperator{\E}{E}
\def \hb{\hat{\beta}}
\def \hs{\hat{\sigma}}
\def \htheta{\hat{\theta}}
\def \s{\sigma}
\def \hy{\hat{y}}
\def \hY{\hat{Y}}
\def \v1{\vec{1}}
\def \e{\varepsilon}
\def \he{\hat{\e}}
\def \z{z}
\def \hVar{\widehat{\Var}}
\def \hCorr{\widehat{\Corr}}
\def \hCov{\widehat{\Cov}}
\def \cN{\mathcal{N}}
\def \P{\mathbb{P}}


\def \putyourname{\fbox{
    \begin{minipage}{42em}
      Фамилия, имя, номер группы:\vspace*{3ex}\par
      \noindent\dotfill\vspace{2mm}
    \end{minipage}
  }
}

\def \checktable{

	\vspace{5pt}
	Табличка для проверяющих работу:

\vspace{5pt}

	\begin{tabular}{|m{2cm}|m{1cm}|m{1cm}|m{1cm}|m{1cm}|m{1cm}|m{2cm}|}
\toprule
		Тест & 1 &  2 & 3 & 4 & 5 & Итого \\
\midrule
		&  &  & & & & \\
		&  &  & & & & \\
 \bottomrule
\end{tabular}
}



\def \testtable{

	\vspace{5pt}
	Внесите сюда ответы на тест:

\vspace{5pt}

\begin{tabular}{|m{2cm}|m{0.6cm}|m{0.6cm}|m{0.6cm}|m{0.6cm}|m{0.6cm}|m{0.6cm}|m{0.6cm}|m{0.6cm}|m{0.6cm}|m{0.6cm}|}
\toprule
		Вопрос & 1 &  2 & 3 & 4 & 5 & 6 & 7 & 8 & 9 & 10 \\
\midrule
		Ответ &  &  & & & & & & & & \\
 \bottomrule
\end{tabular}
}




% [1][3] 1 = one argument, 3 = value if missing
% эта магия создаёт окружение answerlist
% именно в окружении answerlist записаны варианты ответов в подключаемых exerciseXX
% просто \begin{answerlist} сделает ответы в три столбца
% если ответы длинные, то надо в них руками сделать
% \begin{answerlist}[1] чтобы они шли в один столбец
\newenvironment{answerlist}[1][3]{
\begin{multicols}{#1}

\begin{enumerate}[label=\fbox{\emph{\Alph*}},ref=\emph{\alph*}]
}
{
\item Нет верного ответа.
\end{enumerate}
\end{multicols}
}

% BB: unicol version. don't know why \ifthenelse fails in second part of new-env
\newenvironment{answerlistu}{
\begin{enumerate}[label=\fbox{\emph{\Alph*}},ref=\emph{\alph*}]
}
{
\item Нет верного ответа.
\end{enumerate}
}



\excludecomment{solution} % without solutions

\theoremstyle{definition}
\newtheorem{question}{Вопрос}



\begin{document}

\putyourname


\testtable

\checktable


\begin{question}
Стьюдентизированные остатки регрессии используются
\begin{answerlist}
  \item в тесте Саргана
  \item на первом шаге двухшагового МНК
  \item на первом шаге при проведении теста Годфельда-Квандта
  \item в методе главных компонент
  \item для выявления выбросов
\end{answerlist}
\end{question}




\begin{question}
Тест Саргана для проверки валидности инструментов можно использовать
только в том случае, если число инструментов
\begin{answerlist}
  \item меньше числа эндогенных переменных
  \item больше числа эндогенных переменных
  \item совпадает с числом эндогенных переменных
  \item совпадает с числом экзогенных переменных
  \item меньше числа экзогенных переменных
\end{answerlist}
\end{question}




\begin{question}
(1 балл) Какое условие НЕ требуется в теореме Гаусса-Маркова?
\begin{answerlist}[2]
  \item матрица регрессоров \(X\) имеет полный ранг
  \item модель \(Y=X\beta + \varepsilon\) правильно специфицирована
  \item случайные ошибки \(\varepsilon_i\) не коррелированы
  \item случайные ошибки \(\varepsilon_i\) имеют одинаковые дисперсии
  \item случайные ошибки \(\varepsilon_i\) нормально распределены
  \item нет верного ответа
\end{answerlist}
\end{question}

\begin{solution}
\begin{answerlist}
  \item Bad answer :(
  \item Bad answer :(
  \item Bad answer :(
  \item Bad answer :(
  \item Good answer :)
\end{answerlist}
\end{solution}


\begin{question}
(1 балл) Выборочная корреляция между регрессорами \(X\) и \(Z\) равна \(0.5\). В
регрессии \(\hat Y_i = \hat\beta_0 + \hat\beta_1 X_i + \hat\beta_2 Z_i\)
показатель \(VIF\) для регрессора \(X\) равен
\begin{answerlist}
  \item \(1/4\)
  \item \(4/3\)
  \item \(1/2\)
  \item \(2\)
  \item \(3/4\)
  \item нет верного ответа
\end{answerlist}
\end{question}

\begin{solution}
\begin{answerlist}
  \item Bad answer :(
  \item Good answer :)
  \item Bad answer :(
  \item Bad answer :(
  \item Bad answer :(
\end{answerlist}
\end{solution}

\newpage

\begin{question}
Стьюдентизированные остатки регрессии используются
\begin{answerlist}
  \item в методе главных компонент
  \item на первом шаге двухшагового МНК
  \item в тесте Саргана
  \item для выявления выбросов
  \item на первом шаге при проведении теста Годфельда-Квандта
\end{answerlist}
\end{question}




\begin{question}
По 52 наблюдениям студент построил две регрессии,
\(\hat Y_i = 3.1 + 0.8X_i\) и \(\hat X_i = -0.3 + 0.2Y_i\). Коэффициент
\(R^2_{adj}\) для первой регрессии примерно равен
\begin{answerlist}
  \item \(0.14\)
  \item \(0.16\)
  \item \(0.40\)
  \item \(0.32\)
  \item \(0.37\)
\end{answerlist}
\end{question}




\begin{question}
Использование скорректированных стандартных ошибок Уайта при
гомоскедастичности приведет к
\begin{answerlist}
  \item смещённости МНК оценок коэффициентов
  \item повышению эффективности МНК оценок коэффициентов
  \item получению состоятельной оценки дисперсии случайной ошибки
  \item понижению эффективности МНК оценок коэффициентов
  \item несостоятельности МНК оценок коэффициентов
\end{answerlist}
\end{question}

\begin{solution}
========
\end{solution}



\begin{question}
Рассмотрим модель множественной регрессии \(Y=X\beta+\varepsilon\), где
\(\hat Y = X\hat\beta\), \(e=Y-\hat Y\). Величина \(RSS\) --- это
квадрат длины вектора
\begin{answerlist}
  \item \(\hat Y - \bar Y\)
  \item \(\varepsilon\)
  \item \(\hat Y\)
  \item \(Y-\bar Y\)
  \item \(e\)
\end{answerlist}
\end{question}



\newpage

\begin{question}
Рассмотрим модель
\(Y_i= \beta_0 + \beta_z Z_{i} + \beta_w W_{i} + \varepsilon\) при
гетероскедастичности. Стандартная ошибка МНК-оценки, рассчитываемая по
формуле \(se(\hat\beta_w)=\sqrt{RSS \cdot (X'X)^{-1}_{33}/(n-3)}\),
является
\begin{answerlist}
  \item смещённой
  \item несмещённой
  \item состоятельной
  \item смещённой вниз
  \item смещённой вверх
\end{answerlist}
\end{question}




\begin{question}
Чебурашка оценил модель \(Y_i = \beta_0 + \beta_1 X_i + \varepsilon_i\),
а Крокодил Гена --- модель \(X_i = \gamma_0 + \gamma_1 Y_i + u_i\).
Оказалось, что \(\hat\gamma_1 = 0.25/\hat\beta_1\). Величина \(R^2\) в
регрессии Чебурашки равна
\begin{answerlist}
  \item \(1\)
  \item \(0.5\)
  \item \(0\)
  \item \(0.75\)
  \item \(0.25\)
\end{answerlist}
\end{question}

\begin{solution}
\(R^2 = \hat\beta_1 \cdot \hat\gamma_1\)
\end{solution}





\newpage
\rfoot{Задачи}
\putyourname


\begin{enumerate}
  \item Сидоров Вова оценивает два неизвестных параметра: $a$ — где стоят ракеты, $b$ — где продают конфеты.

  Вова оценил параметры методом максимального правдоподобия и получил оценки $\hat a = 1.5$, $\hat b = 2.5$.
  Затем Вова решил проверить гипотезу $H_0$: $a=1$ и $b=2$.

  Значения функции правдоподобия, градиента и оценённой информации Фишера в двух точках
  частично приведены в таблице:


\begin{tabular}{lccc}
\toprule
Точка & $\ell(a, b)$ & $(\ell'_a, \ell'_b)$ &  $\hat I_F$ \\
\midrule
$a=1.5$, $b=2.5$ & -200 &  ? &
$\begin{pmatrix}
16 & -1 \\
-1 & 20 \\
\end{pmatrix}$ \\
$a=1$, $b=2$ & -250 &  $(2, -1)$ &
$\begin{pmatrix}
10 & -1 \\
-1 & 15 \\
\end{pmatrix}$ \\
\bottomrule
\end{tabular}

Помогите Сидорову Вове!

\begin{enumerate}
  \item Заполните пропуск в таблице;
  \item Проверьте гипотезу $H_0$ тремя способами: с помощью $LR$, $LM$ и $W$ статистик.
\end{enumerate}

\item По 200 наблюдениям исследователь Иннокентий оценил модель логистической регрессии для вероятности
сдать экзамен по метрике:
\[
\hat \P(Y_i = 1) = \Lambda(1.5 + 0.3X_i - 0.4 D_i),
\]
где $Y_i$ — бинарная переменная равная 1, если студент сдал экзамен;
$X_i$ — количество часов подготовки студента; $D_i$ — бинарная переменная равная 1,
если студент пробовал пиццу «четыре сыра» в новой столовой.

Оценка ковариационной матрицы оценок коэффициентов имеет вид:

\[
\begin{pmatrix}
0.04 & -0.01 & 0 \\
-0.01 & 0.01 & 0 \\
0 & 0 & 0.09 \\
\end{pmatrix}
\]

\begin{enumerate}
  \item Проверьте гипотезу о том, что количество часов подготовки не влияет на вероятность сдать экзамен.
  \item Посчитайте предельный эффект увеличения каждого регрессора на вероятность сдать экзамен для студента не пробовавшего пиццу и готовившегося 24 часа.
Кратко, одной-двумя фразами, прокомментируйте смысл полученных цифр.
%  \item Постройте 95\%-й доверительный интервал для разницы вероятностей сдать экзамен двумя студентами, если
%  оба студента готовились 20 часов, однако один пробовал пиццу, а второй — нет.
  \item При каком значении $D_i$ предельный эффект увеличения $X_i$ на вероятность сдать экзамен максимален,
  если $X_i=20$?
\end{enumerate}



\newpage
\item Билл Гейтс оценил регрессию $\hat Y_i = 4 + 0.4 X_i + 0.9 W_i$, $RSS = 520$, $R^2 = 2/15$.

Про матрицу регрессоров $X$ известно, что
\[
X'X = \begin{pmatrix}
	29 & 0 & 0 \\
	0 & 50 & 10 \\
	0 & 10 & 80 \\
\end{pmatrix}
\]

\begin{enumerate}
 \item Сколько наблюдений было у Билла Гейтса?
 \item Найдите выборочное среднее переменных $X$, $W$ и $Y$.
 \item Постройте 95\%-й доверительный интервал для значения зависимой (индивидуальный прогноз) переменной при $X=1$ и $W=3$.
\end{enumerate}

\item Величины $X_1$, \ldots, $X_{100}$ распределены независимо и равномерно на отрезке $[-3a;5a]$.
	Оказалось, что $\sum_{i=1}^{100} X_i = 200$ и $\sum_{i=1}^{100}|X_i| = 500$.

\begin{enumerate}
  \item Оцените параметр $a$ методом моментов, используя момент $\E(X_i)$.
  \item Оцените параметр $a$ обобщённым методом моментов, используя моменты $\E(X_i)$ и $\E(|X_i|)$, и взвешивающую матрицу $W = \begin{pmatrix}
		  3 & 0 \\
		  0 & 64 \\
	  \end{pmatrix}$.
\end{enumerate}

\item Контора «Рога и Копыта» определяет
	необходимый запас рогов, $Y$, в зависимости от ожидаемых годовых продаж рогов,
	$X^e$, по формуле $Y_i = \beta_0 + \beta_1 X^e_i$. Коэффициенты $\beta_0$ и $\beta_1$ держатся в строжайшей тайне!

	В распоряжении холдинга «Рог изобилия» оказались данные
	по запасам рогов, $Y$, и фактическим годовым продажам рогов, $X$, конторы «Рога и Копыта». Фактические продажи рогов связаны с ожидаемыми уравнением $X_i = X_i^e + u_i$.

Исследователи холдинга хотят оценить секретные коэффициенты $\beta_0$ и $\beta_1$ с помощью простой регрессии $\hat Y_i = \hat \beta_0 + \hat \beta_1 X_i$ методом наименьших квадратов.

\begin{enumerate}

	\item Найдите предел по вероятности для $\hat \beta_1$ и $\hat\beta_0$. Являются ли оценки состоятельными?
	\item Если оценки не являются состоятельными, то по шагам опишите алгоритм получения состоятельных оценок. Если алгоритм требует получения дополнительных переменных, то укажите, какими свойствами они должны обладать.

\end{enumerate}


Векторы $(X_i^e, u_i)$ одинаково распределены при любом $i$ и независимы.



\end{enumerate}


\newpage
\lfoot{Вариант $\kappa$}
\rfoot{Тест}
\setcounter{question}{0}


\putyourname

\testtable

\checktable


\begin{question}
Стьюдентизированные остатки регрессии используются
\begin{answerlist}
  \item в тесте Саргана
  \item на первом шаге двухшагового МНК
  \item на первом шаге при проведении теста Годфельда-Квандта
  \item в методе главных компонент
  \item для выявления выбросов
\end{answerlist}
\end{question}




\begin{question}
Тест Саргана для проверки валидности инструментов можно использовать
только в том случае, если число инструментов
\begin{answerlist}
  \item меньше числа эндогенных переменных
  \item больше числа эндогенных переменных
  \item совпадает с числом эндогенных переменных
  \item совпадает с числом экзогенных переменных
  \item меньше числа экзогенных переменных
\end{answerlist}
\end{question}




\begin{question}
(1 балл) Какое условие НЕ требуется в теореме Гаусса-Маркова?
\begin{answerlist}[2]
  \item матрица регрессоров \(X\) имеет полный ранг
  \item модель \(Y=X\beta + \varepsilon\) правильно специфицирована
  \item случайные ошибки \(\varepsilon_i\) не коррелированы
  \item случайные ошибки \(\varepsilon_i\) имеют одинаковые дисперсии
  \item случайные ошибки \(\varepsilon_i\) нормально распределены
  \item нет верного ответа
\end{answerlist}
\end{question}

\begin{solution}
\begin{answerlist}
  \item Bad answer :(
  \item Bad answer :(
  \item Bad answer :(
  \item Bad answer :(
  \item Good answer :)
\end{answerlist}
\end{solution}


\begin{question}
(1 балл) Выборочная корреляция между регрессорами \(X\) и \(Z\) равна \(0.5\). В
регрессии \(\hat Y_i = \hat\beta_0 + \hat\beta_1 X_i + \hat\beta_2 Z_i\)
показатель \(VIF\) для регрессора \(X\) равен
\begin{answerlist}
  \item \(1/4\)
  \item \(4/3\)
  \item \(1/2\)
  \item \(2\)
  \item \(3/4\)
  \item нет верного ответа
\end{answerlist}
\end{question}

\begin{solution}
\begin{answerlist}
  \item Bad answer :(
  \item Good answer :)
  \item Bad answer :(
  \item Bad answer :(
  \item Bad answer :(
\end{answerlist}
\end{solution}

\newpage

\begin{question}
Стьюдентизированные остатки регрессии используются
\begin{answerlist}
  \item в методе главных компонент
  \item на первом шаге двухшагового МНК
  \item в тесте Саргана
  \item для выявления выбросов
  \item на первом шаге при проведении теста Годфельда-Квандта
\end{answerlist}
\end{question}




\begin{question}
По 52 наблюдениям студент построил две регрессии,
\(\hat Y_i = 3.1 + 0.8X_i\) и \(\hat X_i = -0.3 + 0.2Y_i\). Коэффициент
\(R^2_{adj}\) для первой регрессии примерно равен
\begin{answerlist}
  \item \(0.14\)
  \item \(0.16\)
  \item \(0.40\)
  \item \(0.32\)
  \item \(0.37\)
\end{answerlist}
\end{question}




\begin{question}
Использование скорректированных стандартных ошибок Уайта при
гомоскедастичности приведет к
\begin{answerlist}
  \item смещённости МНК оценок коэффициентов
  \item повышению эффективности МНК оценок коэффициентов
  \item получению состоятельной оценки дисперсии случайной ошибки
  \item понижению эффективности МНК оценок коэффициентов
  \item несостоятельности МНК оценок коэффициентов
\end{answerlist}
\end{question}

\begin{solution}
========
\end{solution}



\begin{question}
Рассмотрим модель множественной регрессии \(Y=X\beta+\varepsilon\), где
\(\hat Y = X\hat\beta\), \(e=Y-\hat Y\). Величина \(RSS\) --- это
квадрат длины вектора
\begin{answerlist}
  \item \(\hat Y - \bar Y\)
  \item \(\varepsilon\)
  \item \(\hat Y\)
  \item \(Y-\bar Y\)
  \item \(e\)
\end{answerlist}
\end{question}



\newpage

\begin{question}
Рассмотрим модель
\(Y_i= \beta_0 + \beta_z Z_{i} + \beta_w W_{i} + \varepsilon\) при
гетероскедастичности. Стандартная ошибка МНК-оценки, рассчитываемая по
формуле \(se(\hat\beta_w)=\sqrt{RSS \cdot (X'X)^{-1}_{33}/(n-3)}\),
является
\begin{answerlist}
  \item смещённой
  \item несмещённой
  \item состоятельной
  \item смещённой вниз
  \item смещённой вверх
\end{answerlist}
\end{question}




\begin{question}
Чебурашка оценил модель \(Y_i = \beta_0 + \beta_1 X_i + \varepsilon_i\),
а Крокодил Гена --- модель \(X_i = \gamma_0 + \gamma_1 Y_i + u_i\).
Оказалось, что \(\hat\gamma_1 = 0.25/\hat\beta_1\). Величина \(R^2\) в
регрессии Чебурашки равна
\begin{answerlist}
  \item \(1\)
  \item \(0.5\)
  \item \(0\)
  \item \(0.75\)
  \item \(0.25\)
\end{answerlist}
\end{question}

\begin{solution}
\(R^2 = \hat\beta_1 \cdot \hat\gamma_1\)
\end{solution}





\newpage
\rfoot{Задачи}
\putyourname


\begin{enumerate}
  \item Сидоров Вова оценивает два неизвестных параметра: $a$ — где стоят ракеты, $b$ — где продают конфеты.

  Вова оценил параметры методом максимального правдоподобия и получил оценки $\hat a = 1.5$, $\hat b = 2.5$.
  Затем Вова решил проверить гипотезу $H_0$: $a=1$ и $b=2$.

  Значения функции правдоподобия, градиента и оценённой информации Фишера в двух точках
  частично приведены в таблице:


\begin{tabular}{lccc}
\toprule
Точка & $\ell(a, b)$ & $(\ell'_a, \ell'_b)$ &  $\hat I_F$ \\
\midrule
$a=1.5$, $b=2.5$ & -200 &  ? &
$\begin{pmatrix}
16 & -1 \\
-1 & 20 \\
\end{pmatrix}$ \\
$a=1$, $b=2$ & -250 &  $(2, -1)$ &
$\begin{pmatrix}
10 & -1 \\
-1 & 15 \\
\end{pmatrix}$ \\
\bottomrule
\end{tabular}

Помогите Сидорову Вове!

\begin{enumerate}
  \item Заполните пропуск в таблице;
  \item Проверьте гипотезу $H_0$ тремя способами: с помощью $LR$, $LM$ и $W$ статистик.
\end{enumerate}

\item По 200 наблюдениям исследователь Иннокентий оценил модель логистической регрессии для вероятности
сдать экзамен по метрике:
\[
\hat \P(Y_i = 1) = \Lambda(1.5 + 0.3X_i - 0.4 D_i),
\]
где $Y_i$ — бинарная переменная равная 1, если студент сдал экзамен;
$X_i$ — количество часов подготовки студента; $D_i$ — бинарная переменная равная 1,
если студент пробовал пиццу «четыре сыра» в новой столовой.

Оценка ковариационной матрицы оценок коэффициентов имеет вид:

\[
\begin{pmatrix}
0.04 & -0.01 & 0 \\
-0.01 & 0.01 & 0 \\
0 & 0 & 0.09 \\
\end{pmatrix}
\]

\begin{enumerate}
  \item Проверьте гипотезу о том, что количество часов подготовки не влияет на вероятность сдать экзамен.
  \item Посчитайте предельный эффект увеличения каждого регрессора на вероятность сдать экзамен для студента не пробовавшего пиццу и готовившегося 24 часа.
Кратко, одной-двумя фразами, прокомментируйте смысл полученных цифр.
%  \item Постройте 95\%-й доверительный интервал для разницы вероятностей сдать экзамен двумя студентами, если
%  оба студента готовились 20 часов, однако один пробовал пиццу, а второй — нет.
  \item При каком значении $D_i$ предельный эффект увеличения $X_i$ на вероятность сдать экзамен максимален,
  если $X_i=20$?
\end{enumerate}



\newpage
\item Билл Гейтс оценил регрессию $\hat Y_i = 4 + 0.4 X_i + 0.9 W_i$, $RSS = 520$, $R^2 = 2/15$.

Про матрицу регрессоров $X$ известно, что
\[
X'X = \begin{pmatrix}
	29 & 0 & 0 \\
	0 & 50 & 10 \\
	0 & 10 & 80 \\
\end{pmatrix}
\]

\begin{enumerate}
 \item Сколько наблюдений было у Билла Гейтса?
 \item Найдите выборочное среднее переменных $X$, $W$ и $Y$.
 \item Постройте 95\%-й доверительный интервал для значения зависимой (индивидуальный прогноз) переменной при $X=1$ и $W=3$.
\end{enumerate}

\item Величины $X_1$, \ldots, $X_{100}$ распределены независимо и равномерно на отрезке $[-3a;5a]$.
	Оказалось, что $\sum_{i=1}^{100} X_i = 200$ и $\sum_{i=1}^{100}|X_i| = 500$.

\begin{enumerate}
  \item Оцените параметр $a$ методом моментов, используя момент $\E(X_i)$.
  \item Оцените параметр $a$ обобщённым методом моментов, используя моменты $\E(X_i)$ и $\E(|X_i|)$, и взвешивающую матрицу $W = \begin{pmatrix}
		  3 & 0 \\
		  0 & 64 \\
	  \end{pmatrix}$.
\end{enumerate}

\item Контора «Рога и Копыта» определяет
	необходимый запас рогов, $Y$, в зависимости от ожидаемых годовых продаж рогов,
	$X^e$, по формуле $Y_i = \beta_0 + \beta_1 X^e_i$. Коэффициенты $\beta_0$ и $\beta_1$ держатся в строжайшей тайне!

	В распоряжении холдинга «Рог изобилия» оказались данные
	по запасам рогов, $Y$, и фактическим годовым продажам рогов, $X$, конторы «Рога и Копыта». Фактические продажи рогов связаны с ожидаемыми уравнением $X_i = X_i^e + u_i$.

Исследователи холдинга хотят оценить секретные коэффициенты $\beta_0$ и $\beta_1$ с помощью простой регрессии $\hat Y_i = \hat \beta_0 + \hat \beta_1 X_i$ методом наименьших квадратов.

\begin{enumerate}

	\item Найдите предел по вероятности для $\hat \beta_1$ и $\hat\beta_0$. Являются ли оценки состоятельными?
	\item Если оценки не являются состоятельными, то по шагам опишите алгоритм получения состоятельных оценок. Если алгоритм требует получения дополнительных переменных, то укажите, какими свойствами они должны обладать.

\end{enumerate}


Векторы $(X_i^e, u_i)$ одинаково распределены при любом $i$ и независимы.



\end{enumerate}



\newpage
\lfoot{Вариант $\delta$}
\rfoot{Тест}
\setcounter{question}{0}


\putyourname

\testtable

\checktable


\begin{question}
Стьюдентизированные остатки регрессии используются
\begin{answerlist}
  \item в тесте Саргана
  \item на первом шаге двухшагового МНК
  \item на первом шаге при проведении теста Годфельда-Квандта
  \item в методе главных компонент
  \item для выявления выбросов
\end{answerlist}
\end{question}




\begin{question}
Тест Саргана для проверки валидности инструментов можно использовать
только в том случае, если число инструментов
\begin{answerlist}
  \item меньше числа эндогенных переменных
  \item больше числа эндогенных переменных
  \item совпадает с числом эндогенных переменных
  \item совпадает с числом экзогенных переменных
  \item меньше числа экзогенных переменных
\end{answerlist}
\end{question}




\begin{question}
(1 балл) Какое условие НЕ требуется в теореме Гаусса-Маркова?
\begin{answerlist}[2]
  \item матрица регрессоров \(X\) имеет полный ранг
  \item модель \(Y=X\beta + \varepsilon\) правильно специфицирована
  \item случайные ошибки \(\varepsilon_i\) не коррелированы
  \item случайные ошибки \(\varepsilon_i\) имеют одинаковые дисперсии
  \item случайные ошибки \(\varepsilon_i\) нормально распределены
  \item нет верного ответа
\end{answerlist}
\end{question}

\begin{solution}
\begin{answerlist}
  \item Bad answer :(
  \item Bad answer :(
  \item Bad answer :(
  \item Bad answer :(
  \item Good answer :)
\end{answerlist}
\end{solution}


\begin{question}
(1 балл) Выборочная корреляция между регрессорами \(X\) и \(Z\) равна \(0.5\). В
регрессии \(\hat Y_i = \hat\beta_0 + \hat\beta_1 X_i + \hat\beta_2 Z_i\)
показатель \(VIF\) для регрессора \(X\) равен
\begin{answerlist}
  \item \(1/4\)
  \item \(4/3\)
  \item \(1/2\)
  \item \(2\)
  \item \(3/4\)
  \item нет верного ответа
\end{answerlist}
\end{question}

\begin{solution}
\begin{answerlist}
  \item Bad answer :(
  \item Good answer :)
  \item Bad answer :(
  \item Bad answer :(
  \item Bad answer :(
\end{answerlist}
\end{solution}

\newpage

\begin{question}
Стьюдентизированные остатки регрессии используются
\begin{answerlist}
  \item в методе главных компонент
  \item на первом шаге двухшагового МНК
  \item в тесте Саргана
  \item для выявления выбросов
  \item на первом шаге при проведении теста Годфельда-Квандта
\end{answerlist}
\end{question}




\begin{question}
По 52 наблюдениям студент построил две регрессии,
\(\hat Y_i = 3.1 + 0.8X_i\) и \(\hat X_i = -0.3 + 0.2Y_i\). Коэффициент
\(R^2_{adj}\) для первой регрессии примерно равен
\begin{answerlist}
  \item \(0.14\)
  \item \(0.16\)
  \item \(0.40\)
  \item \(0.32\)
  \item \(0.37\)
\end{answerlist}
\end{question}




\begin{question}
Использование скорректированных стандартных ошибок Уайта при
гомоскедастичности приведет к
\begin{answerlist}
  \item смещённости МНК оценок коэффициентов
  \item повышению эффективности МНК оценок коэффициентов
  \item получению состоятельной оценки дисперсии случайной ошибки
  \item понижению эффективности МНК оценок коэффициентов
  \item несостоятельности МНК оценок коэффициентов
\end{answerlist}
\end{question}

\begin{solution}
========
\end{solution}



\begin{question}
Рассмотрим модель множественной регрессии \(Y=X\beta+\varepsilon\), где
\(\hat Y = X\hat\beta\), \(e=Y-\hat Y\). Величина \(RSS\) --- это
квадрат длины вектора
\begin{answerlist}
  \item \(\hat Y - \bar Y\)
  \item \(\varepsilon\)
  \item \(\hat Y\)
  \item \(Y-\bar Y\)
  \item \(e\)
\end{answerlist}
\end{question}



\newpage

\begin{question}
Рассмотрим модель
\(Y_i= \beta_0 + \beta_z Z_{i} + \beta_w W_{i} + \varepsilon\) при
гетероскедастичности. Стандартная ошибка МНК-оценки, рассчитываемая по
формуле \(se(\hat\beta_w)=\sqrt{RSS \cdot (X'X)^{-1}_{33}/(n-3)}\),
является
\begin{answerlist}
  \item смещённой
  \item несмещённой
  \item состоятельной
  \item смещённой вниз
  \item смещённой вверх
\end{answerlist}
\end{question}




\begin{question}
Чебурашка оценил модель \(Y_i = \beta_0 + \beta_1 X_i + \varepsilon_i\),
а Крокодил Гена --- модель \(X_i = \gamma_0 + \gamma_1 Y_i + u_i\).
Оказалось, что \(\hat\gamma_1 = 0.25/\hat\beta_1\). Величина \(R^2\) в
регрессии Чебурашки равна
\begin{answerlist}
  \item \(1\)
  \item \(0.5\)
  \item \(0\)
  \item \(0.75\)
  \item \(0.25\)
\end{answerlist}
\end{question}

\begin{solution}
\(R^2 = \hat\beta_1 \cdot \hat\gamma_1\)
\end{solution}





\newpage
\rfoot{Задачи}
\putyourname



\begin{enumerate}
  \item Сидоров Вова оценивает два неизвестных параметра: $a$ — где стоят ракеты, $b$ — где продают конфеты.

  Вова оценил параметры методом максимального правдоподобия и получил оценки $\hat a = 1.5$, $\hat b = 2.5$.
  Затем Вова решил проверить гипотезу $H_0$: $a=1$ и $b=2$.

  Значения функции правдоподобия, градиента и оценённой информации Фишера в двух точках
  частично приведены в таблице:


\begin{tabular}{lccc}
\toprule
Точка & $\ell(a, b)$ & $(\ell'_a, \ell'_b)$ &  $\hat I_F$ \\
\midrule
$a=1.5$, $b=2.5$ & -200 &  ? &
$\begin{pmatrix}
16 & -1 \\
-1 & 20 \\
\end{pmatrix}$ \\
$a=1$, $b=2$ & -250 &  $(2, -1)$ &
$\begin{pmatrix}
10 & -1 \\
-1 & 15 \\
\end{pmatrix}$ \\
\bottomrule
\end{tabular}

Помогите Сидорову Вове!

\begin{enumerate}
  \item Заполните пропуск в таблице;
  \item Проверьте гипотезу $H_0$ тремя способами: с помощью $LR$, $LM$ и $W$ статистик.
\end{enumerate}

\item По 200 наблюдениям исследователь Иннокентий оценил модель логистической регрессии для вероятности
сдать экзамен по метрике:
\[
\hat \P(Y_i = 1) = \Lambda(1.5 + 0.3X_i - 0.4 D_i),
\]
где $Y_i$ — бинарная переменная равная 1, если студент сдал экзамен;
$X_i$ — количество часов подготовки студента; $D_i$ — бинарная переменная равная 1,
если студент пробовал пиццу «четыре сыра» в новой столовой.

Оценка ковариационной матрицы оценок коэффициентов имеет вид:

\[
\begin{pmatrix}
0.04 & -0.01 & 0 \\
-0.01 & 0.01 & 0 \\
0 & 0 & 0.09 \\
\end{pmatrix}
\]

\begin{enumerate}
  \item Проверьте гипотезу о том, что количество часов подготовки не влияет на вероятность сдать экзамен.
  \item Посчитайте предельный эффект увеличения каждого регрессора на вероятность сдать экзамен для студента не пробовавшего пиццу и готовившегося 24 часа.
Кратко, одной-двумя фразами, прокомментируйте смысл полученных цифр.
%  \item Постройте 95\%-й доверительный интервал для разницы вероятностей сдать экзамен двумя студентами, если
%  оба студента готовились 20 часов, однако один пробовал пиццу, а второй — нет.
  \item При каком значении $D_i$ предельный эффект увеличения $X_i$ на вероятность сдать экзамен максимален,
  если $X_i=20$?
\end{enumerate}



\newpage
\item Билл Гейтс оценил регрессию $\hat Y_i = 4 + 0.4 X_i + 0.9 W_i$, $RSS = 520$, $R^2 = 2/15$.

Про матрицу регрессоров $X$ известно, что
\[
X'X = \begin{pmatrix}
	29 & 0 & 0 \\
	0 & 50 & 10 \\
	0 & 10 & 80 \\
\end{pmatrix}
\]

\begin{enumerate}
 \item Сколько наблюдений было у Билла Гейтса?
 \item Найдите выборочное среднее переменных $X$, $W$ и $Y$.
 \item Постройте 95\%-й доверительный интервал для значения зависимой (индивидуальный прогноз) переменной при $X=1$ и $W=3$.
\end{enumerate}

\item Величины $X_1$, \ldots, $X_{100}$ распределены независимо и равномерно на отрезке $[-3a;5a]$.
	Оказалось, что $\sum_{i=1}^{100} X_i = 200$ и $\sum_{i=1}^{100}|X_i| = 500$.

\begin{enumerate}
  \item Оцените параметр $a$ методом моментов, используя момент $\E(X_i)$.
  \item Оцените параметр $a$ обобщённым методом моментов, используя моменты $\E(X_i)$ и $\E(|X_i|)$, и взвешивающую матрицу $W = \begin{pmatrix}
		  3 & 0 \\
		  0 & 64 \\
	  \end{pmatrix}$.
\end{enumerate}

\item Контора «Рога и Копыта» определяет
	необходимый запас рогов, $Y$, в зависимости от ожидаемых годовых продаж рогов,
	$X^e$, по формуле $Y_i = \beta_0 + \beta_1 X^e_i$. Коэффициенты $\beta_0$ и $\beta_1$ держатся в строжайшей тайне!

	В распоряжении холдинга «Рог изобилия» оказались данные
	по запасам рогов, $Y$, и фактическим годовым продажам рогов, $X$, конторы «Рога и Копыта». Фактические продажи рогов связаны с ожидаемыми уравнением $X_i = X_i^e + u_i$.

Исследователи холдинга хотят оценить секретные коэффициенты $\beta_0$ и $\beta_1$ с помощью простой регрессии $\hat Y_i = \hat \beta_0 + \hat \beta_1 X_i$ методом наименьших квадратов.

\begin{enumerate}

	\item Найдите предел по вероятности для $\hat \beta_1$ и $\hat\beta_0$. Являются ли оценки состоятельными?
	\item Если оценки не являются состоятельными, то по шагам опишите алгоритм получения состоятельных оценок. Если алгоритм требует получения дополнительных переменных, то укажите, какими свойствами они должны обладать.

\end{enumerate}


Векторы $(X_i^e, u_i)$ одинаково распределены при любом $i$ и независимы.



\end{enumerate}



\newpage
\lfoot{Вариант $\omega$}
\rfoot{Тест}
\setcounter{question}{0}


\putyourname
\testtable

\checktable

\begin{question}
Стьюдентизированные остатки регрессии используются
\begin{answerlist}
  \item в тесте Саргана
  \item на первом шаге двухшагового МНК
  \item на первом шаге при проведении теста Годфельда-Квандта
  \item в методе главных компонент
  \item для выявления выбросов
\end{answerlist}
\end{question}




\begin{question}
Тест Саргана для проверки валидности инструментов можно использовать
только в том случае, если число инструментов
\begin{answerlist}
  \item меньше числа эндогенных переменных
  \item больше числа эндогенных переменных
  \item совпадает с числом эндогенных переменных
  \item совпадает с числом экзогенных переменных
  \item меньше числа экзогенных переменных
\end{answerlist}
\end{question}




\begin{question}
(1 балл) Какое условие НЕ требуется в теореме Гаусса-Маркова?
\begin{answerlist}[2]
  \item матрица регрессоров \(X\) имеет полный ранг
  \item модель \(Y=X\beta + \varepsilon\) правильно специфицирована
  \item случайные ошибки \(\varepsilon_i\) не коррелированы
  \item случайные ошибки \(\varepsilon_i\) имеют одинаковые дисперсии
  \item случайные ошибки \(\varepsilon_i\) нормально распределены
  \item нет верного ответа
\end{answerlist}
\end{question}

\begin{solution}
\begin{answerlist}
  \item Bad answer :(
  \item Bad answer :(
  \item Bad answer :(
  \item Bad answer :(
  \item Good answer :)
\end{answerlist}
\end{solution}


\begin{question}
(1 балл) Выборочная корреляция между регрессорами \(X\) и \(Z\) равна \(0.5\). В
регрессии \(\hat Y_i = \hat\beta_0 + \hat\beta_1 X_i + \hat\beta_2 Z_i\)
показатель \(VIF\) для регрессора \(X\) равен
\begin{answerlist}
  \item \(1/4\)
  \item \(4/3\)
  \item \(1/2\)
  \item \(2\)
  \item \(3/4\)
  \item нет верного ответа
\end{answerlist}
\end{question}

\begin{solution}
\begin{answerlist}
  \item Bad answer :(
  \item Good answer :)
  \item Bad answer :(
  \item Bad answer :(
  \item Bad answer :(
\end{answerlist}
\end{solution}

\newpage

\begin{question}
Стьюдентизированные остатки регрессии используются
\begin{answerlist}
  \item в методе главных компонент
  \item на первом шаге двухшагового МНК
  \item в тесте Саргана
  \item для выявления выбросов
  \item на первом шаге при проведении теста Годфельда-Квандта
\end{answerlist}
\end{question}




\begin{question}
По 52 наблюдениям студент построил две регрессии,
\(\hat Y_i = 3.1 + 0.8X_i\) и \(\hat X_i = -0.3 + 0.2Y_i\). Коэффициент
\(R^2_{adj}\) для первой регрессии примерно равен
\begin{answerlist}
  \item \(0.14\)
  \item \(0.16\)
  \item \(0.40\)
  \item \(0.32\)
  \item \(0.37\)
\end{answerlist}
\end{question}




\begin{question}
Использование скорректированных стандартных ошибок Уайта при
гомоскедастичности приведет к
\begin{answerlist}
  \item смещённости МНК оценок коэффициентов
  \item повышению эффективности МНК оценок коэффициентов
  \item получению состоятельной оценки дисперсии случайной ошибки
  \item понижению эффективности МНК оценок коэффициентов
  \item несостоятельности МНК оценок коэффициентов
\end{answerlist}
\end{question}

\begin{solution}
========
\end{solution}



\begin{question}
Рассмотрим модель множественной регрессии \(Y=X\beta+\varepsilon\), где
\(\hat Y = X\hat\beta\), \(e=Y-\hat Y\). Величина \(RSS\) --- это
квадрат длины вектора
\begin{answerlist}
  \item \(\hat Y - \bar Y\)
  \item \(\varepsilon\)
  \item \(\hat Y\)
  \item \(Y-\bar Y\)
  \item \(e\)
\end{answerlist}
\end{question}



\newpage

\begin{question}
Рассмотрим модель
\(Y_i= \beta_0 + \beta_z Z_{i} + \beta_w W_{i} + \varepsilon\) при
гетероскедастичности. Стандартная ошибка МНК-оценки, рассчитываемая по
формуле \(se(\hat\beta_w)=\sqrt{RSS \cdot (X'X)^{-1}_{33}/(n-3)}\),
является
\begin{answerlist}
  \item смещённой
  \item несмещённой
  \item состоятельной
  \item смещённой вниз
  \item смещённой вверх
\end{answerlist}
\end{question}




\begin{question}
Чебурашка оценил модель \(Y_i = \beta_0 + \beta_1 X_i + \varepsilon_i\),
а Крокодил Гена --- модель \(X_i = \gamma_0 + \gamma_1 Y_i + u_i\).
Оказалось, что \(\hat\gamma_1 = 0.25/\hat\beta_1\). Величина \(R^2\) в
регрессии Чебурашки равна
\begin{answerlist}
  \item \(1\)
  \item \(0.5\)
  \item \(0\)
  \item \(0.75\)
  \item \(0.25\)
\end{answerlist}
\end{question}

\begin{solution}
\(R^2 = \hat\beta_1 \cdot \hat\gamma_1\)
\end{solution}





\newpage
\rfoot{Задачи}
\putyourname



\begin{enumerate}
  \item Сидоров Вова оценивает два неизвестных параметра: $a$ — где стоят ракеты, $b$ — где продают конфеты.

  Вова оценил параметры методом максимального правдоподобия и получил оценки $\hat a = 1.5$, $\hat b = 2.5$.
  Затем Вова решил проверить гипотезу $H_0$: $a=1$ и $b=2$.

  Значения функции правдоподобия, градиента и оценённой информации Фишера в двух точках
  частично приведены в таблице:


\begin{tabular}{lccc}
\toprule
Точка & $\ell(a, b)$ & $(\ell'_a, \ell'_b)$ &  $\hat I_F$ \\
\midrule
$a=1.5$, $b=2.5$ & -200 &  ? &
$\begin{pmatrix}
16 & -1 \\
-1 & 20 \\
\end{pmatrix}$ \\
$a=1$, $b=2$ & -250 &  $(2, -1)$ &
$\begin{pmatrix}
10 & -1 \\
-1 & 15 \\
\end{pmatrix}$ \\
\bottomrule
\end{tabular}

Помогите Сидорову Вове!

\begin{enumerate}
  \item Заполните пропуск в таблице;
  \item Проверьте гипотезу $H_0$ тремя способами: с помощью $LR$, $LM$ и $W$ статистик.
\end{enumerate}

\item По 200 наблюдениям исследователь Иннокентий оценил модель логистической регрессии для вероятности
сдать экзамен по метрике:
\[
\hat \P(Y_i = 1) = \Lambda(1.5 + 0.3X_i - 0.4 D_i),
\]
где $Y_i$ — бинарная переменная равная 1, если студент сдал экзамен;
$X_i$ — количество часов подготовки студента; $D_i$ — бинарная переменная равная 1,
если студент пробовал пиццу «четыре сыра» в новой столовой.

Оценка ковариационной матрицы оценок коэффициентов имеет вид:

\[
\begin{pmatrix}
0.04 & -0.01 & 0 \\
-0.01 & 0.01 & 0 \\
0 & 0 & 0.09 \\
\end{pmatrix}
\]

\begin{enumerate}
  \item Проверьте гипотезу о том, что количество часов подготовки не влияет на вероятность сдать экзамен.
  \item Посчитайте предельный эффект увеличения каждого регрессора на вероятность сдать экзамен для студента не пробовавшего пиццу и готовившегося 24 часа.
Кратко, одной-двумя фразами, прокомментируйте смысл полученных цифр.
%  \item Постройте 95\%-й доверительный интервал для разницы вероятностей сдать экзамен двумя студентами, если
%  оба студента готовились 20 часов, однако один пробовал пиццу, а второй — нет.
  \item При каком значении $D_i$ предельный эффект увеличения $X_i$ на вероятность сдать экзамен максимален,
  если $X_i=20$?
\end{enumerate}



\newpage
\item Билл Гейтс оценил регрессию $\hat Y_i = 4 + 0.4 X_i + 0.9 W_i$, $RSS = 520$, $R^2 = 2/15$.

Про матрицу регрессоров $X$ известно, что
\[
X'X = \begin{pmatrix}
	29 & 0 & 0 \\
	0 & 50 & 10 \\
	0 & 10 & 80 \\
\end{pmatrix}
\]

\begin{enumerate}
 \item Сколько наблюдений было у Билла Гейтса?
 \item Найдите выборочное среднее переменных $X$, $W$ и $Y$.
 \item Постройте 95\%-й доверительный интервал для значения зависимой (индивидуальный прогноз) переменной при $X=1$ и $W=3$.
\end{enumerate}

\item Величины $X_1$, \ldots, $X_{100}$ распределены независимо и равномерно на отрезке $[-3a;5a]$.
	Оказалось, что $\sum_{i=1}^{100} X_i = 200$ и $\sum_{i=1}^{100}|X_i| = 500$.

\begin{enumerate}
  \item Оцените параметр $a$ методом моментов, используя момент $\E(X_i)$.
  \item Оцените параметр $a$ обобщённым методом моментов, используя моменты $\E(X_i)$ и $\E(|X_i|)$, и взвешивающую матрицу $W = \begin{pmatrix}
		  3 & 0 \\
		  0 & 64 \\
	  \end{pmatrix}$.
\end{enumerate}

\item Контора «Рога и Копыта» определяет
	необходимый запас рогов, $Y$, в зависимости от ожидаемых годовых продаж рогов,
	$X^e$, по формуле $Y_i = \beta_0 + \beta_1 X^e_i$. Коэффициенты $\beta_0$ и $\beta_1$ держатся в строжайшей тайне!

	В распоряжении холдинга «Рог изобилия» оказались данные
	по запасам рогов, $Y$, и фактическим годовым продажам рогов, $X$, конторы «Рога и Копыта». Фактические продажи рогов связаны с ожидаемыми уравнением $X_i = X_i^e + u_i$.

Исследователи холдинга хотят оценить секретные коэффициенты $\beta_0$ и $\beta_1$ с помощью простой регрессии $\hat Y_i = \hat \beta_0 + \hat \beta_1 X_i$ методом наименьших квадратов.

\begin{enumerate}

	\item Найдите предел по вероятности для $\hat \beta_1$ и $\hat\beta_0$. Являются ли оценки состоятельными?
	\item Если оценки не являются состоятельными, то по шагам опишите алгоритм получения состоятельных оценок. Если алгоритм требует получения дополнительных переменных, то укажите, какими свойствами они должны обладать.

\end{enumerate}


Векторы $(X_i^e, u_i)$ одинаково распределены при любом $i$ и независимы.



\end{enumerate}





\end{document}
