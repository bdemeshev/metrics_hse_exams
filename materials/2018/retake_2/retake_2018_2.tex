\documentclass[12pt, a4paper]{article}



% !TEX root = metrics_hse_exams.tex

\usepackage{libertine}

\usepackage{fontspec}
\usepackage{polyglossia}
\usepackage{csquotes}

\setmainlanguage{russian}
\setotherlanguages{english}

% download "Linux Libertine" OTF-fonts:
% http://www.linuxlibertine.org/index.php?id=91&L=1
% \setmainfont{Linux Libertine O} % or Helvetica, Arial, Cambria
% why do we need \newfontfamily:
% http://tex.stackexchange.com/questions/91507/
% \newfontfamily{\cyrillicfonttt}{Linux Libertine O}
% \newfontfamily{\cyrillicfont}{Linux Libertine O}
% \newfontfamily{\cyrillicfontsf}{Linux Libertine O}

\usepackage{etoolbox} % to use ifdef, must be after babel


\usepackage[paper=a4paper, top=13.5mm, bottom=13.5mm, left=16.5mm, right=13.5mm, includefoot]{geometry}

% \usepackage{etex} % расширение классического tex
% в частности позволяет подгружать гораздо больше пакетов, чем мы и займёмся далее

\usepackage{floatrow} % сильно вниз если сдвинуть - ругается!


\usepackage{makeidx} % для создания предметных указателей
\usepackage{verbatim} % для многострочных комментариев
%\usepackage[pdftex]{graphicx} % для вставки графики
% omit pdftex option if not using pdflatex


%\usepackage{dsfont} % шрифт для единички с двойной палочкой (для индикатора события)
\usepackage{bbm} % шрифт - двойные буквы


\usepackage[usenames, dvipsnames, svgnames, table, rgb]{xcolor}

\usepackage{colortbl}

\usepackage{comment} % для комментирования кусков

% пакет для тестов:
\usepackage[box, % запрет на перенос вопросов
nopage, % убираем колонтитулы страницы
insidebox, % ставим буквы в квадратики
separateanswersheet, % добавляем бланк ответов
nowatermark, % отсутствие надписи "Черновик"
indivanswers,  % показываем верные ответы
%answers,
lang=RU, % локализация слов "нет верных ответов", "вопрос" и тд
completemulti % добавлять "нет правильного ответа" во всех вопросах множественного выбора
]{automultiplechoice}


\usepackage{multicol}
\usepackage{multirow} % Слияние строк в таблице


\usepackage[colorlinks, hyperindex, unicode, breaklinks]{hyperref} % гиперссылки в pdf

\usepackage{dcolumn}

\usepackage{amssymb}
\usepackage{amsmath}
\usepackage{amsthm}
\usepackage{epsfig}
\usepackage{bm}
\usepackage{color}

\usepackage{todonotes} % для вставки в документ заметок о том, что осталось сделать
% \todo{Здесь надо коэффициенты исправить}
% \missingfigure{Здесь будет картина Последний день Помпеи}
% команда \listoftodos — печатает все поставленные \todo'шки


\usepackage{textcomp}  % Чтобы в формулах можно было русские буквы писать через \text{}

%\usepackage{embedfile} % Чтобы код LaTeXа включился как приложение в PDF-файл

\usepackage{subfigure} % для создания нескольких рисунков внутри одного

\usepackage{tikz, pgfplots} % язык для рисования графики из latex'a
\usetikzlibrary{trees} % прибамбас в нем для рисовки деревьев
\usetikzlibrary{arrows} % прибамбас в нем для рисовки стрелочек подлиннее
\usepackage{tikz-qtree} % прибамбас в нем для рисовки деревьев




\usepackage{enumitem}


%\embedfile[desc={Исходный LaTeX файл}]{\jobname.tex} % Включение кода в выходной файл
%\embedfile[desc={Стилевой файл}]{title_bor_utf8.tex}



% вместо горизонтальной делаем косую черточку в нестрогих неравенствах
\renewcommand{\le}{\leqslant}
\renewcommand{\ge}{\geqslant}
\renewcommand{\leq}{\leqslant}
\renewcommand{\geq}{\geqslant}

% делаем короче интервал в списках
\setlength{\itemsep}{0pt}
\setlength{\parskip}{0pt}
\setlength{\parsep}{0pt}

% свешиваем пунктуацию (т.е. знаки пунктуации могут вылезать за правую границу текста, при этом текст выглядит ровнее)
\usepackage{microtype}

% \usepackage{physics}
\usepackage{mathtools}
\DeclarePairedDelimiter{\abs}{\lvert}{\rvert}
\DeclarePairedDelimiter{\norm}{\lVert}{\rVert}

% более красивые таблицы
\usepackage{booktabs}
% заповеди из докупентации:
% 1. Не используйте вертикальные линни
% 2. Не используйте двойные линии
% 3. Единицы измерения - в шапку таблицы
% 4. Не сокращайте .1 вместо 0.1
% 5. Повторяющееся значение повторяйте, а не говорите "то же"

\DeclareMathOperator*{\argmin}{arg\,min}
\DeclareMathOperator*{\argmax}{arg\,max}
\DeclareMathOperator*{\amn}{arg\,min}
\DeclareMathOperator*{\amx}{arg\,max}
\DeclareMathOperator{\Var}{Var}
\DeclareMathOperator{\Cov}{Cov}
\DeclareMathOperator{\Corr}{Corr}

\DeclareMathOperator{\card}{card}
\DeclareMathOperator{\trace}{trace}
\DeclareMathOperator{\tr}{trace}
\DeclareMathOperator{\rank}{rank}

\DeclareMathOperator{\dist}{dist}
\DeclareMathOperator{\sign}{sign}
\DeclareMathOperator{\sgn}{sign}

\DeclareMathOperator{\col}{col}
\DeclareMathOperator{\row}{row}




\let \P\relax
\DeclareMathOperator{\P}{\mathbb{P}}


\newcommand \R{\mathbb R}
\newcommand \N{\mathbb N}
\newcommand \Z{\mathbb Z}

\newcommand \RR{\mathbb R}
\newcommand \NN{\mathbb N}
\newcommand \ZZ{\mathbb Z}


\newcommand{\SSR}{SS^{\text{res}}}
\newcommand{\SSE}{SS^{\text{expl}}}
\newcommand{\SST}{SST}

%на всякий случай пока есть
%теоремы без нумерации и имени
\newtheorem*{theorem}{Теорема}

%"Определения","Замечания"
%и "Гипотезы" не нумеруются
\newtheorem*{definition}{Определение}
%\newtheorem*{rem}{Замечание}
%\newtheorem*{conj}{Гипотеза}

%"Теоремы" и "Леммы" нумеруются
%по главам и согласованно м/у собой
%\newtheorem{theorem}{Теорема}
%\newtheorem{lemma}[theorem]{Лемма}

% Утверждения нумеруются по главам
% независимо от Лемм и Теорем
%\newtheorem{prop}{Утверждение}
%\newtheorem{cor}{Следствие}




\newcommand \useR{$[$R$]$ }

%% эконометрические сокращения
\DeclareMathOperator{\sVar}{sVar}
\DeclareMathOperator{\E}{\mathbb{E}}
\DeclareMathOperator{\sCov}{sCov}
\DeclareMathOperator{\sCorr}{sCorr}
\DeclareMathOperator \hVar{\widehat{\Var}}
\DeclareMathOperator \hCorr{\widehat{\Corr}}
\DeclareMathOperator \hCov{\widehat{\Cov}}

\DeclareMathOperator*{\plim}{plim}
\DeclareMathOperator{\Lin}{Lin}


\newcommand \hb{\hat{\beta}}
\newcommand \hs{\hat{s}}
\newcommand \hy{\hat{y}}
\newcommand \hY{\hat{Y}}
\newcommand \he{\hat{\varepsilon}}
\newcommand \vone{\vec{1}}
\newcommand \cN{\mathcal{N}}
\newcommand \e{\varepsilon}
\newcommand \z{z}


% \DeclareMathOperator{\tr}{tr}

%% лаг
\renewcommand{\L}{\mathrm{L}}

%% алая и белая розы
%% запускается так: \WhiteRose[масштаб], например, \WhiteRose[0.5]
\newcommand{\WhiteRose}[1]{\begingroup
\setbox0=\hbox{\includegraphics[scale=#1]{figures/Yorkshire_rose.pdf}}%
\parbox{\wd0}{\box0}\endgroup}

\newcommand{\RedRose}[1]{\begingroup
\setbox0=\hbox{\includegraphics[scale=#1]{figures/Lancashire_rose.pdf}}%
\parbox{\wd0}{\box0}\endgroup}

\newcommand{\WhiteRoseLine}{
\begin{center}
\WhiteRose{0.3} Версия Белой Розы \WhiteRose{0.3}
\end{center}}

\newcommand{\RedRoseLine}{
\begin{center}
\RedRose{0.3} Версия Алой Розы \RedRose{0.3}
\end{center}}
 % use local copy
\input{emetrix_preamble} % use local copy

% \usepackage{minted}

\unitlength=0.6mm


%%%%%%%%%%%%%%%%%% вставки
%%%%%%%%%%%%%%%%%%%%%%%%%%%%%%%%%%%%%%% Списки без уродских отступов
\newenvironment{enumerate*}{
\begin{enumerate}
  \setlength{\itemsep}{0pt}
  \setlength{\parskip}{0pt}
  \setlength{\parsep}{0pt}
}{\end{enumerate}}

\newenvironment{itemize*}{
\begin{itemize}
  \setlength{\itemsep}{0pt}
  \setlength{\parskip}{0pt}
  \setlength{\parsep}{0pt}
}{\end{itemize}}

\abovedisplayskip=0mm
\abovedisplayshortskip=0mm
\belowdisplayskip=0mm
\belowdisplayshortskip=0mm
%%%%%%%%%%%%%%%%%%%%%%%%%%%%%%%%%%%%%%%%%%%%%%%%%%%%%%%%%%%%%%%%%%%%%%

\newenvironment{centered}{%
  \begin{list}{}{%
    \topsep0pt
  }
  \centering
  \item[]
}
{\end{list}}
%%%%%%%%%%%%%%%%%%%%%%%%%%%%%%%%%%%%%%%%%%%%%%%%%%%%%%%%%%%%%%%%%%%%%%%%%

\usepackage[sorting=none, backend=biber]{biblatex}


\addbibresource{metrics_hse_exams.bib}


\AddEnumerateCounter{\asbuk}{\russian@alph}{щ} % для списков с русскими буквами
\setlist[enumerate, 2]{label=\asbuk*),ref=\asbuk*} % списки уровня 2 будут буквами а) б) ...



\begin{document}




\element{2017_fall_retake_1}{ % в фигурных скобках название группы вопросов
%  %\AMCnoCompleteMulti
\begin{questionmult}{1} % тип вопроса (questionmult — множественный выбор) и в фигурных — номер вопроса

Для набора панельных данных истинна спецификация модели со случайными эффектами, однако Вовочка оценивает модель с фиксированными эффектами. Вовочкины ценки коэффициентов $\beta$ окажутся

\begin{multicols}{2} % располагаем ответы в 3 колонки
\begin{choices} % опция [o] не рандомизирует порядок ответов
       \correctchoice{состоятельными и неэффективными}
       \wrongchoice{несостоятельными}
       \wrongchoice{состоятельными и эффективными}
       \wrongchoice{смещёнными и неэффективными}
       \wrongchoice{несмещёнными и эффективными}
    \end{choices}
   \end{multicols}
\end{questionmult}
}


\element{2017_fall_retake_1}{ % в фигурных скобках название группы вопросов
%  %\AMCnoCompleteMulti
\begin{questionmult}{2} % тип вопроса (questionmult — множественный выбор) и в фигурных — номер вопроса

Винни-Пух пытается понять, от каких переменных может зависеть его потребление мёда. Собрав 100 разных переменных, он построил 100 парных регрессий и проверил в них значимость коэффициента при каждой из переменных на уровне значимости 0.05. Пятачок понимает, что все 100 переменных не имеют никакого отношения к потреблению мёда и на самом деле просто случайные числа. Помогите Пятачку определить, сколько значимых переменных скорее всего найдёт Винни-Пух.

\begin{multicols}{3} % располагаем ответы в 3 колонки
\begin{choices} % опция [o] не рандомизирует порядок ответов
       \correctchoice{5}
       \wrongchoice{10}
       \wrongchoice{0}
       \wrongchoice{100}
       \wrongchoice{Не хватает данных для ответа}
    \end{choices}
   \end{multicols}
\end{questionmult}
}


\element{2017_fall_retake_1}{ % в фигурных скобках название группы вопросов
%  %\AMCnoCompleteMulti
\begin{questionmult}{3} % тип вопроса (questionmult — множественный выбор) и в фигурных — номер вопроса

Общеизвестно, что потребление мёда Винни-Пухом зависит, при этом положительно, от количества стихов, сочинённых им за день. К сожалению, Винни-Пух забывчив и всегда называет число сочинённых им стихов с ошибкой. Тогда оценка $\beta_1$ в регрессии $Honey_i = \beta_0 + \beta_1 Poems_i + \varepsilon_i $ окажется

%\begin{multicols}{3} % располагаем ответы в 3 колонки
\begin{choices}[o] % опция [o] не рандомизирует порядок ответов
       \correctchoice{Несостоятельной, заниженной}
       \wrongchoice{Несостоятельной, завышенной}
       \wrongchoice{Несостоятельной}
       \wrongchoice{Смещённой, но состоятельной}
       \wrongchoice{Несмещенной, но не состоятельной}
    \end{choices}
   %\end{multicols}
\end{questionmult}
}


% bad style
\element{2017_fall_retake_1}{ % в фигурных скобках название группы вопросов
%  %\AMCnoCompleteMulti
\begin{questionmult}{4} % тип вопроса (questionmult — множественный выбор) и в фигурных — номер вопроса

Из откровений внеземного разума известно, что эндогенности в модели $Y_i = \beta_0 + \beta_1 X_i + \varepsilon_i$ нет. Однако Вовочка нашёл хороший инструмент $z_i$, отвечающий всем требованиям, предъявляемым к инструментам, и оценил $\beta_1$ методом инструментальных переменных. Его оценка $\beta_1$ окажется

\begin{multicols}{3} % располагаем ответы в 3 колонки
\begin{choices} % опция [o] не рандомизирует порядок ответов
       \correctchoice{состоятельной, но не эффективной}
       \wrongchoice{состоятельной и эффективной}
       \wrongchoice{несостоятельной}
       \wrongchoice{состоятельной}
       \wrongchoice{невозможно сказать по имеющимся данным}
    \end{choices}
   \end{multicols}
\end{questionmult}
}





\element{2017_fall_retake_1}{ % в фигурных скобках название группы вопросов
%  %\AMCnoCompleteMulti
\begin{questionmult}{5} % тип вопроса (questionmult — множественный выбор) и в фигурных — номер вопроса

Рассмотрим процесс $Y_t = -0.2 Y_{t-1} + \varepsilon_t$. 5-ое значение автокорреляционной функции равно

\begin{multicols}{1} % располагаем ответы в 3 колонки
\begin{choices} % опция [o] не рандомизирует порядок ответов
       \correctchoice{ -0.00032 }
       \wrongchoice{ 0.00032 }
       \wrongchoice{ 0.2 }
       \wrongchoice{ -0.2 }
       \wrongchoice{ 0 }
    \end{choices}
   \end{multicols}
\end{questionmult}
}



\element{2017_fall_retake_1}{ % в фигурных скобках название группы вопросов
%  %\AMCnoCompleteMulti
\begin{questionmult}{6} % тип вопроса (questionmult — множественный выбор) и в фигурных — номер вопроса

Модель коррекции ошибками имеет следующий вид


\begin{multicols}{2} % располагаем ответы в 3 колонки
\begin{choices} % опция [o] не рандомизирует порядок ответов
       \correctchoice{$\Delta Y_t = \delta + \phi \Delta X_{t-1} - \gamma (Y_{t-1} - \alpha - \beta X_{t-1}) + \varepsilon_t$}
       \wrongchoice{$Y_t = \delta + \phi \Delta X_{t-1} - \gamma (Y_{t-1} - \alpha - \beta X_{t-1}) + \varepsilon_t$}
       \wrongchoice{$\Delta Y_t = \delta - \gamma (Y_{t-1} - \alpha - \beta X_{t-1}) + \varepsilon_t$}
       \wrongchoice{$Y_t = \delta - \gamma (Y_{t-1} - \alpha - \beta X_{t-1}) + \varepsilon_t$}
       \wrongchoice{$\Delta Y_t = \delta + \phi \Delta X_{t-1} - \gamma (Y_{t-1} ) + \varepsilon_t$}
    \end{choices}
   \end{multicols}
\end{questionmult}
}







\element{2017_fall_retake_1}{ % в фигурных скобках название группы вопросов
% \AMCnoCompleteMulti
\begin{questionmult}{7} % тип вопроса (questionmult — множественный выбор) и в фигурных — номер вопроса

Пусть $\varepsilon_t$ - белый шум. Тогда стационарным будет следующий процесс

\begin{multicols}{3} % располагаем ответы в 3 колонки
\begin{choices} % опция [o] не рандомизирует порядок ответов
       \correctchoice{$Y_t = \sum_{i = 0}^{10} \varepsilon_{t-i}$}
       \wrongchoice{$Y_t = 2018t + \varepsilon_t$}
       \wrongchoice{$Y_t = t \varepsilon_t$}
       \wrongchoice{$Y_t = Y_{t-1} - \varepsilon_t$}
       \wrongchoice{$Y_t = 2Y_{t-1} - \varepsilon_t$}
    \end{choices}
   \end{multicols}
\end{questionmult}
}





\element{2017_fall_retake_1}{ % в фигурных скобках название группы вопросов
%  %\AMCnoCompleteMulti
\begin{questionmult}{8} % тип вопроса (questionmult — множественный выбор) и в фигурных — номер вопроса

Процесс случайного блуждания с дрейфом описывается уравнением

\begin{multicols}{2} % располагаем ответы в 3 колонки
\begin{choices} % опция [o] не рандомизирует порядок ответов
       \correctchoice{$X_t = \mu + X_{t-1} + \varepsilon_t$}
       \wrongchoice{$X_t = \mu + 0.7 X_{t-1} + \varepsilon_t$}
       \wrongchoice{$X_t = X_{t-1} + \varepsilon_t$}
       \wrongchoice{$X_t = 0.7 X_{t-1} + \varepsilon_t$}
       \wrongchoice{$X_t = \mu + \varepsilon_t$}
    \end{choices}
   \end{multicols}
\end{questionmult}
}







\element{2017_fall_retake_1}{ % в фигурных скобках название группы вопросов
 %\AMCnoCompleteMulti
\begin{questionmult}{9} % тип вопроса (questionmult — множественный выбор) и в фигурных — номер вопроса

Если процесс является стационарным в широком смысле, то

\begin{multicols}{3} % располагаем ответы в 3 колонки
\begin{choices} % опция [o] не рандомизирует порядок ответов

       \wrongchoice{ Он является стационарным в узком смысле }
       \wrongchoice{ Для него выполняется основная гипотеза в тесте Дикки-Фуллера }
       \wrongchoice{ Его автоковариационная функция является постоянной }
       \wrongchoice{ Это белый шум }
       \wrongchoice{ Это AR процесс с корнями характеристического уравнения, меньшими 1 }
    \end{choices}
   \end{multicols}
\end{questionmult}
}




\element{2017_fall_retake_1}{ % в фигурных скобках название группы вопросов
 %\AMCnoCompleteMulti
\begin{questionmult}{10} % тип вопроса (questionmult — множественный выбор) и в фигурных — номер вопроса

При оценивании регрессионной модели $Y_t = a_0 + \sum_{j=1}^3 a_j X_{jt} + \varepsilon_t$ по 20 наблюдениям получено значение статистики Дарбина-Уотсона $d = 0.8$. При уровне значимости 1\% это свидетельствует о


\begin{multicols}{3} % располагаем ответы в 3 колонки
\begin{choices} % опция [o] не рандомизирует порядок ответов
       \correctchoice{Попадании в зону неопределенности}
       \wrongchoice{Положительной автокорреляции}
       \wrongchoice{Отрицательной автокорреляции}
       \wrongchoice{Отсутствии автокорреляции}
       \wrongchoice{Тест Дарбина-Уотсона вообще не проверяет наличие автокорреляции}
%       \wrongchoice{3 и 4}
    \end{choices}
   \end{multicols}
\end{questionmult}
}


\AMCnumero{1} % начинаем нумерацию вопросов с 1го

\cleargroup{all}
\copygroup[10]{2017_fall_retake_1}{all}
\insertgroup{all}

\subsubsection{Задачи}


\begin{enumerate}
\item Винни-Пух и Пятачок хотят оценить неизвестный параметр $a$ обобщённым методом моментов.
Винни-Пух наблюдает независимые и одинаково распределённые величины $X_i$
с математическим ожиданием $\E(X_i)=a+2$. А Пяточку известны независимые и
одинаково распределённые величины $Y_i$ с ожиданием $\E(Y_i)=a-1$.

По выборке из 100 величин $X_i$ и из 100 величин $Y_i$ оказалось,
что $\sum X_i = 500$ и $\sum Y_i = -50$.

\begin{enumerate}
\item Найдите оценку обобщённого метода моментов для единичной взвешивающей матрицы.
\item Оцените оптимальную взвешивающую матрицу, если дополнительно известно,
что $\Var(X_i)=a^2 + 25$, $\Var(Y_i)=16$, $\Cov(X_i, Y_i)=-4$.
\end{enumerate}

\item Кролик считает, что процесс $Y_t$ подчиняется уравнению:
\[
Y_t = 7 - 0.3 Y_{t-1} + u_t + 3u_{t-1},
\]
где процесс $u_t$ — белый шум c дисперсией $\Var(u_t)=\sigma^2_u$.

\begin{enumerate}
\item Есть ли у этого уравнения стационарное решение (является ли данный процесс стационарным)?
Если да, то найдите для него $\E(Y_t)$ и $\Var(Y_t)$.
%\item Приведите пример нестационарного решения данного уравнения.
\item Постройте 95\%-ый предиктивный интервал для $Y_{102}$, если дополнительно известно,
что $Y_{100}=3$, $u_{100}=-1$, а величины $u_t$ имеют нормальное распределение $\cN(0;16)$.
\end{enumerate}

\item
На основе квартальных данных с 2003 по 2008 год было получено следующее уравнение регрессии,
описывающее зависимость цены на товар $P_i$ от нескольких факторов:
\[
\hat P_t=3.5+0.4X_t+1.1W_t, ESS=70.4, RSS=40.5
\]
Когда в уравнение были добавлены фиктивные переменные, соответствующие первым
трем кварталам года $Q_1, Q_2, Q_3$, оцениваемая модель приобрела вид:
\[
P_t=\beta+\beta_X X_t+\beta_W W_t+\beta_{Q_{1t}} Q_{1t}+\beta_{Q_{2t}} Q_{2t}+\beta_{Q_{3t}} Q_{3t}+\e_t
\]
При этом величина $ESS = \sum (\hat P_t - \bar P)^2$ выросла до 86.4.

\begin{enumerate}
\item Аккуратно сформулируйте гипотезу об отсутствии сезонности.
\item На уровне значимости $5\%$ проверьте гипотезу о наличии сезонности.
\end{enumerate}



\item
Наблюдения представляют собой случайную выборку. Зависимые переменные $y_{t1}$ и $y_{t2}$ находятся из системы:

\[
\begin{cases}
y_{t1} = \beta_{11} + \beta_{12} x_t + \e_{t1} \\
y_{t2} = \beta_{21} + \beta_{22} z_t + \beta_{23} y_{t1} + \e_{t2}
\end{cases},
\]
где вектор ошибок $\e_t$ имеет совместное нормальное распределение
\[
\e_t \sim \cN\left(
\begin{pmatrix}
  0 \\
  0
\end{pmatrix};
\begin{pmatrix}
  1 & \rho \\
  \rho & 1
\end{pmatrix}
\right)
\]

Эконометресса Анжела оценивает с помощью МНК первое уравнение, а эконометресса Эвридика — второе.
\begin{enumerate}
\item Найдите пределы по вероятности получаемых ими оценок.
\item Будут ли оценки состоятельными?
\end{enumerate}

\item
Исследовательница Мишель строит оценку $\hb_{IV}$ в регрессии $y$ на $x$ с инструментом $z$.
Исследовательница Аграфена строит обычную МНК оценку в регрессии $\hat y = \hb_{x} x + \hb_{w} w$.

Выразите $w$ через $x$, $z$ и $y$ так, чтобы оценка $\hb_{IV}$ Мишель и оценка $\hb_{x}$ Аграфены совпали.



% Каково истинное значение склонности к потреблению, котороы Вы бы ожидали увидеть в этой модели?

\end{enumerate}


\end{document}
