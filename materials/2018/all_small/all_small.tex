\documentclass[12pt]{article}

\usepackage{tikz} % картинки в tikz
\usepackage{microtype} % свешивание пунктуации

\usepackage{array} % для столбцов фиксированной ширины

\usepackage{indentfirst} % отступ в первом параграфе

\usepackage{sectsty} % для центрирования названий частей
\allsectionsfont{\centering}

\usepackage{amsmath} % куча стандартных математических плюшек

\usepackage{comment}
\usepackage{amsfonts}

\usepackage[top=2cm, left=1cm, right=1cm, bottom=2cm]{geometry} % размер текста на странице

\usepackage{lastpage} % чтобы узнать номер последней страницы

\usepackage{enumitem} % дополнительные плюшки для списков
%  например \begin{enumerate}[resume] позволяет продолжить нумерацию в новом списке
\usepackage{caption}

\usepackage{hyperref} % гиперссылки

\usepackage{multicol} % текст в несколько столбцов


\usepackage{fancyhdr} % весёлые колонтитулы
\pagestyle{fancy}
\lhead{Проверочные}
\chead{2018-12-27}
\rhead{:)}
\lfoot{}
\cfoot{}
\rfoot{}
\renewcommand{\headrulewidth}{0.4pt}
\renewcommand{\footrulewidth}{0.4pt}



\usepackage{todonotes} % для вставки в документ заметок о том, что осталось сделать
% \todo{Здесь надо коэффициенты исправить}
% \missingfigure{Здесь будет Последний день Помпеи}
% \listoftodos --- печатает все поставленные \todo'шки


% более красивые таблицы
\usepackage{booktabs}
% заповеди из докупентации:
% 1. Не используйте вертикальные линни
% 2. Не используйте двойные линии
% 3. Единицы измерения - в шапку таблицы
% 4. Не сокращайте .1 вместо 0.1
% 5. Повторяющееся значение повторяйте, а не говорите "то же"


\usepackage{fontspec}
\usepackage{polyglossia}

\setmainlanguage{russian}
\setotherlanguages{english}

% download "Linux Libertine" fonts:
% http://www.linuxlibertine.org/index.php?id=91&L=1
\setmainfont{Linux Libertine O} % or Helvetica, Arial, Cambria
% why do we need \newfontfamily:
% http://tex.stackexchange.com/questions/91507/
\newfontfamily{\cyrillicfonttt}{Linux Libertine O}

\AddEnumerateCounter{\asbuk}{\russian@alph}{щ} % для списков с русскими буквами
\setlist[enumerate, 2]{label=\asbuk*),ref=\asbuk*}

%% эконометрические сокращения
\DeclareMathOperator{\Cov}{Cov}
\DeclareMathOperator{\Corr}{Corr}
\DeclareMathOperator{\Var}{Var}
\DeclareMathOperator{\E}{E}
\def \hb{\hat{\beta}}
\def \hs{\hat{\sigma}}
\def \htheta{\hat{\theta}}
\def \s{\sigma}
\def \hy{\hat{y}}
\def \hY{\hat{Y}}
\def \v1{\vec{1}}
\def \e{\varepsilon}
\def \he{\hat{\e}}
\def \z{z}
\def \hVar{\widehat{\Var}}
\def \hCorr{\widehat{\Corr}}
\def \hCov{\widehat{\Cov}}
\def \cN{\mathcal{N}}
\def \P{\mathbb{P}}


\begin{document}


\fbox{
  \begin{minipage}{42em}
    Имя, фамилия и номер группы:\vspace*{3ex}\par
    \noindent\dotfill\vspace{2mm}
  \end{minipage}
}

\begin{enumerate}
  \item Найдите SVD-разложение матрицы $
  \begin{pmatrix}
  2 & 1 & -1 \\
  2 & 1 & 0 \\
  \end{pmatrix}$
 \item Найдите дифференциал $d \exp(r^TAr+br)$, где $A^T=A$ и $b$ — это константы.
 \item Постройте регрессию вектора $y = (4,2,-2)^T$ на вектора $x=(2,1,-1)^T$ и $z=(-1,0,2)^T$ без константы. Будет ли в этой модели $TSS=RSS+ESS$?
 \item Известно, что $y=2x + 3z$. Винни-Пух построил регрессию $\hat y_i = \hat\beta_1 + 0.16 x_i$.
 Пятачок построил регрессию $\hat x_i = \hat \alpha_1 + 1\cdot y_i$.

 Помогите Сове найти коэффициент $\hat \gamma_2$ в регрессии $\hat y_i = \hat\gamma_1 + \hat\gamma_2 z_i$.

\item Величины $U_1$ и $U_2$ независимы и равномерны $U[0;1]$. Рассмотрим пару величин $Y_1 = R\cdot \cos \alpha$, $Y_2 = R\cdot \sin \alpha$, где $R=\sqrt{-2\ln U_1}$, а $\alpha = 2\pi U_2$.
\begin{enumerate}
  \item Найдите вероятностностную дифференциальную форму для пары $Y_1$, $Y_2$;
  \item Как называется совместный закон распределения $Y_1$ и $Y_2$?
  \item Верно ли, что $Y_1$ и $Y_2$ независимы?
\end{enumerate}

 \item Василий проецирует вектор $u=(u_1, u_2, u_3, u_4, u_5)'$ на линейную оболочку векторов $a=(1,1,2,2,2)'$ и $b=(1,3,3,3,3)'$. Вектор $u$ имеет стандартное многомерное нормальное распределение. Обозначим проекцию $\hat u$.
\begin{enumerate}
  \item Найдите $\hat u$ и $||\hat u||^2$.
  \item Как распределена величина $||\hat u||^2$?
\end{enumerate}

 \item Докажите, для любой линейной по $y$ несмещённой оценки $\hb^*$ выполнено условие $\Cov(\hb^* - \hb, \hb)=0$, где $\hb$ — оценка МНК.

\end{enumerate}

\end{document}
