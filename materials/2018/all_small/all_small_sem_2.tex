\documentclass[12pt]{article}

\usepackage{tikz} % картинки в tikz
\usepackage{microtype} % свешивание пунктуации

\usepackage{array} % для столбцов фиксированной ширины

\usepackage{indentfirst} % отступ в первом параграфе

\usepackage{sectsty} % для центрирования названий частей
\allsectionsfont{\centering}

\usepackage{amsmath} % куча стандартных математических плюшек

\usepackage{comment}
\usepackage{amsfonts}

\usepackage[top=2cm, left=1cm, right=1cm, bottom=2cm]{geometry} % размер текста на странице

\usepackage{lastpage} % чтобы узнать номер последней страницы

\usepackage{enumitem} % дополнительные плюшки для списков
%  например \begin{enumerate}[resume] позволяет продолжить нумерацию в новом списке
\usepackage{caption}

\usepackage{hyperref} % гиперссылки

\usepackage{multicol} % текст в несколько столбцов


\usepackage{fancyhdr} % весёлые колонтитулы
\pagestyle{fancy}
\lhead{Проверочные}
\chead{2019-03-30}
\rhead{:)}
\lfoot{}
\cfoot{}
\rfoot{}
\renewcommand{\headrulewidth}{0.4pt}
\renewcommand{\footrulewidth}{0.4pt}



\usepackage{todonotes} % для вставки в документ заметок о том, что осталось сделать
% \todo{Здесь надо коэффициенты исправить}
% \missingfigure{Здесь будет Последний день Помпеи}
% \listoftodos --- печатает все поставленные \todo'шки


% более красивые таблицы
\usepackage{booktabs}
% заповеди из докупентации:
% 1. Не используйте вертикальные линни
% 2. Не используйте двойные линии
% 3. Единицы измерения - в шапку таблицы
% 4. Не сокращайте .1 вместо 0.1
% 5. Повторяющееся значение повторяйте, а не говорите "то же"


\usepackage{fontspec}
\usepackage{polyglossia}

\setmainlanguage{russian}
\setotherlanguages{english}

% download "Linux Libertine" fonts:
% http://www.linuxlibertine.org/index.php?id=91&L=1
\setmainfont{Linux Libertine O} % or Helvetica, Arial, Cambria
% why do we need \newfontfamily:
% http://tex.stackexchange.com/questions/91507/
\newfontfamily{\cyrillicfonttt}{Linux Libertine O}

\AddEnumerateCounter{\asbuk}{\russian@alph}{щ} % для списков с русскими буквами
\setlist[enumerate, 2]{label=\asbuk*),ref=\asbuk*}

%% эконометрические сокращения
\DeclareMathOperator{\Cov}{Cov}
\DeclareMathOperator{\Corr}{Corr}
\DeclareMathOperator{\Var}{Var}
\DeclareMathOperator{\E}{E}
\def \hb{\hat{\beta}}
\def \hs{\hat{\sigma}}
\def \htheta{\hat{\theta}}
\def \s{\sigma}
\def \hy{\hat{y}}
\def \hY{\hat{Y}}
\def \v1{\vec{1}}
\def \e{\varepsilon}
\def \he{\hat{\e}}
\def \z{z}
\def \hVar{\widehat{\Var}}
\def \hCorr{\widehat{\Corr}}
\def \hCov{\widehat{\Cov}}
\def \cN{\mathcal{N}}
\def \P{\mathbb{P}}


\begin{document}


\fbox{
  \begin{minipage}{42em}
    Имя, фамилия и номер группы:\vspace*{3ex}\par
    \noindent\dotfill\vspace{2mm}
  \end{minipage}
}

\begin{enumerate}
\item Кот Василий предполагает, что количество
сметаны, оставшейся в холодильнике, $y_i$, зависит от количества гостей, пришедших к хозяевам на ужин $x_i$.

Василий предполагает линейную зависимость $y_i = \beta x_i + u_i$.

У кота Василия всего два наблюдения, $x_1$, $x_2$ и $y_1$, $y_2$. Ковариационная матрица
ошибок известна с точностью до сомножителя и равна $
\begin{pmatrix}
  4\sigma^2 & - \sigma^2 \\
  - \sigma^2 & 9\sigma^2 \\
\end{pmatrix}$.

Найдите эффективные оценки коэффициента $\beta$, линейные по $y_i$.

\item Кот Василий продолжает исследовать зависимость количества сметаны в холодильнике, $y_i$,
от количества пришедших гостей, $x_i$, $y_i = \beta x_i + u_i$.
На этот раз Василий предполагает независимые и
одинаково распределённые наблюдения с условной гетероскедастичностью $\Var(u_i |x_i) = h(x_i)$.

Василий собрал данные за 4 дня:

\begin{tabular}{ccccc}
\toprule
$y_i$ & 3 & 5 & 3 & 3 \\
$x_i$ & 2 & 4 & 2 & 1 \\
\bottomrule
\end{tabular}

\begin{enumerate}
  \item Найдите $\hb$, $se(\hb)$ предполагая гомоскедастичность.
  \item Найдите $se_{HC0}(\hb)$ и $se_{HC1}(\hb)$ предполагая гетероскедастичность неизвестной формы.
\end{enumerate}

\item Что такое False Negative? True positive?
\item Постройте кривую ROC и посчитайте AUC для модели бинарного выбора:

\begin{tabular}{cccccc}
\toprule
$y_i$ & 0 & 0 & 1 & 1 & 1 \\
$x_i$ & 0.1 & 0.9 & 0.8 & 0.7 & 0.6 \\
\bottomrule
\end{tabular}




\end{enumerate}

\end{document}
