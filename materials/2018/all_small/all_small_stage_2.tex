\documentclass[12pt]{article}

\usepackage{tikz} % картинки в tikz
\usepackage{microtype} % свешивание пунктуации

\usepackage{array} % для столбцов фиксированной ширины

\usepackage{indentfirst} % отступ в первом параграфе

\usepackage{sectsty} % для центрирования названий частей
\allsectionsfont{\centering}

\usepackage{amsmath} % куча стандартных математических плюшек

\usepackage{comment}
\usepackage{amsfonts}

\usepackage[top=2cm, left=1cm, right=1cm, bottom=2cm]{geometry} % размер текста на странице

\usepackage{lastpage} % чтобы узнать номер последней страницы

\usepackage{enumitem} % дополнительные плюшки для списков
%  например \begin{enumerate}[resume] позволяет продолжить нумерацию в новом списке
\usepackage{caption}

\usepackage{hyperref} % гиперссылки

\usepackage{multicol} % текст в несколько столбцов


\usepackage{fancyhdr} % весёлые колонтитулы
\pagestyle{fancy}
\lhead{Проверочные}
\chead{2019-06-19}
\rhead{:)}
\lfoot{}
\cfoot{}
\rfoot{}
\renewcommand{\headrulewidth}{0.4pt}
\renewcommand{\footrulewidth}{0.4pt}



\usepackage{todonotes} % для вставки в документ заметок о том, что осталось сделать
% \todo{Здесь надо коэффициенты исправить}
% \missingfigure{Здесь будет Последний день Помпеи}
% \listoftodos --- печатает все поставленные \todo'шки


% более красивые таблицы
\usepackage{booktabs}
% заповеди из докупентации:
% 1. Не используйте вертикальные линни
% 2. Не используйте двойные линии
% 3. Единицы измерения - в шапку таблицы
% 4. Не сокращайте .1 вместо 0.1
% 5. Повторяющееся значение повторяйте, а не говорите "то же"


\usepackage{fontspec}
\usepackage{polyglossia}

\setmainlanguage{russian}
\setotherlanguages{english}

% download "Linux Libertine" fonts:
% http://www.linuxlibertine.org/index.php?id=91&L=1
\setmainfont{Linux Libertine O} % or Helvetica, Arial, Cambria
% why do we need \newfontfamily:
% http://tex.stackexchange.com/questions/91507/
\newfontfamily{\cyrillicfonttt}{Linux Libertine O}

\AddEnumerateCounter{\asbuk}{\russian@alph}{щ} % для списков с русскими буквами
\setlist[enumerate, 2]{label=\asbuk*),ref=\asbuk*}

%% эконометрические сокращения
\DeclareMathOperator{\Cov}{Cov}
\DeclareMathOperator{\Corr}{Corr}
\DeclareMathOperator{\Var}{Var}
\DeclareMathOperator{\E}{E}
\def \hb{\hat{\beta}}
\def \hs{\hat{\sigma}}
\def \htheta{\hat{\theta}}
\def \s{\sigma}
\def \hy{\hat{y}}
\def \hY{\hat{Y}}
\def \v1{\vec{1}}
\def \e{\varepsilon}
\def \he{\hat{\e}}
\def \z{z}
\def \hVar{\widehat{\Var}}
\def \hCorr{\widehat{\Corr}}
\def \hCov{\widehat{\Cov}}
\def \cN{\mathcal{N}}
\def \P{\mathbb{P}}


\begin{document}


\fbox{
  \begin{minipage}{42em}
    Имя, фамилия и номер группы:\vspace*{3ex}\par
    \noindent\dotfill\vspace{2mm}
  \end{minipage}
}

\begin{enumerate}
  \item По выдаче софта выпишите уравнение ARIMA модели.
  \begin{verbatim}
  SARIMA(1,2,1)-(0,1,2)[3]
  ar1 = 0.7
  ma1 = 0.3
  sma1 = 0.1
  sma2 = 0.9
  \end{verbatim}
  \item Рассмотрим уравнение на $(y_t)$, где $(\varepsilon_t)$ — белый шум.
  \[
  y_t - 0.6y_{t-1} - 2y_{t-2} = \varepsilon_t + 3\varepsilon_{t-1}    
  \]
  \begin{enumerate}
      \item Сколько стационарных и нестационарных решений имеет уравнение?
      \item Являются ли стационарные решения заглядывающими в будущее относительно $(\varepsilon_t)$?
  \end{enumerate}
  \item Для ETS(AAN) модели известно, что $y_{100}=-3$, $\ell_{100}=-2$, $b_{100}=1$, 
  $\sigma=4$, $\alpha = 0.5$, $\beta = 0.1$.
  \begin{enumerate}
      \item Постройте 95\%-й доверительный интервал для $y_{102}$.
      \item Выпишите функцию правдоподобия для оценки данной модели по трём наблюдениям
      \footnote{Конечно, безумие оценивать ETS(AAN) по трём наблюдениям, смысл этого пункта в понимании, как устроена функция правдоподобия}.
  \end{enumerate}
  \item Запишите выражение без использования лага 
  \[
      \frac{L}{1-0.8L+0.15L^2} y_t
  \]
  \item Идентифицируйте параметры ARIMA модели по уравнению
  \[
     y_t - 0.6 y_{t-1} - 0.4 y_{t-2} = \varepsilon_t + 0.1 \varepsilon_{t-1}
  \]
  \item Рассмотрим стационарный процесс, удовлетворяющий уравнению
  \[
    y_t - 0.8y_{t-1} + 0.12y_{t-2} = 2 + \varepsilon_t + \varepsilon_{t-1}    
  \]

\begin{enumerate}
    \item Найдите $\E(y_t)$ и $\Var(y_t)$;
    \item Первые два значения автокорреляционной и частной автокорреляционной функций.
\end{enumerate}

\item Рассмотрим модель $y_i = \beta x_i + u_i$, где $u_i \sim \cN(0; \sigma^2)$ и независимы.
Известно, что $\sum y_i x_i = -2$, $\sum x_i^2 = 10$, $\sum y_i^2 = 20$, $n=100$.

С помощью трёх тестов, $LR$, $LM$, $W$, проверьте гипотезу $H_0$: $\beta = 2$ и, одновременно, $\sigma = 1$.

\item По 1000 наблюдений оценена логит-модель $\hat\P(y_i = 1) = \Lambda(1.5 + 0.03i)$.
Известно, что $se(\hb_1) = 0.8$, $se(\hb_2) = 0.9$, $\widehat{\Cov}(\hb_1, \hb_2) = -0.001$.

Постройте 95\%-й доверительный интервал для $\P(y_i=1 \mid i =10)$.
\newpage
\item Известны результаты оценивания методом максимального правдоподобия без ограничений и с двумя версиями ограничений

\begin{tabular}{@{}ccccccc@{}}
    \toprule
    Ограничение & $\hat a$ & $\hat b$  & $d\ell/da$ & $d\ell/db$ & $\ell$ & $\hat I$ \\
    \midrule
    нет & 6 & 9 & ? & ? & -1000 & $\begin{pmatrix}
    9 & -1 \\
    -1 & 4
    \end{pmatrix}$
    \\
    $a=7$ & ? & 8 & -3 & ? & -1100 & $\begin{pmatrix}
    8 & -1 \\
    -1 & 3
    \end{pmatrix}$
    \\
    $a+b=10$ & 3 & ? & 2 & 4 & -1300 & $\begin{pmatrix}
    7 & -1 \\
    -1 & 5
    \end{pmatrix}$
    \\
    \bottomrule
\end{tabular}

\begin{enumerate}
    \item С помощью $LM$ теста проверьте гипотезу $H_0$: $a=7$;
    \item С помощью $W$ теста проверьте гипотезу $H_0$: $a+b=10$;
\end{enumerate}

\item Рассмотрим модель $y_i = \beta_1 + \beta_2 x_i + u_i$, где $u_i \sim \cN(0; \sigma^2 i^2)$.
Найдите эффективную оценку $\hb_1$ и $\hb_2$.

\item Рассмотрим модель $y_i = \beta x_i + u_i$, где структура гетероскедастичности неизвестна.
Известно, что $x' = (1, 2, 2, -3)$, $y' = (1, 3, 3, -1)$. Найдите $se(\hb)$, $se_{HC0}(\hb)$, $se_{HC3}(\hb)$.

\item Проверьте выполнение критерия ранга и критерия порядка для каждого уравнения

\[
\begin{cases}
    Z_t = X_t + Y_t \\
    X_t = \beta_1 + \beta_2 R_t + \beta_3 W_t + \beta_4 P_t + \beta_5 S_t + u_t \\
    Y_t = \alpha_1 + \alpha_2 W_t + \alpha_3 Z_t + \alpha_4 Q_t + \varepsilon_t \\
\end{cases}
\]
Эндогенные переменные: $X_t$, $Y_t$, $Z_t$.

\item Рассмотрим результаты оценивания классификационного алгоритма:

\begin{tabular}{@{}cc@{}}
    \toprule
    $y_i$ & $\hat p_i$ \\
    \midrule
    0 & $0.6$ \\
    0 & $0.2$ \\
    $a$ & $0.9$ \\
    1 & $0.8$ \\
    1 & $0.5$ \\
    \bottomrule
\end{tabular}

\begin{enumerate}
    \item Нарисуйте ROC кривую и найдите AUC в зависимости от параметра $a$.
    \item Найдите вероятность того, что случайно выбранное наблюдение класса 1 будет иметь $\hat p$ выше,
    чем у случайно выбранного наблюдения класса 0.
\end{enumerate}
\end{enumerate}

\end{document}
