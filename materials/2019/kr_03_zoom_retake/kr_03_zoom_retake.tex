\documentclass[12pt]{article}

\usepackage{tikz} % картинки в tikz
\usepackage{microtype} % свешивание пунктуации

\usepackage{array} % для столбцов фиксированной ширины

\usepackage{indentfirst} % отступ в первом параграфе

\usepackage{sectsty} % для центрирования названий частей
\allsectionsfont{\centering}

\usepackage{amsmath, amssymb, amsthm} % куча стандартных математических плюшек

\usepackage{amsfonts}

\usepackage{comment}

\usepackage{physics} % \abs \norm, переопределяет \cos, \sin...

\usepackage[top=2cm, left=1.2cm, right=1.2cm, bottom=2cm]{geometry} % размер текста на странице

\usepackage{lastpage} % чтобы узнать номер последней страницы

\usepackage{enumitem} % дополнительные плюшки для списков
%  например \begin{enumerate}[resume] позволяет продолжить нумерацию в новом списке
\usepackage{caption}


\usepackage{hyperref} % гиперссылки

\usepackage{multicol} % текст в несколько столбцов


\usepackage{fancyhdr} % весёлые колонтитулы
\pagestyle{fancy}
\lhead{Эконометрика, НИУ-ВШЭ}
\chead{контрольная работа №3, шанс №2}
\rhead{2020-06-03}
\lfoot{Вариант $\xi$}
\cfoot{Ни пуха, ни пера!}
\rfoot{\thepage/3}
\renewcommand{\headrulewidth}{0.4pt}
\renewcommand{\footrulewidth}{0.4pt}



\usepackage{todonotes} % для вставки в документ заметок о том, что осталось сделать
% \todo{Здесь надо коэффициенты исправить}
% \missingfigure{Здесь будет Последний день Помпеи}
% \listoftodos - печатает все поставленные \todo'шки


% более красивые таблицы
\usepackage{booktabs}
% заповеди из докупентации:
% 1. Не используйте вертикальные линни
% 2. Не используйте двойные линии
% 3. Единицы измерения - в шапку таблицы
% 4. Не сокращайте .1 вместо 0.1
% 5. Повторяющееся значение повторяйте, а не говорите "то же"



\usepackage{fontspec}
\usepackage{polyglossia}

\setmainlanguage{russian}
\setotherlanguages{english}

% download "Linux Libertine" fonts:
% http://www.linuxlibertine.org/index.php?id=91&L=1
\setmainfont{Linux Libertine O} % or Helvetica, Arial, Cambria
% why do we need \newfontfamily:
% http://tex.stackexchange.com/questions/91507/
\newfontfamily{\cyrillicfonttt}{Linux Libertine O}

\AddEnumerateCounter{\asbuk}{\russian@alph}{щ} % для списков с русскими буквами
\setlist[enumerate, 2]{label=\asbuk*),ref=\asbuk*}

%% эконометрические сокращения
\let\P\relax
\DeclareMathOperator{\Cov}{\mathbb{C}ov}
\DeclareMathOperator{\Corr}{\mathbb{C}orr}
\DeclareMathOperator{\Var}{\mathbb{V}ar}
\DeclareMathOperator{\E}{\mathbb{E}}
\DeclareMathOperator{\P}{\mathbb{P}}
%\DeclareMathOperator{\tr}{trace}
\def \hb{\hat{\beta}}
\def \hs{\hat{\sigma}}
\def \htheta{\hat{\theta}}
\def \s{\sigma}
\def \hy{\hat{y}}
\def \hY{\hat{Y}}
\def \v1{\vec{1}}
\def \e{\varepsilon}
\def \he{\hat{\e}}
\def \z{z}
\def \hVar{\widehat{\Var}}
\def \hCorr{\widehat{\Corr}}
\def \hCov{\widehat{\Cov}}
\def \cN{\mathcal{N}}





\def \putyourname{\fbox{
    \begin{minipage}{42em}
      Фамилия, имя, номер группы:\vspace*{3ex}\par
      \noindent\dotfill\vspace{2mm}
    \end{minipage}
  }
}

\def \checktable{
\begin{minipage}{42em}
\begin{tabular}{|m{2cm}|m{2cm}|m{2cm}|m{2cm}|m{2cm}|}
\hline
Тест & 1 &  2 & 3 & Итого \\ \hline
&  &  &  & \\
 &  &   & & \\
 \hline
\end{tabular} $\longleftarrow$ для проверяющего!
\end{minipage}
}

\def \testtable{
\begin{minipage}{42em}
\vspace{4pt}

Ответы на тест:

\vspace{2pt}
\begin{tabular}{|m{1cm}|m{1cm}|m{1cm}|m{1cm}|m{1cm}|m{1cm}|m{1cm}|m{1cm}|m{1cm}|m{1cm}|}
\hline
1 & 2 &  3 & 4 & 5 & 6 & 7 & 8 & 9 & 10 \\ 
\hline
 &  &   &  &  &  &  &  &  &  \\ 
 &  &   &  &  &  &  &  &  &  \\ 
\hline
\end{tabular}
\end{minipage}

}





% [1][3] 1 = one argument, 3 = value if missing
% эта магия создаёт окружение answerlist
% именно в окружении answerlist записаны варианты ответов в подключаемых exerciseXX
% просто \begin{answerlist} сделает ответы в три столбца
% если ответы длинные, то надо в них руками сделать
% \begin{answerlist}[1] чтобы они шли в один столбец
\newenvironment{answerlist}[1][3]{
\begin{multicols}{#1}
\begin{enumerate}[label=\fbox{\emph{\Alph*}},ref=\emph{\alph*}]
}
{
\end{enumerate}
\end{multicols}
}

\newenvironment{answerlist1}{
\begin{enumerate}[label=\fbox{\emph{\Alph*}},ref=\emph{\alph*}]
}
{
\end{enumerate}
}



\excludecomment{solution} % without solutions

\theoremstyle{definition}
\newtheorem{question}{Вопрос}




\begin{document}

\checktable

\putyourname

\testtable

\subsection*{Тест}

\begin{comment}


\begin{question}
  Использование МНК к регрессии с бинарной зависимой переменной приведет к возникновению:
\begin{answerlist}
  \item Гетероскедастичности остатков
  \item Незначимости всей регрессии
  \item Мультиколлинеарности в модели
  \item Остатки модели будут иметь нормальное распределение  
  \item нет верного ответа
\end{answerlist}
\end{question}


\begin{question}
  В качестве функции правдоподобия для оценки ММП парной регрессионной модели выступает функция:
\begin{answerlist}[2]
  \item \( L( \beta_0 ;\beta_1)= \Pi_{i=1}^{n}{(p_{i}^{y_i}+(1-p_i)^{1-y_i})} \)
  \item \( L( \beta_0 ;\beta_1)= \Pi_{i=1}^{n}{(p_{i}^{y_i} \cdot (1-p_i)^{1-y_i})} \)
  \item \( L( \beta_0 ;\beta_1) = \Pi_{i=1}^{n}{(p_{i}^{y_i} – (1-p_i)^{1-y_i})} \)
  \item\( L( \beta_0 ;\beta_1) = \Pi_{i=1}^{n}{(1-p_i)^{1-y_i}}) \)
  \item нет верного ответа
\end{answerlist}
\end{question}



\begin{question}

Была оценена логистическая регрессия зависимости вероятности просрочки (1 — есть просрочка, 0 — нет) по кредиту в зависимости от возраста заемщика (Age):

\[
  \P(Y = 1) = F (Z), Z = -2,101 - 0,025 \cdot Age+ u 
\]
   
Абсолютная разница в вероятности просрочки для заемщика 36 лет и заемщика 55 лет, округленная до сотых, составляет:
\begin{answerlist}
  \item нельзя найти по имеющимся данным
  \item $0$, отсутствует
  \item $0.5$
  \item $0.02$
  \item $0.38$
  \item $0.05$
  \item нет верного ответа
\end{answerlist}
\end{question}


\begin{question}
Истинная зависимость имеет вид  \( Y_i = \beta_0 + \beta_1 \cdot Z_i + v_i \). 
При этом \( Z_i \) измеряется с ошибкой: \( Z^{obs}_i= Z_i + w_i \). 
Известно, что \( \beta_1= -0,4, \sigma^2_{w} = 6 \), \( \sigma^2_{z} = 3 \), 
\(\Cov(w_i, v_i) = 0\). 
Исследователь оценивает регрессию  $\hat Y_i = \hat\beta_0 + \hat\beta_1 \cdot Z^{obs}_i$. 
Предел по вероятности оценки $\hat\beta_1$ будет отличаться от истинного значения параметра на
\begin{answerlist}
  \item $-0.1(3)$
  \item $0.1(3)$
  \item $0.2(6)$
  \item $-0.2(6)$
  \item Оценка не будет асимптотически смещена
  \item нет верного ответа
\end{answerlist}
\end{question}


\begin{question}
Валидность инструмента \( Z_i \) в модели \( Y_i = \beta_0 + \beta_1 \cdot X_i + \epsilon_i \) обозначает:
\begin{answerlist}
  \item Инструмент \( Z_i \) коррелирует с \( X_i \)
  \item Инструмент \( Z_i \) не коррелирует с ошибкой
  \item Инструмент \( Z_i \) не коррелирует с \( X_i \)
  \item Инструмент \( Z_i \) коррелирует с ошибкой
  \item В модели есть эндогенность
  \item нет верного ответа
\end{answerlist}
\end{question}

\begin{question}
В линейной модели \( Y_i = \beta_0 + \beta_1 \cdot X_i + \epsilon_i \) регрессор \( X_i \) 
является эндогенным. 
Состоятельные оценки коэффициентов можно получить с помощью
\begin{answerlist}
  \item МНК
  \item Обобщенного МНК
  \item Взвешенного МНК   
  \item Метода инструментальных переменных  
  \item нет верного ответа
\end{answerlist}
\end{question}





\begin{question}
Известно, что \( Y_i = \beta_0 + \epsilon_i \), при этом \( \Var(\epsilon_i)=i^2 \). 
Какая из этих оценок  \( \beta_0 \) будет эффективной?

\begin{answerlist}[2]
  \item \( \frac{\sum_{i=1}^{n} \frac{Y_i}{i^2}}{\sum_{i=1}^{n} \frac{1}{i^2}} \)
  \item \( \frac{\sum_{i=1}^{n} \frac{Y_i}{i}}{\sum_{i=1}^{n} \frac{1}{i}} \)
  \item \( \overline{(\frac{1}{i}}) \)
  \item \( \overline{(\frac{1}{i^2})} \)
  \item \( \overline Y \)
  \item нет верного ответа
\end{answerlist}
\end{question}

\begin{question}
Какой из этих тестов на гетероскедастичность не требует выбора переменной, 
  по которой подозревается гетероскедастичность:
\begin{answerlist}
  \item Тест Уайта
  \item Тест Голдфелда-Куандта
  \item Тест Глейзера
  \item Тест Дарбина-Уотсона
  \item Тест Хаусмана
  \item нет верного ответа
\end{answerlist}
\end{question}


\begin{question}
При использовании МНК оценок параметров регрессионного уравнения и робастных ошибок в форме Уайта,
\begin{answerlist1}
\item Оценки \( \widehat{\beta} \)  будут состоятельными и неэффективными, доверительные интервалы, 
полученные по \( \widehat{Var_{HCE}} (\widehat{\beta}) \) можно использовать
\item Оценки \( \widehat{\beta} \) будут состоятельными и эффективными, доверительные интервалы, 
полученные по  \( \widehat{Var_{HCE}} (\widehat{\beta}) \)  можно использовать
\item Оценки \( \widehat{\beta} \)  будут несостоятельными и неэффективными, доверительные интервалы, 
полученные по  \( \widehat{Var_{HCE}} (\widehat{\beta}) \)   нельзя использовать
\item Оценки \( \widehat{\beta} \)  будут несостоятельными и неэффективными, доверительные интервалы, 
полученные по  \( \widehat{Var_{HCE}} (\widehat{\beta}) \)   можно использовать
\item Оценки \( \widehat{\beta} \)  будут состоятельными и эффективными, доверительные интервалы, 
полученные по  \( \widehat{Var_{HCE}} (\widehat{\beta}) \)   нельзя использовать
\item нет верного ответа
\end{answerlist1}
\end{question}

\begin{question}
Для модели \( Y=X\beta + \epsilon \) с \( \Var(\epsilon) = \Omega \neq \sigma_{\epsilon}^{2}I \)
где $I$ – единичная матрица,
эффективные оценки параметров $\beta$ можно получить с помощью критерия:
\begin{answerlist}
\item \( \min (Y – X \beta)' \Omega (Y – X \beta ) \) 
\item \( \min (Y – X \beta)' \Omega^{-1} (Y – X \beta ) \) 
\item \( \min (Y – X \beta)'  (Y – X \beta ) \) 
\item \( \min (Y – X \beta)^2 \) 
\item Минимизация невозможна
\item нет верного ответа
\end{answerlist}
\end{question}

\begin{question}
Если функция плотности удовлетворяет условиям регулярности, 
  то оценки метода максимального правдоподобия являются
\begin{answerlist}
\item несмещенными
\item несостоятельными
\item неотрицательными
\item инвариантными
\item равномерно распределенными
\item нет верного ответа
\end{answerlist}
\end{question}

\begin{question}
Тест Саргана для проверки валидности инструментов можно использовать, 
  если число эндогенных переменных среди объясняющих
\begin{answerlist}
\item Больше числа экзогенных переменных
\item Больше числа инструментов
\item Меньше числа инструментов
\item Не превышает 10
\item Меньше 3
\item нет верного ответа
\end{answerlist}
\end{question}


\begin{question}
  Если обобщенный метод моментов будет применен в случае, когда число используемых моментов 
  совпадает с числом оцениваемых параметров, то минимизируемая функция в точке оптимума 
\begin{answerlist}
\item Равна нулю
\item Больше нуля
\item Меньше нуля
\item Может быть как больше нуля, так и меньше нуля
\item Равна числу моментных тождеств
\item нет верного ответа
\end{answerlist}
\end{question}



\begin{question}
При проверке гипотезы \( H_0\): \(g(\beta)=0 \) 
для параметров модели \( Y_i=\beta_0+\beta_1 X_{i} + \beta_2 Z_i + \beta_3 W_i + \epsilon_i\), 
\(\epsilon \sim \cN(0,\sigma^2) \) 
с помощью теста множителей Лагранжа, необходимо знать оценки параметров
  \begin{answerlist}
  \item Регрессии на константу   
  \item Регрессии на все факторы, кроме константы   
  \item Только модели без ограничений
  \item Только модели с ограничениями
  \item Как модели с ограничениями, так и модели без ограничений 
  \item нет верного ответа
\end{answerlist}
  \end{question}

  
  \begin{question}
  Для проверки значимости коэффициента регрессии \( Y_i=\beta_0+\beta_1 X_{i} + \beta_2 Z_i + \beta_3 W_i + \epsilon_i\), 
  \(\epsilon \sim \cN(0,\sigma_\epsilon^2) \), 
  оцененной с помощью ММП по \( n \) наблюдениям, 
  исследователь использует LR статистику. Она имеет распределение
    \begin{answerlist}
    \item точно \( \cN(0, 1) \)
    \item асимптотически \( \cN(0, 1) \)
    \item $t_{n-4}$
    \item асимптотически $\chi^2_{1}$
    \item асимптотически $\chi^2_{n-4}$
    \item нет верного ответа
    \end{answerlist}
    \end{question}
    





\end{comment}







\newpage
\putyourname

\subsection*{Задачи}

\begin{enumerate}
\item  

Винни-Пух качает пресс на карантине, а также есть блины с мёдом и бутерброды со сгущёнкой. 
Количество подъемов туловища в разные дни равно $X_i$, $\E(X_i) = \mu$ и $\Var(X_i) = \sigma^2$.
Количество блинов с мёдом, $M_i$, равно  $M_i = X_i + u_i$, где $\E(u_i) = 0$ и $\Var(u_i)= 2\sigma^2$. 
Количество бутербродов со сгущенкой, $S_i$, равно  $S_i = X_i + w_i$, где $\E(w_i) = 0$ и $\Var(w_i)= 3\sigma^2$. 
Винни-Пух себя ведёт независимо в разные дни, кроме того, $X_i$, $w_i$ и $u_i$ независимы.


За 100 дней карантина Винни-Пух съел 600 бутербродов и 800 блинов. 

\begin{enumerate}
  \item Оцените ожидаемое количество подъемов туловища в день $\mu$ обобщённым методом моментов 
с матрицей весов 
$W= \begin{pmatrix}
1 & 0 \\
0 & 3 \\
\end{pmatrix}.$

\item Найдите оптимальную матрицу весов.
\end{enumerate}


\end{enumerate}



\end{document}