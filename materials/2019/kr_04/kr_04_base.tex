\documentclass[12pt]{article}

\usepackage{tikz} % картинки в tikz
\usepackage{microtype} % свешивание пунктуации

\usepackage{array} % для столбцов фиксированной ширины

\usepackage{indentfirst} % отступ в первом параграфе

\usepackage{sectsty} % для центрирования названий частей
\allsectionsfont{\centering}

\usepackage{amsmath, amssymb, amsthm} % куча стандартных математических плюшек

\usepackage{amsfonts}

\usepackage{comment}

\usepackage[top=2cm, left=1.2cm, right=1.2cm, bottom=2cm]{geometry} % размер текста на странице

\usepackage{lastpage} % чтобы узнать номер последней страницы

\usepackage{enumitem} % дополнительные плюшки для списков
%  например \begin{enumerate}[resume] позволяет продолжить нумерацию в новом списке
\usepackage{caption}


\usepackage{hyperref} % гиперссылки

\usepackage{multicol} % текст в несколько столбцов


\usepackage{fancyhdr} % весёлые колонтитулы
\pagestyle{fancy}
\lhead{Эконометрика, НИУ-ВШЭ}
\chead{контрольная работа №4}
\rhead{2020-06-xx}
\lfoot{Вариант $\xi$}
\cfoot{Ни пуха, ни пера!}
\rfoot{\thepage/3}
\renewcommand{\headrulewidth}{0.4pt}
\renewcommand{\footrulewidth}{0.4pt}



\usepackage{todonotes} % для вставки в документ заметок о том, что осталось сделать
% \todo{Здесь надо коэффициенты исправить}
% \missingfigure{Здесь будет Последний день Помпеи}
% \listoftodos - печатает все поставленные \todo'шки


% более красивые таблицы
\usepackage{booktabs}
% заповеди из докупентации:
% 1. Не используйте вертикальные линни
% 2. Не используйте двойные линии
% 3. Единицы измерения - в шапку таблицы
% 4. Не сокращайте .1 вместо 0.1
% 5. Повторяющееся значение повторяйте, а не говорите "то же"



\usepackage{fontspec}
\usepackage{polyglossia}

\setmainlanguage{russian}
\setotherlanguages{english}

% download "Linux Libertine" fonts:
% http://www.linuxlibertine.org/index.php?id=91&L=1
\setmainfont{Linux Libertine O} % or Helvetica, Arial, Cambria
% why do we need \newfontfamily:
% http://tex.stackexchange.com/questions/91507/
\newfontfamily{\cyrillicfonttt}{Linux Libertine O}

\AddEnumerateCounter{\asbuk}{\russian@alph}{щ} % для списков с русскими буквами
\setlist[enumerate, 2]{label=\asbuk*),ref=\asbuk*}

%% эконометрические сокращения
\let\P\relax
\DeclareMathOperator{\Cov}{\mathbb{C}ov}
\DeclareMathOperator{\Corr}{\mathbb{C}orr}
\DeclareMathOperator{\Var}{\mathbb{V}ar}
\DeclareMathOperator{\E}{\mathbb{E}}
\DeclareMathOperator{\P}{\mathbb{P}}
\DeclareMathOperator{\tr}{trace}
\def \hb{\hat{\beta}}
\def \hs{\hat{\sigma}}
\def \htheta{\hat{\theta}}
\def \s{\sigma}
\def \hy{\hat{y}}
\def \hY{\hat{Y}}
\def \v1{\vec{1}}
\def \e{\varepsilon}
\def \he{\hat{\e}}
\def \z{z}
\def \hVar{\widehat{\Var}}
\def \hCorr{\widehat{\Corr}}
\def \hCov{\widehat{\Cov}}
\def \cN{\mathcal{N}}





\def \putyourname{\fbox{
    \begin{minipage}{42em}
      Фамилия, имя, номер группы:\vspace*{3ex}\par
      \noindent\dotfill\vspace{2mm}
    \end{minipage}
  }
}

\def \checktable{
\begin{minipage}{42em}
\begin{tabular}{|m{2cm}|m{2cm}|m{2cm}|m{2cm}|m{2cm}|}
\hline
Тест & 1 &  2 & 3 & Итого \\ \hline
&  &  &  & \\
 &  &   & & \\
 \hline
\end{tabular} $\longleftarrow$ для проверяющего!
\end{minipage}
}

\def \testtable{
\begin{minipage}{42em}
\vspace{4pt}

Ответы на тест:

\vspace{2pt}
\begin{tabular}{|m{1cm}|m{1cm}|m{1cm}|m{1cm}|m{1cm}|m{1cm}|m{1cm}|m{1cm}|m{1cm}|m{1cm}|}
\hline
1 & 2 &  3 & 4 & 5 & 6 & 7 & 8 & 9 & 10 \\ 
\hline
 &  &   &  &  &  &  &  &  &  \\ 
 &  &   &  &  &  &  &  &  &  \\ 
\hline
\end{tabular}
\end{minipage}

}





% [1][3] 1 = one argument, 3 = value if missing
% эта магия создаёт окружение answerlist
% именно в окружении answerlist записаны варианты ответов в подключаемых exerciseXX
% просто \begin{answerlist} сделает ответы в три столбца
% если ответы длинные, то надо в них руками сделать
% \begin{answerlist}[1] чтобы они шли в один столбец
\newenvironment{answerlist}[1][3]{
\begin{multicols}{#1}
\begin{enumerate}[label=\fbox{\emph{\Alph*}},ref=\emph{\alph*}]
}
{
\end{enumerate}
\end{multicols}
}

\newenvironment{answerlist1}{
\begin{enumerate}[label=\fbox{\emph{\Alph*}},ref=\emph{\alph*}]
}
{
\end{enumerate}
}



\excludecomment{solution} % without solutions

\theoremstyle{definition}
\newtheorem{question}{Вопрос}




\begin{document}

\checktable

\putyourname

\testtable

\subsection*{Тест}

\begin{question}
Величина $Y_i$ зависит от регрессора $W_i$, $Y_i = 2 + 3W_i +\e_i$ 
и все предпосылки теоремы Гаусса-Маркова на $\e_i$ выполнены. 
Однако Илон Маск строит регрессию $\hat Y_i =\hb_0 + \hb_1 X_i$.
Какая будет дисперсия у $\hb_1$?
\begin{answerlist}
  \item $\sigma^2 /\sum (X_i - \bar X)^2$
  \item $\sigma^2 /\sum (W_i - \bar W)^2$  
  \item $\sigma^2 /\sum (W_i - \bar W)(X_i - \bar X)$  
  \item $\sigma^2 \sum (W_i - \bar W)^2/\sum (X_i - \bar X)^2$
  \item $\sigma^2 \sum (X_i - \bar X)^2/\sum (W_i - \bar W)^2$
  \item нет верного ответа
\end{answerlist}
\end{question}

\begin{question}
Илон Маск проверяет гипотезу $H_0$, состоящую из трёх уравнений, $\beta_1 + \beta_2 = 0$, 
$\beta_1 = -5$, $\beta_2 = +5$. Всего в модели оценивается 5 коэффициентов бета по 500 наблюдениям.
$F$-тест для проверки гипотезы $H_0$ имеет распределение:
\begin{answerlist}
    \item $F_{2, 495}$
    \item $F_{3, 495}$
    \item $F_{5, 497}$
    \item $F_{3, 497}$
    \item $F_{3, 492}$
    \item нет верного ответа
\end{answerlist}
\end{question}
  
\begin{question}
Рассмотрим модели $A$: $Y_i = \beta_0 + \beta_1 X_i + \e_i$, $B$: $Y_i = \beta_0 + \beta_1 \ln X_i + \e_i$ 
и $С$: $\ln Y_i = \beta_0 + \beta_1 X_i + \e_i$. Выберите верное утверждение.
\begin{answerlist}
      \item модели можно сравнить с помощью $AIC$
      \item модели можно сравнить с помощью попарных $F$-тестов
      \item модели можно сравнить с помощью $\hat \sigma^2$
      \item модели можно сравнить с помощью $R^2_{adj}$
      \item модели можно сравнить с помощью $RSS$
      \item нет верного ответа
\end{answerlist}
\end{question}
  

\begin{question}
Старик оценил модель $\hat Y_i = \underset{(1)}{3} + \underset{(2)}{6}X_i$ по 100 наблюдениям.
В скобках указаны стандартные ошибки, $\hat\sigma^2 = 5$. 
Однако Золотая Рыбка сказала Старику, что истинная константа $\beta_0$ равна 2.
Помогите Старику построить 95\%-ый доверительный интервал для индивидуального $Y$ (предиктивный интервал) для $X=1$
с учётом всей имеющейся информации. 

\begin{answerlist}
  \item $[0; 16]$
  \item $[1; 15]$
  \item $[2; 14]$
  \item $[3; 13]$
  \item $[4; 12]$
  \item нет верного ответа
\end{answerlist}
\end{question}
  



\newpage
\putyourname

\subsection*{Задачи}
\begin{enumerate}
\item Рассмотрим модель $Y_i = 5 + 6Y_{i-1} + 3X_{i-1} + u_i$, 
где $u_i$ независимы и нормально распределены $\cN(0; 9)$.

Известно, что $Y_{100} = 10$, $X_{100} = 2$ и $X_{t} = 1$ при $t\geq 101$.
\begin{enumerate}
  \item Постройте 95\%-ый предиктивный интервал (доверительный интервал для индивидуального значения) для $Y_{101}$.
  \item Постройте 95\%-ый предиктивный интервал (доверительный интервал для индивидуального значения) для $Y_{102}$.
  \item Постройте 95\%-ый предиктивный интервал (доверительный интервал для индивидуального значения) произвольного $Y_{t}$, $t\geq 101$.
\end{enumerate}

\item Рассмотрим систему одновременных уравнений
\[
\begin{cases}
D_i = \alpha_0 + \alpha_1 W_i + \alpha_2 T_i + \alpha_3 P_i + \alpha_5 Z_i + \e_i^D \\
S_i = \beta_0 + \beta_1 W_i + \beta_3 P_i + \beta_4 N_i + \e_i^S \\
W_i = \gamma_0 + \gamma_1 P_i + \e_i^W \\
D_i = S_i 
\end{cases}
\]
Эндогенными являются переменные $D_i$, $S_i$, $P_i$ и $W_i$. Все остальные переменные являются экзогенными.

\begin{enumerate}
  \item Проверьте идентифицируемость первого уравнения с помощью условия порядка.
  \item Проверьте идентифицируемость второго уравнения с помощью условия порядка и условия ранга.
  Если уравнение идентифицируемо, то кратко опишите способ идентификации.
\end{enumerate}


\item Рассмотрим систему одновременных уравнений
\[
\begin{cases}
A_i = \alpha_0 + \alpha_1 W_i + \alpha_3 P_i + \alpha_4 N_i + \e_i^A \\
B_i = \beta_0 + \beta_1 W_i + \beta_2 T_i + \beta_3 P_i + \beta_5 Z_i + \e_i^B \\
W_i = \gamma_0 + \gamma_1 P_i + \e_i^W \\
A_i = B_i 
\end{cases}
\]
Эндогенными являются переменные $A_i$, $B_i$, $P_i$ и $W_i$. Все остальные переменные являются экзогенными.

\begin{enumerate}
  \item Проверьте идентифицируемость второго уравнения с помощью условия порядка.
  \item Проверьте идентифицируемость первого уравнения с помощью условия порядка и условия ранга.
  Если уравнение идентифицируемо, то кратко опишите способ идентификации.
\end{enumerate}


\newpage

\item Рассмотрим модель 
\[
\begin{cases}
  a_t = a_{t-1} + b_{t-1} + u_t, \, u_t \sim \cN(0; 4) \\
  b_t = b_{t-1} + \nu_t, \, \nu_t \sim \cN(0;1) \\
\end{cases}
\]

Ошибки $(u_t)$ и $(\nu_t)$ независимы. Известно, что $a_{100}=100$, $b_{100}=2$. 
\begin{enumerate}
  \item Постройте точечный прогноз для $a_{101}$ и $a_{102}$.
  \item Постройте 95\%-й предиктивный интервал для $a_{101}$.
  \item Постройте 95\%-й предиктивный интервал для $a_{102}$.
\end{enumerate}


\item Рассмотрим модель 
\[
\begin{cases}
  a_t = a_{t-1} + b_{t-1} + u_t, \, u_t \sim \cN(0; 4) \\
  b_t = b_{t-1} + \nu_t, \, \nu_t \sim \cN(0;1) \\
\end{cases}
\]

Ошибки $(u_t)$ и $(\nu_t)$ независимы. Известно, что $a_{100}=200$, $b_{100}=1$. 
\begin{enumerate}
  \item Постройте точечный прогноз для $a_{101}$ и $a_{102}$.
  \item Постройте 95\%-й предиктивный интервал для $a_{101}$.
  \item Постройте 95\%-й предиктивный интервал для $a_{102}$.
\end{enumerate}

\item Рассмотрим систему одновременных уравнений
\[
\begin{cases}
A_i = \alpha_1 + \alpha_2 P_i + u_i^A \\
B_i = \beta_1 + \beta_2 P_i + \beta_3 W_i + u_i^B \\
A_i = B_i
\end{cases}
\]

Переменные $A_i$, $B_i$, $P_i$ являются эндогенными, $W_i$ — экзогенная. 

Известны результаты оценивания регрессий:
  \begin{align*}
    \hat A_i = 2 + 3 W_i \\
    \hat P_i = 3 - 2 W_i  
  \end{align*}

\begin{enumerate}
  \item Проверьте идентифицируемость каждого уравнения с помощью условия порядка. 
  \item Найдите оценки коэффициентов идентифицируемого уравнения. 
\end{enumerate}

\item Рассмотрим систему одновременных уравнений
\[
\begin{cases}
A_i = \alpha_1 + \alpha_2 P_i + \alpha_3 W_i + u_i^A \\
B_i = \beta_1 + \beta_2 P_i + u_i^B \\
A_i = B_i
\end{cases}
\]

Переменные $A_i$, $B_i$, $P_i$ являются эндогенными, $W_i$ — экзогенная. 

Известны результаты оценивания регрессий:
  \begin{align*}
    \hat A_i = 2 + 3 W_i \\
    \hat P_i = 3 - 2 W_i     
  \end{align*}

\begin{enumerate}
  \item Проверьте идентифицируемость каждого уравнения с помощью условия порядка. 
  \item Найдите оценки коэффициентов идентифицируемого уравнения. 
\end{enumerate}




\end{enumerate}



\end{document}