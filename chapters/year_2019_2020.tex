% !TEX root = ../metrics_hse_exams.tex

\subsection{ИП, вспомнить всё!}

\begin{enumerate}

    \item Сфорулируйте теорему о трёх перпендикулярах и обратную к ней. Нарисуйте картинку.
  
    \item Для матрицы
  $
    A=\begin{pmatrix}
    7 & 5  \\
    5 & 7  \\
    \end{pmatrix}
  $
  
    \begin{enumerate}
    \item Найдите собственные числа и собственные векторы матрицы;
    \item Найдите определитель $\det A$ и след $\tr A$;
   \item Известно, что $B = 2019A^{-1} + 2018I$, где $I$ — единичная матрица.
   Найдите собственные числа $B$, определитель $\det B$ и след $\tr B$.
  
    \end{enumerate}
  
  
    \item Блондинка Маша встретила 200 динозавров.
    Средний рост динозавров оказался равен 20 метров, а выборочное стандартное отклонение — 5 метров.
  
    \begin{enumerate}
      \item Постройте 95\% доверительный интервал для математического ожидания роста динозавра.
      \item На уровне значимости 1\% проверьте гипотезу о том, что математическое ожидание
      роста равно 22 метрам. Против альтернативной гипотезе о неравенстве.
      \item Укажите $P$-значение для теста в предыдущем пункте.
    \end{enumerate}
  
   \item В убийстве равновероятно виноват либо Джон, либо Билл. 
   На месте убийства найдена кровь убийцы, совпадающая с группой крови Джона. Такой группой крови обладает 10\% населения. Группа крови Билла неизвестна. 
   Какова услованя вероятность того, что Билл — убийца?
  
  \end{enumerate}
  

\subsection{Контрольная работа №1, базовая часть, 19.10.2019}

время написания: 1 час 20 минут

\subsubsection*{Тест}

% \setcounter{question}{0}

\begin{question}
Случайные величины $X$ и $Y$ независимы и имеют нормальное распределение с $\E(X) = 0$, $\Var(X)=1$, $\E(Y)=5$, $\Var(Y) = 4$. 
Величина $Z = 2X + Y$ имеет распределение
\begin{answerlist}
  \item $\cN(5;5)$
  \item $\cN(5;8)$
  \item $\chi^2_2$
  \item $t_2$
  \item $F_{1,1}$
  \item нет верного ответа
\end{answerlist}
\end{question}


\begin{question}
Оценка $T_n = T(X_1, X_2, \ldots, X_n)$ называется несмещённой оценкой параметра $\theta$, если
\begin{answerlist}[2]
  \item $\E(T_n) = T_n$
  \item $T_n = 0$
  \item $\lim_{n\to\infty} \P(|T_n - \theta|>\e) = 0$ при $\e>0$
  \item $\E(T_n) = 0$
  \item $\E(T_n) = \theta$
  \item нет верного ответа
\end{answerlist}
\end{question}



\begin{question}
Оценена регрессия $\hat Y = 300 + 6W$, где $R^2 = 0.85$ и $W_i = X_i / X_{i-1}$.

Если объясняющая переменная будет выражена в процентах, $\tilde W_i = 100(X_i - X_{i-1})/X_{i-1}$, то результаты оценки регрессии примут вид
\begin{answerlist}
  \item $\hat Y_i = 3 + 6 \tilde W_i$, $R^2= 0.85$
  \item $\hat Y_i = 300 + 600 \tilde W_i$, $R^2= 0.85$
  \item $\hat Y_i = 306 + 0.06 \tilde W_i$, $R^2= 0.85$
  \item $\hat Y_i = 300 + 6 \tilde W_i$, $R^2= 0.085$
  \item $\hat Y_i = 300 + 6 \tilde W_i$, $R^2= 0.85$
  \item нет верного ответа
\end{answerlist}
\end{question}


\begin{question}
Оценка ковариационной матрицы оценок коэффициентов регрессии $Y=X\beta + \e$ пропорциональна
\begin{answerlist}
  \item $(XX^T)^{-1}$
  \item $X^TX$
  \item $(X^TX)^{-1}$
  \item $XX^T$
  \item $X^TY$
  \item нет верного ответа
\end{answerlist}
\end{question}

\begin{question}
Среди предпосылок теоремы Гаусса-Маркова фигурирует условие
\begin{answerlist}
  \item $\E(Y_i)=0$
  \item $\e_i \sim \cN(0;\sigma^2)$
  \item $\E(\e_i)=1$
  \item $\Var(\e_i)=const$
  \item $\Var(\e_i)=1$
  \item нет верного ответа
\end{answerlist}
\end{question}


\begin{question}
Оценено уравнение парной регрессии $Y_i = \beta_0 + \beta_1 X_i + \e_i$, причём МНК-оценка
коэффициента $\beta_1$ равна 5, а стандартная ошибка оценки равна $0.25$.

Значение $t$-статистики для проверки гипотезы, что этот коэффициент равен 4, есть
\begin{answerlist}
  \item $-2$
  \item $4$
  \item $-4$
  \item $2$
  \item $20$
  \item нет верного ответа
\end{answerlist}
\end{question}

\begin{question}
P-значение при проверке некоторой гипотезы $H_0$ оказалось равно $0.002$.

Гипотеза $H_0$ не отвергается при уровне значимости
\begin{answerlist}
    \item 10\%
    \item 0.1\%  
    \item 1\%
  \item 5\%
  \item всех перечисленных
  \item нет верного ответа
\end{answerlist}
\end{question}


\begin{question}
Известно, что выборочный коэффициент корреляции между $X$ и $Y$ равен $0.25$. 
В регрессии $Y$ на константу и $X$ коэффициент $R^2$ равен
\begin{answerlist}
  \item $25$
  \item $0.25$
  \item $0.5$
  \item $0.0625$
  \item $\sqrt{0.5}$
  \item нет верного ответа
\end{answerlist}
\end{question}



\begin{question}
Исследователь оценил регрессию $\hat Y_i = 90 + 3 X_i$. Если увеличить переменную $X$ на 10\%,
а $Y$ — на 10 единиц, то 

\begin{answerlist}[2]
  \item оценка коэффициента $\beta_0$ уменьшится, а $\beta_1$ — увеличится
  \item оценка коэффициента $\beta_0$ увеличится, а $\beta_1$ — уменьшится
  \item оценки коэффициентов $\beta_0$, $\beta_1$ не изменятся
  \item оценки коэффициентов $\beta_0$, $\beta_1$ уменьшатся
  \item оценки коэффициентов $\beta_0$, $\beta_1$ увеличатся
  \item нет верного ответа
\end{answerlist}
\end{question}

\begin{question}
Исследователь оценил регрессию $\hat Y_i = \underset{(0.1)}{30} + \underset{(0.5)}{6} X_i$, причём $\sum_i (X_i - \bar X)^2=4$. Все предпосылки теоремы Гаусса-Маркова выполнены. 

В скобках приведены стандартные ошибки коэффициентов. 
Несмещённая оценка дисперсии ошибок регрессии равна 
\begin{answerlist}
  \item $0.25$
  \item $2$
  \item $1$
  \item $0.125$
  \item $2\sqrt{0.5}$
  \item нет верного ответа
\end{answerlist}
\end{question}


\subsubsection*{Задачи}

\begin{enumerate}
\item Найдите величины Q1, \ldots, Q10, пропущенные в таблицах: 

\begin{tabular}{lr} \toprule
Indicator & Value \\
\midrule
Multiple R          & Q1 \\
$R^2$     			& Q2 \\
Adjusted $R^2$     	& 0.54 \\
Standart error 		& Q3 \\
Observations		& 800 \\
\bottomrule
\end{tabular}\hspace{2cm}
\begin{tabular}{lrrrrr} \toprule
ANOVA     	 &  df 	& SS		& MS 	& F & Significance F \\
\midrule
Regression   & Q4   	& 42.9  	& 42.9	&  923 	& 	0	\\
Residual     & 798  	& 37.0  	& 46	&  	&     	\\
Total        & 799  	& Q5        &    	&  	&     	\\
\bottomrule
\end{tabular}


\begin{tabular}{rrrrrrr}
\toprule
 			& Coef. 	& St. error	& t-stat & P-value	& Lower 95\% 	& Upper 95\% \\
\midrule
Intercept 	& -25.24 	& 2.0 	& Q6 		& 0 	&  Q7		& -21.31 \\
totspan		& 1.7		& Q8    & 30.4 	    & 0 	&  Q9	    & Q10 \\
\bottomrule
\end{tabular}


\item Грета Тунберг оценила зависимость средней температуры на Земном шаре в градусах, $Y_i$, 
от количества своих постов в твиттере в соответствующий день, $X_i$, по 52 дням:

\[
\hat Y_i = \underset{(1.24)}{-1.53} + \underset{(0.12)}{0.14} X_i, \text{ где } \sum_i (X_i - \bar X)^2 =52.4 \text{ и } \bar X = 10
\]

\begin{enumerate}
\item (2 балла) Проверьте гипотезы о незначимости каждого коэффициента при уровне значимости $\alpha = 0.01$.
\item (2 балла) Проверьте гипотезу о равенстве углового коэффициента 2 при альтернативной гипотезе, что 
коэффициент больше 2 и уровне значимости $\alpha = 0.01$.
\item (1 балл) Найдите оценку дисперсии $\e_i$ в модели $Y_i = \beta_0 + \beta_1 X_i + \e_i$.
\item (3 балла) Постройте 95\%-ый доверительный интервал для индивидуального прогноза $Y$, если $X=10$.
    
    
\end{enumerate}
    
\item Рассмотрим парную регрессию $\hat Y_i = \hat\beta_0 + \hat\beta_1 X_i$.
    
\begin{enumerate}
\item (1 балл) Дайте определение коэффициента детерминации $R^2$.
\item (1 + 2 балла) В каких пределах может лежать $R^2$ в указанной парной регрессии? Докажите сформулированное утверждение.
\item (1 + 2 балла) Как связан коэффициент $R^2$ и выборочная корреляция зависимой переменной и регрессора? Докажите сформулированное утверждение. 
\end{enumerate}
    
    
% Примечание: при доказательстве можно использовать условия первого порядка, важно их аккуратно сформулировать. 



\end{enumerate}


\subsection{Контрольная работа №1, базовая часть — решения}

Тест: BECCD BBDBC

\begin{enumerate}
\item $Q1 = 0.73$, $Q2 = 0.537$, $Q3 = 0.215$, $Q4 = 1$, $Q5 = 799$, 
$Q6 = -12.62$, $Q7 = -29.16$, $Q8 = 0.056$, $Q9 = 1.59$, $Q10 = 1.81$


\item 
\begin{enumerate}
\item $t_{crit} = 2.58$, $t_{\hb_0} = -1.23$, $t_{\hb_1} = 1.17$, оба коэффициента не значимы.
\item $t_{crit} = 2.4$, $t_{obs} = -15.5$, гипотеза $H_0$ отвергается.
\item $\hat \sigma^2 = 0.12^2 \cdot 52.4 = 0.755$
\item $\hat Y = -1.53 + 1.4 = -0.13$, $[-0.13 - 2\cdot \sqrt{0.77}; -0.13 + 2\cdot \sqrt{0.77} ]$
\end{enumerate}

\item $R^2 \in [0;1]$, коэффициент $R^2$ равен квадрату выборочной корреляции между вектором $X$ и вектором $Y$.
\[
\frac{ESS}{TSS} = \left(\frac{\sum (Y_i - \bar Y)(X_i - \bar X)}{\sqrt{\sum (Y_i - \bar Y)^2 \sum (X_i - \bar X)^2}}\right)^2
\]
  
\end{enumerate}


\subsection{Контрольная работа №1, ИП часть, 19.10.2019}

время написания: 2 час 20 минут


Ровно 207 лет назад, 19 октября 1812 года, Наполеон покинул Москву! :)

\begin{enumerate}

\item Известно, что $A$ — постоянная симметричная матрица, $r$ — вектор и $f(r) = r^T Ar / r^Tr$.

\begin{enumerate}
\item Найдите $df$. 
\item Перепешите условие $df=0$ в виде $Ar = const \cdot r$. Докажите, что в любом экстремуме функции $f$ вектор $r$ будет собственным вектором матрицы $A$. 
\end{enumerate}


\item Рассмотрим модель $y = X\beta + u$ с неслучайными регрессорами $X$, $\E(u)=0$ и $\Var(u) = \sigma^2 I$.

\begin{enumerate}
\item Найдите $\Var(\hat y)$, $\Var(y - \hat y)$, $\E(y - \hat y)$. Укажите размеры каждой найденной матрицы. 
\end{enumerate}

Есть дополнительная тестовая выборка, $y^{new}$, $X^{new}$, и для неё $y^{new} = X^{new}\beta + u^{new}$ с $\E(u^{new})=0$ и $\Var(u^{new}) = \sigma^2 I$
В тестовой выборке $n^{new}$ наблюдений. Ошибки двух выборок некоррелированы, $\Cov(u, u^{new}) = 0$.
Прогнозы для тестовой выборки мы строим, используя старые оценки $\hat\beta$, то есть $\hat y^{new} = X^{new}\hat\beta$.

\begin{enumerate}[resume]
\item Найдите $\Var(\hat y^{new})$, $\Var(y^{new} - \hat y^{new})$, $\E(y^{new} - \hat y^{new})$. Укажите размеры каждой найденной матрицы. 
\end{enumerate}
    

\item В выборке всего 5 наблюдений. Исследователь Бонапарт оценивает парную регрессию $\hat y_i = \hat \beta_1 + \hat \beta_2 x_i$.
Однако, истинная модель имеет вид $y_i = 1 + 2 z_i + u_i$. 
Известно, что $u \sim \cN(0;\sigma^2 \cdot I)$, $x^T = (1, 2, 3, 4, 5)$.

\begin{enumerate}
\item Найдите $\E(\hb)$, $\Var(\hb)$, $\E(RSS)$, если $z^T = (2, 3, 4, 5, 6)$.
\item Найдите $\E(\hb)$, $\Var(\hb)$, $\E(RSS)$, если $z^T = (5, 4, 3, 2, 1)$.
\end{enumerate}

\item Грета Тунберг, Илон Маск и Джеки Чан выбрали ортогональный базис в $5$-мерном пространстве, 
$v_1$, $v_2$, $v_3$, $v_4$, $v_5$. Вектор $v_1$ — это вектор из единичек. 

Грета Тунберг построила регрессию $y$ на $v_1$, $v_2$ и $v_3$.
Илон Маск построил регрессию того же вектора $y$ на $v_1$, $v_4$, $v_5$.
Джеки Чан построил регрессию того же вектора $y$ на все элементы базиса.

\begin{enumerate}

\item Изобразите в $5$-мерном пространстве остатки и прогнозы всех трёх регрессий. 

\item Как связаны между собой $RSS$, $ESS$ и $TSS$ всех трёх регрессий?

\item Как связаны между собой оценки коэффициентов всех трёх регрессий?

\end{enumerate}

\item Рассмотрим модель $y_i = \beta_1 + \beta_2 x_i + u_i$ с неслучайным регрессором.

\begin{enumerate}
\item Максимально аккуратно сформулируйте теорему Гаусса-Маркова. С «если» и «то». 
С формальным пояснением к любому используемому статистическому термину. 
\end{enumerate}

Дополнительно известно, что $\beta_2 = 0$.
\begin{enumerate}[resume]
\item Найдите $\E(R^2)$.
\item Найдите $\E(R^2_{adj})$.
% \item Обобщите найденные ожидания на случай множественной регрессии с константой. 
\end{enumerate}

\end{enumerate}



\subsection{Контрольная работа №1, ИП часть — решения}



\subsection{Контрольная работа №2, базовая часть}


\subsubsection*{Тест}

% 1
\begin{question}
При проверке модели $Y_i = \beta_0 + \beta_1 X_{i1} + \ldots + \beta_k X_{ik} + \varepsilon_i$ на адекватность нулевая гипотеза имеет вид:
\begin{answerlist}
  \item $\beta_0 = \beta_1 = \ldots = \beta_k = 0$
  \item $\beta_1 = \ldots = \beta_k = 0$
  \item $\beta_0 = \beta_1 = \ldots = \beta_k $
  \item $\beta_1 = \ldots = \beta_k $
  \item $X_{i1} = \ldots = X_{ik} = 0$
  \item нет верного ответа
\end{answerlist}
\end{question}

% 2 done
\begin{question}
При проверке модели множественной регрессии $Y_i = \beta_0 + \beta_1 X_i + \beta_2 Z_i + \varepsilon_i$ на мультиколлинеарность оказалось, что $VIF_X = VIF_Z = 10.26$. Из этого можно сделать вывод, что 
\begin{answerlist}[2]
  \item МНК-оценки коэффициентов будут несмещённые, но не состоятельные
  \item МНК-оценки коэффициентов будут несмещённые и состоятельные
  \item МНК-оценки коэффициентов будут смещённые
  \item МНК-оценки коэффициентов не существуют
  \item оценк коэффициентов $\hb_1$ и $\hb_2$ будут незначимы
  \item нет верного ответа
\end{answerlist}
\end{question}


% 3 done
\begin{question}
МНК-оценка уравнения регрессии без константы имеет вид $\hat Y_i = 5 X_i$, 
оценка остаточной дисперсии равна $\hat\sigma^2 = 2$. 
Вектор регрессоров имеет вид $X= (-2, 1, 1)^T$. 
Все предпосылки теоремы Гаусса-Маркова выполнены.

Оценка дисперсии ошибки прогноза индивидуального значения $Y_{new}$ при $X_{new}=3$ равна
\begin{answerlist}
  \item $1$
  \item $5$
  \item $1.5$
  \item $2.25$
  \item $0.5$
  \item нет верного ответа
\end{answerlist}
\end{question}


% old 3
\begin{comment}
\begin{question}
МНК-оценка уравнения регрессии имеет вид $\hat Y_i = 1.56 + 0.21 X_i$, оценка остаточной дисперсии равна $\hat\sigma^2 = 0.04$. Транспонированная матрица регрессоров имеет вид 
$\begin{pmatrix}
1 & 1 & 1 & 1 & 1 \\
1 & 3 & 4 & 5 & 7 \\
\end{pmatrix}$. Отношение ширин 90\%-х доверительных интервалов для индивидуального значения $Y_{n+1}$ и ожидаемого значения $Y_{n+1}$ при $X_{n+1}=3$ равно
\begin{answerlist}
  \item $1$
  \item $5$
  \item $1.5$
  \item $2.25$
  \item $0.5$
  \item нет верного ответа
\end{answerlist}
\end{question}
\end{comment}


% 4 done
\begin{question}
Для переменных $X$, $Z$ и $W$ известны выборочные корреляции, $\hCorr(Z, W) = 0$, $\hCorr(X, W)=0$ и $\hCorr(X, Z) = -0.7$. 
Наибольшее собственное число выборочной корреляционной матрицы равно $1.7$. 

Первая главная компонента, выраженная через стандартизованные переменные, имеет вид
\begin{answerlist}
  \item $(x + z)/\sqrt{2}$
  \item $(x - z)/\sqrt{2}$
  \item $(x + z - \sqrt{2}w)/2$
  \item $(\sqrt{2} x + z +w)/2$
  \item $(x -\sqrt{2}z + w)/2$
  \item нет верного ответа
\end{answerlist}
\end{question}

% 5 done 
\begin{question}
По данным 28 фирм была оценена зависимость выпуска $Y$ от труда $L$ и капитала $K$ с помощью двух моделей: $\ln Y_i = \beta_0 + \beta_1 \ln L_i + \beta_2 \ln K_i + \varepsilon_i$ и $\ln Y_i = \beta_0 + \beta_1 \ln (L_i  K_i) + \varepsilon_i$. Коэффициенты детерминации равны $0.9$ и $0.8$. 

Значение $F$-статистики для проверки гипотезы о равенстве эластичностей по труду и по капиталу равно
\begin{answerlist}
  \item $25$
  \item $12.5$
  \item $20$
  \item $12$
  \item $0.04$
  \item нет верного ответа
\end{answerlist}
\end{question}

% 6 done
\begin{question}
Зависимость величины спроса $Y$ в штуках от цены $Р$ в тысячах рублей имеет вид $\ln \hat Y_i = 30 - 0.03 P_i$. Все коэффициенты регрессии значимы. Спрос снизится на 3\% при увеличении цены примерно на
\begin{answerlist}
  \item $100$ тысяч рублей
  \item $10$ тысяч рублей
  \item $1$ тысячу рублей
  \item $1\%$ 
  \item $10\%$
  \item нет верного ответа
\end{answerlist}
\end{question}

% 7 done
\begin{question}
Исследователь интересуется зависимостью среднегодового прироста работающих $E$ от прироста валового национального продукта $X$. Обе величины измеряются в процентах. Исследователь оценил три парных регрессии $E$ на $X$: по выборке для 30 развитых  стран, по выборке для 24 развивающихся стран и по общей выборке.

В этих регрессиях суммы квадратов остатков оказались равны $25$, $35$ и $120$.

Значение $F-$статистики для проверки гипотезы о том, что изучаемая зависимость едина для развитых и развивающихся стран, равно
\begin{answerlist}
    \item $25$
    \item $25.5$  
    \item $26$
  \item $26.5$
  \item $27$
  \item нет верного ответа
\end{answerlist}
\end{question}

% 8 done
\begin{question}
Известно, что выборочная корреляция между переменными $Z$ и $W$ равна $0.5$.
Величина $VIF_X$ в регрессии $Y_i = \beta_1 + \beta_2 X_i + \beta_3 Z_i + \beta_4 W_i + u_i$
\begin{answerlist}
  \item не менее 2
  \item не менее 4/3
  \item не более 2
  \item не более 4/3
  \item не может быть оценена ни сверху, ни снизу
  \item нет верного ответа
\end{answerlist}
\end{question}


% 9 done
\begin{question}
Исследователь исключил из регрессии со свободным членом переменную, $t$-статистика коэффициента при которой меньше 1 по модулю. 
Скорректированный коэффициент множественной детерминации при этом 
\begin{answerlist}[2]
  \item не увеличится
  \item не уменьшится
  \item может измениться в любую сторону
  \item станет равным нулю
  \item станет равным $1/n$, где $n$ — число наблюдений
  \item нет верного ответа
\end{answerlist}
\end{question}

% 10
\begin{question}
С помощью метода максимального правдоподобия оценили зависимость веса индивида $W_i$ от его роста $H_i$, $W_i^{(\theta)} = \beta_0 + \beta_1 H_i^{(\lambda)} + \varepsilon_i$. Здесь $W_i^{(\theta)}$ и $H_i^{(\lambda)}$ — вес и рост после преобразований Бокса-Кокса с параметрами $\theta$ и $\lambda$.
Были проверены три гипотезы:

\begin{tabular}{lrrr} \toprule
$H_0$ & $\theta=\lambda=-1$ & $\theta=\lambda=0$ & $\theta=\lambda=1$ \\
\midrule
$P$-значение & $0.00$ & $0.53$ & $0.00$ \\
\bottomrule
\end{tabular}

На основании имеющейся информации исследователю следует предпочесть модель
\begin{answerlist}
  \item $W_i = \beta_0 + \beta_1 H_i + \varepsilon_i$
  \item $\ln W_i = \beta_0 + \beta_1 H_i + \varepsilon_i$  
  \item $\ln W_i = \beta_0 + \beta_1 \ln H_i + \varepsilon_i$  
  \item $W_i = \beta_0 + \beta_1 \ln H_i + \varepsilon_i$  
  \item $W_i = \beta_0 + \beta_1 H_i + \beta_2 H_i^2 + \varepsilon_i$
  \item нет верного ответа
\end{answerlist}
\end{question}


\subsubsection*{Задачи}

\begin{enumerate}
\item Исследователь рассматривает уравнение зависимости расходов на питание (W) от доходов (Income), с учетом сезона. Переменная сезон (S) принимает следующие значения: 1 – зима, 2 – весна, 3 – лето и 4 – осень. Исследователь предполагает, что в каждый сезон может выполняться своя линейная зависимость. 

\begin{enumerate}
    \item (2 балла) Выпишите уравнение оцениваемой модели. Укажите смысл всех включенных в модель переменных.
    \item (2 балла) Как проверить гипотезу о единой линейной зависимости расходов на питание для всех сезонов? Выпишите аккуратно основную и альтернативную гипотезы, формулу расчета статистики и способ проверки.
\end{enumerate}

% q2 done
\item Рассмотрим модель $y_i = \beta_1 x_i + \beta_2 z_i + \varepsilon_i$ в стандартизированных переменных, оцениваемую по $n$ наблюдениям с помощью гребневой (ridge) регрессии с параметром регуляризации $\lambda$. 

\begin{enumerate}
    \item (2 балла) Выпишите условия первого порядка для задачи гребневой регрессии.
    \item (3 балла) Выведите оценки гребневой регрессии $\hb_1$ и $\hb_2$.
    \item (1 балл) Что произойдёт с оценками при $\lambda = 0$?
    \item (1 балл) Что произойдёт с оценками при $\lambda \to +\infty$?
\end{enumerate}



\item По 24 наблюдениям была оценена модель:

\[
\widehat{Y}_i=15-4Z_i+3W_i
\]

Известно, что случайные ошибки нормально распределены, $RSS=180$, и

\[
(X'X)^{-1} =
\begin{pmatrix}{}
  0.216 & -0.112 & -0.075 \\ 
  -0.112 & 0.119 & 0.021 \\ 
  -0.075 & 0.021 & 0.047 \\ 
  \end{pmatrix}
\]


\begin{enumerate}
\item (1 балл) Проверьте гипотезу $H_0: \beta_Z = 0$ против $H_a: \beta_Z \neq 0$ на уровне значимости~5\%.
\item (3 балла) Проверьте гипотезу $H_0: \beta_Z + \beta_W = 0$  против $H_a: \beta_Z + \beta_W \neq 0$ на уровне значимости~5\%.
\item (2 балла) Выпишите использованные при проверке гипотез предпосылки о случайных ошибках модели.
\end{enumerate}


\item Исследовательница Глафира изучает зависимость спроса на молоко от цены молока и дохода семьи. В её распоряжении есть следующие переменные:

\begin{itemize}
\item $price$ — цена молока в рублях за литр
\item $income$ — ежемесячный доход семьи в тысячах рублей
\item $milk$ — расходы семьи на молоко за последние семь дней в рублях
\end{itemize}

В данных указано, проживает ли семья в сельской или городской местности. Поэтому Глафира оценила три регрессии: (All) — по всем данным, (Urban) — по городским семьям, (Rural) — по сельским семьям.


\begin{tabular}{lD{.}{.}{3}D{.}{.}{3}D{.}{.}{3}}
\toprule
 & 
\multicolumn{1}{c}{(All)} & 
\multicolumn{1}{c}{(Urban)} & 
\multicolumn{1}{c}{(Rural)}\\
\midrule
(Intercept)    & -1.765       & -4.059       & -0.155      \\
               & (4.943)      & (6.601)      & (7.812)     \\
income         &  0.308^{***} &  0.341^{***} &  0.281^{***}\\
               & (0.052)      & (0.072)      & (0.079)     \\
price          & -0.383^{*}   & -0.352       & -0.391      \\
               & (0.161)      & (0.253)      & (0.221)     \\
\midrule
R-squared      &    0.304 &   0.356 &    0.273\\
adj. R-squared &    0.290 &   0.325 &    0.245\\
sigma          &    4.912 &   4.857 &    5.036\\
F              &   21.216 &  11.593 &    9.744\\
P-value        &    0.000 &   0.000 &    0.000\\
RSS            & 2340.080 & 990.839 & 1318.741\\
n observations &  100     &  45     &   55    \\
\bottomrule
\end{tabular}

Выборочная ковариационная матрица регрессоров по полной выборке имеет вид:

\begin{tabular}{rrrr}
  \hline
 & price & income & milk \\ 
  \hline
price & 9.45 & -1.73 & -4.15 \\ 
  income & -1.73 & 90.19 & 28.43 \\ 
  milk & -4.15 & 28.43 & 33.98 \\ 
   \hline
\end{tabular}



\begin{enumerate}
\item (1 балл) Проверьте значимость в целом регрессии (All) на 5\%-ом уровне значимости.
\item (2 балла) На 5\%-ом уровне значимости проверьте гипотезу, что зависимость спроса на молоко является единой для городской и сельской местности.
\item (3 балла) Разложите коэффициент детерминации $R^2$ в модели (All) в сумму эффектов переменных $income$ и $price$.
\end{enumerate}



\end{enumerate}

\subsection{Контрольная работа 3}

\begin{question}
  Использование МНК к регрессии с бинарной зависимой переменной приведет к возникновению:
\begin{answerlist}
  \item Гетероскедастичности остатков
  \item Незначимости всей регрессии
  \item Мультиколлинеарности в модели
  \item Остатки модели будут иметь нормальное распределение  
  \item нет верного ответа
\end{answerlist}
\end{question}


\begin{question}
  В качестве функции правдоподобия для оценки ММП парной регрессионной модели выступает функция:
\begin{answerlist}[2]
  \item \( L( \beta_0 ;\beta_1)= \Pi_{i=1}^{n}{(p_{i}^{y_i}+(1-p_i)^{1-y_i})} \)
  \item \( L( \beta_0 ;\beta_1)= \Pi_{i=1}^{n}{(p_{i}^{y_i} \cdot (1-p_i)^{1-y_i})} \)
  \item \( L( \beta_0 ;\beta_1) = \Pi_{i=1}^{n}{(p_{i}^{y_i} – (1-p_i)^{1-y_i})} \)
  \item\( L( \beta_0 ;\beta_1) = \Pi_{i=1}^{n}{(1-p_i)^{1-y_i}}) \)
  \item нет верного ответа
\end{answerlist}
\end{question}



\begin{question}
Была оценена логистическая регрессия зависимости вероятности просрочки (1 — есть просрочка, 0 — нет) по кредиту в зависимости от возраста заемщика (Age):

\[
  \P(Y = 1) = F (Z), Z = -2,101 - 0,025 \cdot Age+ u 
\]
   
Абсолютная разница в вероятности просрочки для заемщика 36 лет и заемщика 55 лет, округленная до сотых, составляет:
\begin{answerlist}
  \item нельзя найти по имеющимся данным
  \item $0$, отсутствует
  \item $0.5$
  \item $0.02$
  \item $0.38$
  \item $0.05$
  \item нет верного ответа
\end{answerlist}
\end{question}


\begin{question}
Истинная зависимость имеет вид  \( Y_i = \beta_0 + \beta_1 \cdot Z_i + v_i \). 
При этом \( Z_i \) измеряется с ошибкой: \( Z^{obs}_i= Z_i + w_i \). 
Известно, что \( \beta_1= -0,4, \sigma^2_{w} = 6 \), \( \sigma^2_{z} = 3 \), 
\(\Cov(w_i, v_i) = 0\). 
Исследователь оценивает регрессию  $\hat Y_i = \hat\beta_0 + \hat\beta_1 \cdot Z^{obs}_i$. 
Предел по вероятности оценки $\hat\beta_1$ будет отличаться от истинного значения параметра на
\begin{answerlist}
  \item $-0.1(3)$
  \item $0.1(3)$
  \item $0.2(6)$
  \item $-0.2(6)$
  \item Оценка не будет асимптотически смещена
  \item нет верного ответа
\end{answerlist}
\end{question}


\begin{question}
Валидность инструмента \( Z_i \) в модели \( Y_i = \beta_0 + \beta_1 \cdot X_i + \epsilon_i \) обозначает:
\begin{answerlist}
  \item Инструмент \( Z_i \) коррелирует с \( X_i \)
  \item Инструмент \( Z_i \) не коррелирует с ошибкой
  \item Инструмент \( Z_i \) не коррелирует с \( X_i \)
  \item Инструмент \( Z_i \) коррелирует с ошибкой
  \item В модели есть эндогенность
  \item нет верного ответа
\end{answerlist}
\end{question}

\begin{question}
В линейной модели \( Y_i = \beta_0 + \beta_1 \cdot X_i + \epsilon_i \) регрессор \( X_i \) 
является эндогенным. 
Состоятельные оценки коэффициентов можно получить с помощью
\begin{answerlist}
  \item МНК
  \item Обобщенного МНК
  \item Взвешенного МНК   
  \item Метода инструментальных переменных  
  \item нет верного ответа
\end{answerlist}
\end{question}





\begin{question}
Известно, что \( Y_i = \beta_0 + \epsilon_i \), при этом \( \Var(\epsilon_i)=i^2 \). 
Какая из этих оценок  \( \beta_0 \) будет эффективной?

\begin{answerlist}[2]
  \item \( \frac{\sum_{i=1}^{n} \frac{Y_i}{i^2}}{\sum_{i=1}^{n} \frac{1}{i^2}} \)
  \item \( \frac{\sum_{i=1}^{n} \frac{Y_i}{i}}{\sum_{i=1}^{n} \frac{1}{i}} \)
  \item \( \overline{(\frac{1}{i}}) \)
  \item \( \overline{(\frac{1}{i^2})} \)
  \item \( \overline Y \)
  \item нет верного ответа
\end{answerlist}
\end{question}

\begin{question}
Какой из этих тестов на гетероскедастичность не требует выбора переменной, 
  по которой подозревается гетероскедастичность:
\begin{answerlist}
  \item Тест Уайта
  \item Тест Голдфелда-Куандта
  \item Тест Глейзера
  \item Тест Дарбина-Уотсона
  \item Тест Хаусмана
  \item нет верного ответа
\end{answerlist}
\end{question}


\begin{question}
При использовании МНК оценок параметров регрессионного уравнения и робастных ошибок в форме Уайта,
\begin{answerlist}
\item Оценки \( \widehat{\beta} \)  будут состоятельными и неэффективными, доверительные интервалы, 
полученные по \( \widehat{Var_{HCE}} (\widehat{\beta}) \) можно использовать
\item Оценки \( \widehat{\beta} \) будут состоятельными и эффективными, доверительные интервалы, 
полученные по  \( \widehat{Var_{HCE}} (\widehat{\beta}) \)  можно использовать
\item Оценки \( \widehat{\beta} \)  будут несостоятельными и неэффективными, доверительные интервалы, 
полученные по  \( \widehat{Var_{HCE}} (\widehat{\beta}) \)   нельзя использовать
\item Оценки \( \widehat{\beta} \)  будут несостоятельными и неэффективными, доверительные интервалы, 
полученные по  \( \widehat{Var_{HCE}} (\widehat{\beta}) \)   можно использовать
\item Оценки \( \widehat{\beta} \)  будут состоятельными и эффективными, доверительные интервалы, 
полученные по  \( \widehat{Var_{HCE}} (\widehat{\beta}) \)   нельзя использовать
\item нет верного ответа
\end{answerlist}
\end{question}

\begin{question}
Для модели \( Y=X\beta + \epsilon \) с \( \Var(\epsilon) = \Omega \neq \sigma_{\epsilon}^{2}I \)
где $I$ – единичная матрица,
эффективные оценки параметров $\beta$ можно получить с помощью критерия:
\begin{answerlist}
\item \( \min (Y – X \beta)' \Omega (Y – X \beta ) \) 
\item \( \min (Y – X \beta)' \Omega^{-1} (Y – X \beta ) \) 
\item \( \min (Y – X \beta)'  (Y – X \beta ) \) 
\item \( \min (Y – X \beta)^2 \) 
\item Минимизация невозможна
\item нет верного ответа
\end{answerlist}
\end{question}

\begin{question}
Если функция плотности удовлетворяет условиям регулярности, 
  то оценки метода максимального правдоподобия являются
\begin{answerlist}
\item несмещенными
\item несостоятельными
\item неотрицательными
\item инвариантными
\item равномерно распределенными
\item нет верного ответа
\end{answerlist}
\end{question}

\begin{question}
Тест Саргана для проверки валидности инструментов можно использовать, 
  если число эндогенных переменных среди объясняющих
\begin{answerlist}
\item Больше числа экзогенных переменных
\item Больше числа инструментов
\item Меньше числа инструментов
\item Не превышает 10
\item Меньше 3
\item нет верного ответа
\end{answerlist}
\end{question}


\begin{question}
  Если обобщенный метод моментов будет применен в случае, когда число используемых моментов 
  совпадает с числом оцениваемых параметров, то минимизируемая функция в точке оптимума 
\begin{answerlist}
\item Равна нулю
\item Больше нуля
\item Меньше нуля
\item Может быть как больше нуля, так и меньше нуля
\item Равна числу моментных тождеств
\item нет верного ответа
\end{answerlist}
\end{question}



\begin{question}
При проверке гипотезы \( H_0\): \(g(\beta)=0 \) 
для параметров модели \( Y_i=\beta_0+\beta_1 X_{i} + \beta_2 Z_i + \beta_3 W_i + \epsilon_i\), 
\(\epsilon \sim \cN(0,\sigma^2) \) 
с помощью теста множителей Лагранжа, необходимо знать оценки параметров
  \begin{answerlist}
  \item Регрессии на константу   
  \item Регрессии на все факторы, кроме константы   
  \item Только модели без ограничений
  \item Только модели с ограничениями
  \item Как модели с ограничениями, так и модели без ограничений 
  \item нет верного ответа
\end{answerlist}
  \end{question}

  
\begin{question}
Для проверки значимости коэффициента регрессии \( Y_i=\beta_0+\beta_1 X_{i} + \beta_2 Z_i + \beta_3 W_i + \epsilon_i\), 
\(\epsilon \sim \cN(0,\sigma_\epsilon^2) \), 
оцененной с помощью ММП по \( n \) наблюдениям, 
исследователь использует LR статистику. Она имеет распределение
\begin{answerlist}
\item точно \( \cN(0, 1) \)
\item асимптотически \( \cN(0, 1) \)
\item $t_{n-4}$
\item асимптотически $\chi^2_{1}$
\item асимптотически $\chi^2_{n-4}$
\item нет верного ответа
\end{answerlist}
\end{question}
    

\subsection*{Задачи}

\begin{enumerate}
\item 
В линейной модели \( Y_i = \beta_0 + \beta_1 \cdot X_i + \epsilon_i \) регрессор X коррелирован с ошибкой: \( corr(X_i, \epsilon_i  ) \neq 0 \).

\begin{enumerate}
  \item Объясните, каким образом для данной модели можно получить состоятельные оценки с помощью двухшагового МНК. 
  Какие модели необходимо оценить на каждом шаге?
  \item Покажите, что в случае с одним регрессором оценка коэффициента $\beta_1$, 
  полученная с помощью двухшагового МНК и одним инструментом, эквивалентна IV-оценке.
\end{enumerate}

\item Илон Маск оценивает два параметра по выборке из 1000 наблюдений методом максимального правдоподобия. Известна логарифмическая функция правдоподобия:

\( \ell ( \gamma, \beta) = -6 \gamma^2-4 \beta^2+ \gamma \beta– \beta+3 \)

\begin{enumerate}
\item  Найдите оценки параметров \( \gamma \) и \( \beta \) методом максимального правдоподобия

\item Найдите оценку информационной матрицы Фишера.

\item  Постройте 95\%-й интервал для параметра \( \beta \) 

\item  С помощью LM-теста проверьте гипотезу \( \gamma \) = 1 на уровне значимости 5\%.

\item С помощью теста Вальда проверьте гипотезу \( \gamma \) = \( \beta \)  на уровне значимости 5\%.
\end{enumerate}

Критические значения хи-квадрат распределения для 5\%-го уровня значимости равны 3.8, 6.0, 7.8, 9.5, 11.1 (для 1-5 степеней свобод). 


\item По данным открытого скрининга здоровья людей (2730 наблюдений) на наличие диабета второго типа 
($Y_i= 0$ если диабета нет и $Y_i=1$, если диабет обнаружен) 
была построена логит-модель вероятности наличия диабета в зависимости от параметров:

пол (gen, 0 — М, 1 — Ж), возраст (полных лет, от 18-ти), частота сердечного ритма, пульс (ударов в минуту, pulse).

Результаты оценки модели представлены ниже:

  \begin{tabular}{@{}cll@{}}
  \toprule
  \textbf{}      & $\hat\beta$ & $se(\hat \beta)$ \\ \midrule
  \textbf{age}   & 0.07                           & 0.005                              \\ 
  \textbf{pulse} & 0.1                            & 0.265                              \\ 
  \textbf{gen}   & -0.3                           & 0.103                              \\
  \textbf{const} & -8.7                           & 0.611                              \\ \bottomrule
  \end{tabular}
	
Ниже дана таблица классификации при пороге отсечения 0,25:

  \begin{tabular}{@{}lll@{}}
  \toprule
             & $\hat Y_i = 0$ & $\hat Y_i = 1$ \\ \midrule
  $Y_i = 0$ & 1560       & 600        \\
  $Y_i = 1$ & 220        & 350        \\ \bottomrule
  \end{tabular}
 

\begin{enumerate}
\item Какие проблемы возникнут при рассмотрении линейной регрессии вместо логистической в этой модели?
\item Рассчитайте значения чувствительности и специфичности данной модели

\item Посчитайте вероятность наличия диабета для мужчины возрастом 60 лет, пульсом 80.

\item Оцените предельный эффект увеличения возраста для женщины 43 лет и со значением пульса 80. 
Кратко, одной-двумя фразами, прокомментируйте смысл полученных цифр.
\end{enumerate}

\item 
Известно, что количество решённых задачек по экзамене по 
Очень Сложному Предмету зависит от количества выпитого накануне кофе и любви к котикам.
Обе переменные непрерывные, даже любовь к котикам! 
И вообще, любовь к котикам может быть даже бесконечной, ведь нет предела совершенству! 

Истинная зависимость имеет вид:

\( Np_i = \beta_0 + \beta_1 Coffee_i + \beta_2 Cats_i + \varepsilon_i \)

Исследователь Вениамин разделил выборку на три части по степени любви к кофе 
и оценил отдельные регрессии для каждой из трёх подвыборок. 
Известно, что в выборку кофефилов вошло 33 человека, в выборку кофефобов 23, 
а неопределившихся оказалось 44. После этого Вениамин провёл 
тест Голдфелда-Куандта и хотел было вписать его результаты в текст исследования, 
но тут, к его ужасу, прибежала кошка и съела часть цифр! Помогите Вениамину восстановить их!

\begin{enumerate}
\item Зная, что значение тестовой статистики равно 6 и \( RSS \) в группе кофефобов выше, скажите, какие степени свободы имела тестовая статистика 
\item Найдите \( RSS_{coffeephils} \), зная, что \( RSS_{coffeephobs} = 30 \)
\item Предложите способ получить эффективные оценки \( \hat{\beta} \), если известно, что \( \sigma^2_{coffeephils}=1, \sigma^2_{neutrals}=2, \sigma^2_{coffeephobs}=10 \). 
\end{enumerate}



\end{enumerate}


\subsection{Контрольная 3, переписывание}


\begin{enumerate}
  \item 
  Винни-Пух качает пресс на карантине, а также есть блины с мёдом и бутерброды со сгущёнкой. 
  Количество подъемов туловища в разные дни равно $X_i$, $\E(X_i) = \mu$ и $\Var(X_i) = \sigma^2$.
  Количество блинов с мёдом, $M_i$, равно  $M_i = X_i + u_i$, где $\E(u_i) = 0$ и $\Var(u_i)= 2\sigma^2$. 
  Количество бутербродов со сгущенкой, $S_i$, равно  $S_i = X_i + w_i$, где $\E(w_i) = 0$ и $\Var(w_i)= 3\sigma^2$. 
  Винни-Пух себя ведёт независимо в разные дни, кроме того, $X_i$, $w_i$ и $u_i$ независимы.


  За 100 дней карантина Винни-Пух съел 600 бутербродов и 800 блинов. 

  Сова оценивает ожидаемое количество подъемов туловища в день $\mu$ обобщённым методом моментов.
  Первое моментное условие основано на величине $S_i$, второе моментное условие — на величине $M_i$.

  \begin{enumerate}
    \item Какая оценка обобщённого метода моментов получится для матрицы весов
  $W= \begin{pmatrix}
  1 & 0 \\
  0 & 3 \\
  \end{pmatrix}$?

  \item Найдите оптимальную матрицу весов.
  \end{enumerate}

  \item По данным открытого скрининга здоровья людей (400 наблюдений) на наличие диабета второго типа 
  ($Y_i= 0$ если диабета нет и $Y_i=1$, если диабет обнаружен) 
  была оценена модель вероятности наличия диабета в зависимости от параметров:

  пол (gen, 0 — М, 1 — Ж), возраст (полных лет, от 18-ти), частота сердечного ритма, пульс (ударов в минуту, pulse).

  Результаты оценки модели представлены ниже:

    \begin{tabular}{@{}cll@{}}
    \toprule
    \textbf{}      & $\hat\beta$ & $se(\hat \beta)$ \\ \midrule
    \textbf{age}   & 0.07                           & 0.005                              \\ 
    \textbf{pulse} & 0.1                            & 0.265                              \\ 
    \textbf{gen}   & -0.3                           & 0.103                              \\
    \textbf{const} & -8.7                           & 0.611                              \\ \bottomrule
    \end{tabular}
   
\begin{enumerate}
  \item Какую модель разумно было бы оценить для данной задачи? Поясните, как рассчитываются оценки подобной модели. 
  \item Какие коэффициенты являются значимыми?
  \item Оцените вероятность диабета и предельный эффект увеличения возраста на эту вероятность для 
  мужчины 43 лет с пульсом равным 75. 
\end{enumerate}

\item Для каждого студента количество конфет Белочка, необходимое для времени выполнения домашней работы по эконометрике, 
$y_i$, зависит от центрированного ожидаемого количества часов работы, необходимых для ее выполнения $x_i^e$:
\[
y_i = \beta_1 + \beta_2 x_i^e + u_i.  
\]

Величины $x_i^e$ и $u_i$ независимы и одинаково распределены с нулевым ожиданием. 
Наблюдения представляют собой случайную выборку. 
Фактическое центрированное количество затраченных часов описывается уравнением $x_i = x_i^e + u_i$.

Компания Бабаевский с помощью МНК оценивает регрессию
\[
\hat y_i = \hb_1 + \hb_2 x_i.  
\]

\begin{enumerate}
  \item Найдите предел по вероятности для $\hb_2$.
\item Найдите предел по вероятности для $\hb_1$.
\item Являются ли оценки состоятельными?
\item Если оценки не являются состоятельными, то опишите способ получения состоятельных оценок. 
\end{enumerate}

\item 
Известно, что количество решённых задачек по экзамене по 
Очень Сложному Предмету зависит от количества выпитого накануне кофе и любви к котикам.
Обе переменные непрерывные, даже любовь к котикам! 
И вообще, любовь к котикам может быть даже бесконечной, ведь нет предела совершенству! 

Истинная зависимость имеет вид:

\[ 
  Np_i = \beta_0 + \beta_1 Coffee_i + \beta_2 Cats_i + \varepsilon_i 
\]
  
По 400 наблюдениям Вениамин оценил данную зависимость, а так же зачем-то оценил вспомогательную регрессию:
\[
  \widehat{ \abs{\hat \varepsilon_i}} = \hat \gamma_1 + \hat \gamma_2 Coffee_i^2.
\]

Во вспомогательно регрессии $F$-статистика для проверки гипотезы о незначимости регрессии в целом оказалась равной 25. 

\begin{enumerate}
  \item Какое предположение о виде дисперсии случайных ошибок сделал Вениамин при оценке вспомогательной регрессии?
  \item Обнаружил ли Вениамин гетероскедастичность?
  \item Предположим, что Вениамин верно предположил вид гетероскедастичности. 
  Как нужно провести взвешенный МНК для борьбы с гетероскедастичностью?
\end{enumerate}

\end{enumerate}



\subsection{Экзамен}

\subsubsection*{Тест}

\begin{enumerate}
  \item В логит-модели от объясняющих переменных линейно зависит
  \begin{enumerate}
    \item зависимая переменная;
    \item вектор остатков;
    \item латентная переменная. 
  \end{enumerate}
  \item Рассмотрим динамическую модель $y_t = \alpha + \beta x_t + \gamma y_{t-3} + \varepsilon_t$, 
  где $\varepsilon_t$ — MA(1) процесс, $\varepsilon_t = u_t + \delta u_{t-1}$. 
  Известно, что белый шум $(u_t)$ не зависит от стационарного процесса $(x_t)$, 
  а также от прошлых значения зависимой переменной: $u_t$ не зависит от $y_{t-1}$, $y_{t-2}$ и так далее. 

  Каким требования должен удовлетворять инструмент $z_t$ для оценки коэффициентов динамической модели?
  \begin{enumerate}
    \item $\Corr(z_t, y_{t-3}) = 0$;
    \item $\Corr(z_t, u_t) \neq 0$;
    \item $\Corr(z_t, x_t) = 0$;
    \item Инструмент для данной модели не нужен. 
  \end{enumerate}
  \item ...
\end{enumerate}

\subsubsection*{Задачулечки}

\begin{enumerate}
  \item Исследователь Вася оценивает влияние часов подготовки и часов игры в компьютерные игры на успешность прохождения сессии своими товарищами. 
  Для этого он собрал панель из 70 своих
  друзей за четыре последние сессии и оценил регрессию 
  \[
  \widehat{\text{Grade}}_{it} =4 - 0.1 \text{HoursPrep}_{it} + 0.2 \text{HoursGames}_{it}.
  \]
  Все переменные значимы на уровне значимости $\alpha=0.01$, $RSS=700$.

  \begin{enumerate}
    \item После оценки этой регрессии Вася почитал учебник эконометрики и обнаружил там модель с фиксированными индивидуальными эффектами, 
    которую можно использовать для панельных данных. 
    Для этой модели $RSS_{FE}=520$. 
    Помогите Васе выбрать модель при помощи подходящего теста.
    \item После оценки RE-модели, Вася зачем-то провёл тест Хаусмана и получил расчётное значение статистики, равное 10. 
    Напомните Васе, зачем нужен тест Хаусмана и помогите проинтерпретировать полученные расчёты.
    Скажите, как будет распределена и сколько степеней свободы будет иметь тестовая статистика, сформулируйте $H_0$ и $H_a$, сделайте вывод.
    \item Влияние каких переменных не получится оценить в FE модели? 
    Приведите пример такой переменной, подходящей к рассматриваемой задаче. 
    Что делать Васе, если ему очень хочется и оценить модель с индивидуальными эффектами, и оценить влияние таких переменных?
  \end{enumerate}
  
  \item Рассмотрим процесс $Y_{t}=0.8 Y_{t-1}-0.15 Y_{t-2}+\varepsilon_t$, где $\varepsilon_t$ белый шум.
  \begin{enumerate}
    \item Есть ли у этого уравнения стационарное решение вида $MA(\infty)$ относительно $\varepsilon_t$?
    \item Если такое решение есть, то явно выпишите его.
    \item Найдите $\Var(Y_t)$, $\Cov(Y_t, Y_{t-1})$, $\Cov(Y_t, Y_{t-2})$ для стационарного решения.
  \end{enumerate}

  \item Рассмотрим систему одновременных уравнений
  \[
  \begin{cases}
  A_{i}=\alpha_{1}+\alpha_{2} P_{i}+\alpha_{3} W_{i}+u_{i}^{A} \\
  B_{i}=\beta_{1}+\beta_{2} P_{i}+u_{i}^{B} \\
  A_{i}=B_{i}
  \end{cases}
  \]
  Переменные $A_{i}$, $B_{i}$, $P_{i}$ являются эндогенными, $W_{i}$ — экзогенной. 
  Известны результаты оценивания регрессий:
  \[
  \begin{aligned}
  \hat{A}_{i} &=2+3 W_{i} \\
  \hat{P}_{i} &=3-2 W_{i}.
  \end{aligned}
  \]

  \begin{enumerate}
    \item Проверьте идентифицируемость каждого уравнения с помощью условия порядка.
    \item Найдите оценки коэффициентов идентифицируемого уравнения. 
  \end{enumerate}

  \item Для изучения зависимости потребления от дохода была предложена модель частичной корректировки:
  \[
  \begin{cases}
  C_{t}^{*}       = \beta_0 + \beta_1 W_t + \varepsilon_t \\
  C_{t} - C_{t-1} = \lambda (C_t^{*}-C_{t-1}) \\
  \varepsilon_t=\rho \varepsilon_{t-1} + u_t, 
  \end{cases}
  \]
  где $u_t$ — белый шум, $\varepsilon_t$ — стационарный процесс, $C_t$ — совокупное потребление, $W_t$ — заработная плата. 
  Доход и потребление измеряются в миллиардах долларов в базовых ценах 2000 г.

  \begin{enumerate}
    \item Покажите, что модель может быть сведена к виду $C_{t}=\alpha_{0}+\alpha_{1} W_{t}+\alpha_{2} C_{t-1}+\eta_{t}$.
    \item Как следует проводить оценивание полученного приведённого уравнения, чтобы получить состоятельные оценки параметров?
    \item Используя данные за 2000-2019 годы, исследователь получил следующие оценки:
    \[
    \hat{C}_{t}=\underset{(1.2)}{0.9}+\underset{(0.2)}{0.6} W_{t}+\underset{(0.04)}{0.2} C_{t-1}.
    \]
    Найти предельную склонность к потреблению по заработной плате в краткосрочном и долгосрочном периоде. 
  \end{enumerate}

  В скобках указаны стандартные ошибки оценок коэффициентов.

\end{enumerate}


\subsection{Экзамен, решения}

\begin{enumerate}
  \item 
\end{enumerate}