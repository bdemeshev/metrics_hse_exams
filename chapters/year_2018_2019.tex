\subsection{ИП, вспомнить всё!}

\begin{enumerate}

  \item Сфорулируйте теорему о трёх перпендикулярах и обратную к ней. Нарисуйте картинку.

  \item Для матрицы
$
  A=\begin{pmatrix}
  4 & 5  \\
  5 & 4  \\
  \end{pmatrix}
$

  \begin{enumerate}
  \item Найдите собственные числа и собственные векторы матрицы;
  \item Найдите определитель $\det A$ и след $\tr A$;
 \item Известно, что $B = A^{-1} + 2018I$, где $I$ — единичная матрица.
 Найдите собственные числа $B$, определитель $\det B$ и след $\tr B$.

  \end{enumerate}


  \item Блондинка Маша встретила 100 динозавров.
  Средний рост динозавров оказался равен 20 метров, а выборочное стандартное отклонение — 5 метров.

  \begin{enumerate}
    \item Постройте 95\% доверительный интервал для математического ожидания роста динозавра.
    \item На уровне значимости 1\% проверьте гипотезу о том, что математическое ожидание
    роста равно 22 метрам. Против альтернативной гипотезе о неравенстве.
    \item Укажите $P$-значение для теста в предыдущем пункте.
  \end{enumerate}

 \item На брег выходят один за одним 33 богатыря. Двадцать вторым по счёту выходит
 богатырь Мефодий. Какова вероятность того, что Мефодий окажется вторым по силе из всех богатырей,
   если известно, что он самый сильный из всех вышедших до него?

\end{enumerate}


\subsection{Вспомнить всё, ответы}

\begin{enumerate}
\item[2.]
\begin{enumerate}
  \item $\lambda^A_1 = -1$, $\lambda^A_2 = 9$,
  $h_1 = \begin{pmatrix}
  1 & -1
  \end{pmatrix}^T$,
  $h_2 = \begin{pmatrix}
  1 & 1
  \end{pmatrix}^T$
  \item $\det(A) = \lambda^A_1 \cdot \lambda^A_2 = -9$, $\tr(A) = \lambda^A_1 +
  \lambda^A_2 = 8$
  \item $\lambda^B_1 = 1 / \lambda^A_1 + 2018 = 2017$,
  $\lambda^B_2 = 1 / \lambda^A_2 + 2018 = 2018 + 1/9$,
  $\det(B) = \lambda^B_1 \cdot \lambda^B_2 = 2017 \cdot (2018 + 1/9)$,
  $\tr(B) = \lambda^B_1 + \lambda^B_2 = 4035 + 1/9$
\end{enumerate}
\item[3.]
\begin{enumerate}
\item $\left[20 - 1.96 \cdot 5 / \sqrt{100}; 20 + 1.96 \cdot 5 / \sqrt{100} \right]$
\item $z_{obs} = -4$, $z_{crit} = \pm 2.6$, основная гипотеза отвергается
\item $p-value \approx 0 $
\end{enumerate}
\item[4.] Заметим, что неважно, каким идёт Мефодий, а главное, что он самый сильный из
$22$ вышедших. Из $22$ вышедших всё равно есть кто-то самый сильный, и если его
считать Мефодием, то ничего не изменится. Значит, нам нужна вероятность того,
что второй лучший из всех попадёт на $22$ места из $33$, а самый лучший на $11$
мест из $32$ оставшихся. Искомая вероятность — $11/48$.

Или по формуле условной вероятности. Пусть $A$ означает, что Мефодий второй по силе
из всех, а $B$ — что он первый по силе из вышедших. Тогда
\[
\P(B) = 1/22,
\]
так как Мефодий должен быть самым сильным из вышедших, и
\[
\P(A \cap B) = 11/33 \cdot 1/32,
\]
так как на $22$-ом месте должен быть самый сильный из вышедших и второй по силе из
всех. Если он второй по силе из вышедших, то первый оказался среди $11$ невышедших,
и эта вероятность равна $11/33$, а вероятность быть вторым по силе среди всех равна
$1/32$. Итого,
\[
\P(A|B) = \frac{\P(A \cap B)}{\P(B)} = \frac{11}{48}.
\]
\end{enumerate}



\subsection{Задачи миниконтрольных ИП}

\begin{enumerate}
  \item Найдите SVD-разложение матрицы $
  \begin{pmatrix}
  2 & 0 & -1 \\
  2 & 1 & 0 \\
  \end{pmatrix}$
 \item Найдите дифференциал $d \cos(r^TAr+br)$, где $A^T=A$ и $b$ — это константы.
 \item Постройте регрессию вектора $y = (4,2,-2)^T$ на вектора $x=(1,0,-1)^T$ и $z=(1,1,-1)^T$
 без константы.
 \item Известно, что $y=x + 2z$. Винни-Пух построил регрессию $\hat y_i = \hat\beta_1 + 0.16 x_i$.
 Пятачок построил регрессию $\hat x_i = \hat \alpha_1 + 1\cdot y_i$.

 Помогите Сове найти коэффициент $\hat \gamma_2$ в регрессии $\hat y_i = \hat\gamma_1 + \hat\gamma_2 z_i$.
\end{enumerate}


\subsection{Контрольная работа-1. Базовая часть}

%
\begin{question}
Стьюдентизированные остатки регрессии используются
\begin{answerlist}
  \item в тесте Саргана
  \item на первом шаге двухшагового МНК
  \item на первом шаге при проведении теста Годфельда-Квандта
  \item в методе главных компонент
  \item для выявления выбросов
\end{answerlist}
\end{question}



%
\begin{question}
Тест Саргана для проверки валидности инструментов можно использовать
только в том случае, если число инструментов
\begin{answerlist}
  \item меньше числа эндогенных переменных
  \item больше числа эндогенных переменных
  \item совпадает с числом эндогенных переменных
  \item совпадает с числом экзогенных переменных
  \item меньше числа экзогенных переменных
\end{answerlist}
\end{question}



%
\begin{question}
(1 балл) Какое условие НЕ требуется в теореме Гаусса-Маркова?
\begin{answerlist}[2]
  \item матрица регрессоров \(X\) имеет полный ранг
  \item модель \(Y=X\beta + \varepsilon\) правильно специфицирована
  \item случайные ошибки \(\varepsilon_i\) не коррелированы
  \item случайные ошибки \(\varepsilon_i\) имеют одинаковые дисперсии
  \item случайные ошибки \(\varepsilon_i\) нормально распределены
  \item нет верного ответа
\end{answerlist}
\end{question}

\begin{solution}
\begin{answerlist}
  \item Bad answer :(
  \item Bad answer :(
  \item Bad answer :(
  \item Bad answer :(
  \item Good answer :)
\end{answerlist}
\end{solution}

%
\begin{question}
(1 балл) Выборочная корреляция между регрессорами \(X\) и \(Z\) равна \(0.5\). В
регрессии \(\hat Y_i = \hat\beta_0 + \hat\beta_1 X_i + \hat\beta_2 Z_i\)
показатель \(VIF\) для регрессора \(X\) равен
\begin{answerlist}
  \item \(1/4\)
  \item \(4/3\)
  \item \(1/2\)
  \item \(2\)
  \item \(3/4\)
  \item нет верного ответа
\end{answerlist}
\end{question}

\begin{solution}
\begin{answerlist}
  \item Bad answer :(
  \item Good answer :)
  \item Bad answer :(
  \item Bad answer :(
  \item Bad answer :(
\end{answerlist}
\end{solution}

%
\begin{question}
Стьюдентизированные остатки регрессии используются
\begin{answerlist}
  \item в методе главных компонент
  \item на первом шаге двухшагового МНК
  \item в тесте Саргана
  \item для выявления выбросов
  \item на первом шаге при проведении теста Годфельда-Квандта
\end{answerlist}
\end{question}



%
\begin{question}
По 52 наблюдениям студент построил две регрессии,
\(\hat Y_i = 3.1 + 0.8X_i\) и \(\hat X_i = -0.3 + 0.2Y_i\). Коэффициент
\(R^2_{adj}\) для первой регрессии примерно равен
\begin{answerlist}
  \item \(0.14\)
  \item \(0.16\)
  \item \(0.40\)
  \item \(0.32\)
  \item \(0.37\)
\end{answerlist}
\end{question}



%
\begin{question}
Использование скорректированных стандартных ошибок Уайта при
гомоскедастичности приведет к
\begin{answerlist}
  \item смещённости МНК оценок коэффициентов
  \item повышению эффективности МНК оценок коэффициентов
  \item получению состоятельной оценки дисперсии случайной ошибки
  \item понижению эффективности МНК оценок коэффициентов
  \item несостоятельности МНК оценок коэффициентов
\end{answerlist}
\end{question}

\begin{solution}
========
\end{solution}


%
\begin{question}
Рассмотрим модель множественной регрессии \(Y=X\beta+\varepsilon\), где
\(\hat Y = X\hat\beta\), \(e=Y-\hat Y\). Величина \(RSS\) --- это
квадрат длины вектора
\begin{answerlist}
  \item \(\hat Y - \bar Y\)
  \item \(\varepsilon\)
  \item \(\hat Y\)
  \item \(Y-\bar Y\)
  \item \(e\)
\end{answerlist}
\end{question}



%
\begin{question}
Рассмотрим модель
\(Y_i= \beta_0 + \beta_z Z_{i} + \beta_w W_{i} + \varepsilon\) при
гетероскедастичности. Стандартная ошибка МНК-оценки, рассчитываемая по
формуле \(se(\hat\beta_w)=\sqrt{RSS \cdot (X'X)^{-1}_{33}/(n-3)}\),
является
\begin{answerlist}
  \item смещённой
  \item несмещённой
  \item состоятельной
  \item смещённой вниз
  \item смещённой вверх
\end{answerlist}
\end{question}



%
\begin{question}
Чебурашка оценил модель \(Y_i = \beta_0 + \beta_1 X_i + \varepsilon_i\),
а Крокодил Гена --- модель \(X_i = \gamma_0 + \gamma_1 Y_i + u_i\).
Оказалось, что \(\hat\gamma_1 = 0.25/\hat\beta_1\). Величина \(R^2\) в
регрессии Чебурашки равна
\begin{answerlist}
  \item \(1\)
  \item \(0.5\)
  \item \(0\)
  \item \(0.75\)
  \item \(0.25\)
\end{answerlist}
\end{question}

\begin{solution}
\(R^2 = \hat\beta_1 \cdot \hat\gamma_1\)
\end{solution}




\begin{enumerate}
  \item (5 баллов) Случайные величины $X$ и $Y$ независимы и имеют хи-квадрат распределение
  с 5 и с 10 степенями свободы, соответственно. Случайная величина $Z$ равна $Z = (X+Y)/X$.

  Найдите значение $z^*$ такое, что $\P(Z > z^*)=0.05$.
  \item (5 баллов) Докажите, что для модели парной регрессии $Y_i = \beta_0 + \beta_1 X_i + \varepsilon_i$,
оцененной с помощью МНК, выполнено равенство $\sum_{i=1}^n Y_i = \sum_{i=1}^n \hat Y_i$.

  \item (5 баллов) Аккуратно сформулируйте теорему Гаусса-Маркова для случая парной регрессии.

  \item (10 баллов) На основании 62 наблюдений Чебурашка оценил функцию спроса на апельсины:

 \[
 \hat Y_i = \underset{(1.6)}{3} - \underset{(0.2)}{1.25} X_i, \text{ где } \sum_i (X_i - \bar X)^2 =2.25
 \]

 В скобках приведены стандартные ошибки коэффициентов, случайные ошибки в регрессии можно считать нормальными.


  \begin{enumerate}
    \item Проверьте гипотезы о значимости каждого из коэффициентов регрессии при уровне значимости 5\%.
    \item Проверьте гипотезу о равенстве коэффициента наклона -1 при уровне значимости 5\%
    и односторонней альтернативной гипотезе, что коэффициент наклона меньше -1.
    \item Найдите оценку дисперсии ошибок.
    \item Найдите 95\% интервальный индивидуальный прогноз в точке $X=8$.
  \end{enumerate}
\end{enumerate}



\subsection{Контрольная работа-1. ИП часть}

\begin{enumerate}
  \item Храбрый исследователь Вениамин поделил выборку на обучающую $(X, y)$ и тестовую $(X_{test}, y_{test})$.
  Регрессоры $X$ и $X_{test}$ Вениамин считает нестохастическими, а предпосылки
  теоремы Гаусса-Маркова — выполненными на всей исходной выборке. Естественно,
  $\hat y_{test} = X_{test}\hat\beta$, где $\hat\beta$ оценивается по обучающей выборке.

  Помогите Вениамину найти $\Var(\hat y_{test})$ и $\Cov(\hat \beta, \hat y_{test})$.

  \item Рассмотрим матрицу $X$ полного ранга с $n$ наблюдениями и $k$ столбцами.
  В каких границах могут лежать диагональные элементы матрицы-шляпницы $H$?
  Чему равно их среднее значение?

  Подсказка: найдите $\Var(\hat y)$ и $\Var(\hat u)$ в рамках предпосылок теоремы Гаусса-Маркова.

  \item Рассмотрим стандартный $t$-тест на равенство некоторого коэффициента бета нулю.
  Докажите, что
  \[
         t^2 = \frac{RSS_r - RSS_{ur}}{RSS_{ur}/(n-k)},
  \]
  где $RSS_r$ — сумма квадратов остатков в модели без тестируемого коэффициента
  (выкинут регрессор при проверямом коэффициенте),
  $RSS_{ur}$ — аналогичная сумма в модели с включённым тестируемым коэффициентом, $k$ —
  число оцениваемых коэффициентов бета в модели с тестируемым коэффициентом, $n$ —
  количество наблюдений.

  Утешительный приз: упростите эту формулу для случая парной регрессии и докажите её :)

  \item Рассмотрим стандартную ошибку оценки коэффициента бета при регрессоре $z$
  в множественной регрессии.
  Докажите, что

  \[
         se^2(\hat\beta_z) = \frac{RSS / (n-k)}{\sum (z_i - \bar z)^2} \cdot \frac{1}{1 - R^2_z},
  \]
  где $R^2_z$ — коэффициент детерминации во вспомогательной регресии объясняющей переменной
  $z$ на остальные объясняющие переменные.

  Утешительный приз: упростите эту формулу для случая парной регрессии и докажите её :)


   \item У Винни-Пуха есть случайный вектор $w$ и одномерная случайную величину $z$.
   Винни-Пуху известны величины $\Cov(w, w) = A$ и $\Cov(w, z) = b$.

   К сожалению, у Винни-Пуха опилки в голове, а он очень хочет найти такую линейную комбинацию
   компонент вектора $w$, которая была бы сильнее всего коррелирована со случайной
   величиной $z$.

   Помогите Винни-Пуху!

   Как выглядят веса этой линейной комбинации?
   Чему равна максимально возможная корреляция?

 \item Машенька построила парную регрессию по 11 наблюдениям с $R^2=
0.95$. Чтобы напакостить Машеньке, Вовочка переставил в случайном
порядке значения зависимой переменной и предложил Машеньке заново оценить модель.

Какой ожидаемый $R^2$ получит Машенька?


\end{enumerate}
