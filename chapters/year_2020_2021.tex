% !TEX root = ../metrics_hse_exams.tex

\subsection{Контрольная 3, 2021-03-30}

Правила: 120 минут, онлайн, самодельный прокторинг в зуме. Можно использовать лист А4.


\begin{enumerate}
    \item Исследователи Стефано и Сергей уже 60 дней играют в Overwatch. Каждый день они записывают количество побед $Wins_i$, температуру на улице $Temp_i$ и число прочитанных статей по эконометрике $Papers_i$. Оценив регрессию $Wins_i$ на константу, $Temp_i$ и $Papers_i$, ребята решили проверить наличие в модели гетероскедастичности. 
    
    Случайные ошибки независимы и нормально распределены. 
    
    \begin{enumerate}
        \item (2 балла) Для тестирования гетероскедастичности Стефано оценил две вспомогательные регрессии по 20 самым тёплым и 20 самым холодным дням со всеми исходными регрессорами. В результате оказалось $RSS_{Warm} = 320$ и $RSS_{Cold} = 140$ соответственно. Сформулируйте нулевую и альтернативную гипотезы подходящего теста и проведите его.
        % 2 балла
        % BB: прописать H_a
        \item (3 балла) Для тестирования гетероскедастичности Серёжа оценил вспомогательную регрессию на константу, исходные переменные, их квадраты и попарные произведения по всем наблюдениям. И получил в ней $R^2_{aux} = 0.3$. Сформулируйте нулевую и альтернативную гипотезы подходящего теста и проведите его.
        % 3 балла
    \end{enumerate}
    
    
    \item Рассмотрим линейную регрессионную модель 
    $Y  = X\beta + \varepsilon$
    с диагональной ковариационной матрицей ошибок регрессии, где $\Var(\varepsilon_i) = c \cdot i$.
   
   Проверьте, выполнено ли для оценок модели свойство ортогональности $\hat Y$  и  $e = Y - \hat Y$, если
   \begin{enumerate}
       \item (1 балл) коэффициенты $\beta$ оценивают с помощью МНК;
       % 1 балл
       \item (3 балла) коэффициенты $\beta$ оценивают с помощью взвешенного МНК.
       % 3 балла
   \end{enumerate}
   
   Ответ обоснуйте.
    
    \item Величины $X_1$, \ldots, $X_n$ независимы и равномерно распределены на отрезке $[2a, 3a]$, где $a > 0$.
    
    \begin{enumerate}
        \item (2 балла) Найти оценку максимального правдоподобия параметра $a$.
        % 2 балла
        \item (3 балла) Является ли эта оценка несмещенной? Ответ обоснуйте.
    Если оценка является смещенной, то найдите величину смещения.
        % 3 балла
    % \item Посчитайте значение оценки для выборки $X_1 = 5$, $X_2 = 10$, $X_3 = 10$ и $X_4 = 7$.
    \end{enumerate}
% В разных вариантах эти 2a, 3a можно менять
    
    \item Исследователь Михаил 100 дней решал задачки по эконометрике вместе с гостящими у него друзьями. % подругами?
    Михаил предполагает, что число решённых задачек $Problems_i$ зависит от числа выпитых банок газировки $Drinks_i$ и количества съеденных пачек чипсов $Chips_i$. Оценив соответствующую регрессию, он получил результат:
    
    \[
        \widehat{Problems}_i = 2.43 + 10.26 Drinks_i + 0.47 Chips_i
    \]
    
    Помимо этого, зная что ошибки в модели нормальны, он получил логарифм функции правдоподобия для этой модели: $\ell_{UR} = -535.82$. Оценка ковариационной матрицы оценок коэффициентов имеет вид: 
    
    \[
        \begin{pmatrix}
        61.30 & -8.51 & 0 \\
        -8.51 & 2.13  & 0 \\
        0     & 0     & 1.20 \\
        \end{pmatrix}
    \]
    
    Из исследований других умных студентов известно, что очень часто влияние газировки и чипсов на эффективность учёбы оказывается одинаковым по величине, то есть $\beta_{Drinks} = \beta_{Chips}$. Для проверки этой гипотезы Миша даже оценил вспомогательную регрессию и получил:
    
    \[
        \widehat{Problems}_i = 27.51 + 3.99 (Drinks_i + Chips_i)
    \]
    
    Логарифм правдоподобия для этой модели $\ell_R = -548.84$.

    Помогите Мише проверить гипотезу при помощи
    
    \begin{enumerate}
        \item (2 балла) $t$-теста; % 2 балла
        \item (2 балла) теста отношения правдоподобия; % 2 балла
        \item (2 балла) теста Вальда. % 2 балла
    \end{enumerate}
    % тест Вальда соответствует $t$-тесту. Это задумка?
    
    \item Маркетолог-аналитик Евгений пытается оценить зависимость предполагаемого объёма продаж, $y_i$, от размера инвестиций $x_i$,
\[    
y_i = \beta_1 + \beta_2 x_i + \varepsilon_i.
\]
Оказалось, что предполагаемый объем продаж измеряется с ошибкой $w_i$, то есть наблюдаемый объём продаж равен $z_i = y_i + w_i$.   

Поэтому маркетолог оценивает следующее уравнение регрессии:
\[ 
\hat z_i = \hat\beta_1 + \hat\beta_2 x_i
\]
Известно, что $\E(w_i)=\E(\varepsilon_i)=0$ и $\Var(w_i)=\sigma^2_w$, $\Var(\varepsilon_i)=\sigma^2_\varepsilon$. Ошибка $w_i$ не коррелирует с $x_i$ и не коррелирует с $y_i$. Ошибка $\varepsilon_i$ не коррелирует с $x_i$.

\begin{enumerate}
    \item (2 балла) Найдите предел по вероятности для $\hat{\beta}_2$. 
    \item (1 балл) Является ли оценка состоятельной? Поясните.
\end{enumerate}


\item Амбидекстр Леонардо пишет левой и правой рукой с одинаковой ожидаемой скоростью $a$ слов в минуту. 
    
    Величина $L_i$ — количество слов, написанных левой рукой за минуту $i$, величина $R_i$ — количество слов, написанных правой рукой за минуту $i$. Величины, относящиеся к разным минутам, независимы. 
    
    Обратная ковариационная матрица вектора $(L_i, R_i)$ равна
    \[
        \Var^{-1} \begin{pmatrix}
        L_i \\
        R_i \\
        \end{pmatrix} = 
        \begin{pmatrix}
        4a^2 & a \\
        a & a^2+4 \\
        \end{pmatrix}.
    \]
    
    Пронаблюдав за Леонардо 100 минут я обнаружил, что $\bar L = 20$, а $\bar R = 30$.
    
    \begin{enumerate}
        \item (2 балла) Найдите оценку параметра $a$, используя обобщённый метод моментов с единичной взвешивающей матрицей. Обозначим эту оценку $\hat a$. % 2 балла 
        \item (1 балл) Оцените оптимальную взвешивающую матрицу, используя полученную предварительную оценку $\hat a$. % 1 балл
        \item (2 балла) Найдите финальную оценку параметра $a$, используя обобщённый метод моментов с оценённой взвешивающей матрицей. % 2 балла
        

    \end{enumerate}

    \item 
Для вероятности высоких кассовых сборов фильма в США за первые выходные (выше 40 млн. долл.) была оценена логистическая модель. Зависимая переменная  $Y$ равна $1$, если кассовые сборы больше 40 млн. долл. В модель включены регрессоры: 

$\text{Bud}$ — бюджет фильма в млн. долларов, 

$G$ — рейтинг кинокритиков в процентах (от 0 до 100), 

$W$ — жанр фильма: $W=1$ для фантастики или мультфильмов, $W=0$ в остальных случаях.

\[
\hat \P(y_i = 1) = \Lambda(-8.68 + 0.027 \text{Bud}_i + 0.052 G_i + 1.91 W_i)    
\]

\begin{enumerate}
% \item  Выпишите логарифмическую функцию правдоподобия, используемую для оценивания представленной линейной регрессионной модели. Выпишите процедуру получения оценок коэффициентов регрессии.
\item (2 балла) Найдите предельный  эффект влияния бюджета фильма при условии жанра $W=1$, рейтинга кинокритиков $G = 70\%$ и бюджета $50$ млн. долл. Проинтерпретируйте полученное значение. 
% \item (2 балла) Вариант 2. Определите предельный эффект влияния жанра на вероятность получить более высокие кассовые сборы при рейтинге кинокритиков $80\%$ и бюджете в $100$ млн. долл. Проинтерпретируйте полученное значение. 

\item (2 балла) Найдите максимальный предельный эффект влияния бюджета при $W=0$ и рейтинге кинокритиков $G=70\%$. Проинтерпретируйте полученное значение. 

\item (1 балл) Найдите значение чувствительности этой модели в точке отсечения $0.23$ по таблице ниже:

\begin{tabular}{@{}cccc@{}}
\toprule
           & $Y=1$ & $Y=0$ & Всего\\ \midrule
$\hat Y=1$ & 30 & 15 &  \\
$\hat Y=0$ & 5 & 80 &  \\
Всего &  &  & 130   \\ \bottomrule
\end{tabular}

Как изменится значение чувствительности при увеличении границы отсечения с $0.23$ до $0.5$? 
\end{enumerate}


\item Рассмотрим систему уравнений:
\[
\begin{cases}
y_1 = \beta_1 y_3 + \gamma_1 x_2 + u_1 \\
y_2 = \beta_2 y_1 + \beta_3 y_3 + \gamma_2 x_1 + \gamma_3 x_2 + u_2 \\
y_3 = \gamma_4 x_1 + u_3 \\
\end{cases}
\]

Наблюдения представляют собой случайную выборку. Переменные $y_i$ — эндогенные, переменные $x_i$ — экзогенные.

Для каждого из уравнений системы:

\begin{enumerate}
\item (2 балла) Проверьте выполнение условий порядка и условие ранга для каждого уравнения. 
\item  (1 балл) Проверьте идентифицируемость системы в целом. 
\item  (2 балла) Опишите схему получения оценок коэффициентов идентифицируемых уравнений. Укажите все регрессии, которые вы будете использовать. 

% \item  Укажите количество оцененных параметров при применении к этой модели методов 2SLS и 3SLS и распишите процедуру оценивания в каждом из случаев.
\end{enumerate}


Выберите \textbf{одну из двух} задач по выбору:

        
\item[9A.] Дополнительная задача 1. (2 балла) Приведите содержательный экономический пример, в котором следует использовать мультиномиальную логит модель.
 
\item[9B.] Дополнительная задача 2. (2 балла) Чем отличается тобит модель от модели Хекмана?
\end{enumerate}

