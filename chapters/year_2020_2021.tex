% !TEX root = ../metrics_hse_exams.tex

\subsection{Контрольная 1, 2020-10-22}


\begin{enumerate}


\item Рассеянный исследователь Вовочка 555 дней замерял своё потребление шоколада 
и число решённых задач по эконометрике. 
Вовочка оценил по своим данным парную регрессию числа решенных задач на потребление шоколада 
(регрессию с константой), но потерял все результаты вычислений и не справится без вашей помощи!

\begin{enumerate}
    \item Вовочка запомнил, что 95\%-ый доверительный интервал для коэффициента 
    при шоколаде был от 1.72 до 8.28. Помогите ему восстановить оценку $\hat\beta_{choc}$  и оценку её стандартного отклонения. 

    Ошибки в модели для этого и следующего пунктов считайте нормальными.
\item Помогите Вовочке проверить значимость  $\hat\beta_{choc}$ на 10\% уровне значимости.
\item Можно ли было бы считать полученные МНК-оценки коэффициентов несмещёнными и эффективными 
в случае равномерных от -5 до 5 ошибок? Почему?
\item Можно ли было бы считать полученные МНК-оценки коэффициентов несмещёнными и эффективными 
в случае равномерных от 0 до 5 ошибок? Почему?
\end{enumerate}


\item Рассмотрим уравнение линейной регрессии $Y_i = \beta X_i + u_i$. 
Все предпосылки теоремы Гаусса-Маркова выполнены.
\begin{enumerate}
\item Найдите МНК  оценку коэффициента $\hb$.
\item Проверьте,  является  ли эта оценка несмещенной.
\item Выведите формулу для несмещённой оценки дисперсии этой оценки. 
\item По выборке оказалось, что $\hb = 2.25$ и $se(\hb)=0.2$. 
Проинтерпретируйте значение оценки коэффициента.
\item Выведите формулу оценки дисперсии для ошибки прогноза $\hat Y_{N+1}$.
\end{enumerate}

\item Для модели $X_i = \beta_0 + \beta_1 Y_i + u_i$ известна МНК-оценка коэффициента $\hb_1 = -1$. 
Также для данной регрессии известны $N=102$, $\sum (Y_i - \bar Y)^2=10$ и  $TSS=200$. 
\begin{enumerate}
    \item  Найдите коэффициент детерминации $R^2$ для этой модели.
\item Найдите оценку дисперсии оценки коэффициента $\hb_1$.
\item Для регрессии $\hat Y_i = \hat\alpha_0 + \hat\alpha_1 X_i$ найдите оценку $\hat\alpha_1$.
\item Найдите выборочный коэффициент корреляции $\hCorr(X,Y)$.
\end{enumerate}



\end{enumerate}


\subsection{Контрольная 1, 2020-10-22, решения}

\begin{enumerate}
    \item Вместо $t_{553}$-распределения можно использовать нормальное $\cN(0;1)$.
\begin{enumerate}
    \item {[2]} Критическое значение равно $t_{crit} = 1.96$. 

    Отсюда находим $\hb = (8.28 + 1.72)/2=5$ и $se(\hb) = (8.28 - 1.72)/(2\cdot 1.96) = 1.67$.
    \item {[1]} $t_{obs} = 5/1.67 = 3$. Таблицы не нужны, достаточно помнить, 
    что при уровне значимости $\alpha=0.05$ критическое значение равно $1.96$. 
    При более высоком уровне значимости критическое значение падает. Значит $H_0$ отвергается. 

    \item {[1]} Ожидание ошибки равно нулю, дисперсия постоянна, значит условия теоремы Гаусса-Маркова выполнены. 
    Обе оценки являются несмещёнными и эффективными среди линейных несмещённых оценок. 

    \item {[2]} Ожидание ошибки равно $2.5$, дисперсия постоянна, значит условия теоремы Гаусса-Маркова нарушены.
    Однако при переносе $2.5$ из ошибки в константу нарушение исчезает. 
    Оценка наклона: несмещена и эффективна среди линейных несмещённых оценок.

    Оценка константы: смещена на 2.5, поэтому оценка не лежит в классе линейных несмещенных оценок, 
    и говорить об её эффективности в этом классе бессмысленно. 
    При этом дисперсия оценки константы равна дисперсии эффективной оценки.

\end{enumerate}

\item Кратко:
\begin{enumerate}
    \item {[1]} $\hb = \frac{\sum X_i Y_i}{\sum X_i^2}$
    \item {[1]} $\E(\hb) = \beta$
    \item {[2]} $\hVar(\hb) = \frac{\hat\sigma^2_u}{\sum X_i^2}$
    \item {[1]} $t_{obs} = 11.25$, коэффициент значимо отличен от нуля. 
    Зависимая переменная в среднем в $2.25$ раз больше регрессора.
    \item {[2]} $\hVar(\hat Y_{N+1} - Y_{N+1}) = \hat\sigma^2_u \left( 1 + \frac{X_{N+1}^2}{\sum X_i^2}  \right)$
\end{enumerate}

\item Обратите внимание, что $Y_i$ является регрессором. 
\begin{enumerate}
    \item {[2]} $\hb_1 = \frac{\sum (X_i - \bar X)(Y_i - \bar Y)}{\sum (Y_i - \bar Y)^2}$.

    Следовательно, $\sum (X_i - \bar X)(Y_i - \bar Y) = -10$. 

    Решаем одним махом г) и а)!

    \[
    \hCorr(X, Y) = \frac{\sum (X_i - \bar X)(Y_i - \bar Y)}{\sqrt{10\cdot 200}} = -\sqrt{5}/10    
    \]

    Отсюда:
    \[
    R^2 = 5/100 = 1/20   
    \]

    \item {[2]} $RSS = 0.95 \cdot 200 = 190$. Отсюда $\hat\sigma^2_u = 190/100 = 1.9$ и $se^2(\hb_1) = 1.9/10=0.19$.
    \item {[2]} В узких кругах широко известно, что корреляции по модулю равна среднему геометрическому оценок 
    в прямой и обратной моделях. 

    \[
    R^2 = \hat\alpha_1 \cdot \hb_1    
    \]

    Следовательно, $\hat\alpha_1 = -1/20$.

    \item {[1]} уже решили!
\end{enumerate}


\end{enumerate}
    



\subsection{Промежуточный экзамен, 2020-12-23}

\begin{enumerate}

    \item \dots
    \item Из предыдущих исследований и откровений внеземного разума известно, что время, 
    проведённое студентом за игрой в Cyberpunk 2077 $\text{Time}_i$, 
    зависит от числа экзаменов в ближайшую неделю $\text{Exams}_i$ и объёма выпитого кофе $\text{Coffee}_i$. 
    Известно также, что $\Var(\text{Exams}_i)=8$, $\Var(\text{Coffee}_i) = 20$, $\Corr(\text{Coffee}_i, \text{Exams}_i)=0.7$.
    Совместное распределение переменных хорошо аппроксимируется многомерным нормальным.
    
    Исследователь Стёпчик, имея большое число наблюдений, оценил соответствующую регрессию и получил результат:
    \[
    \widehat{\text{Time}}_i=\underset{(5.7)}{20}-\underset{(4.8))}{3} \text{Exams}_i+ \underset{(1.4)}{1.2} \text{Coffee}_i.
    \]

    В скобках указаны $t$-статистики.
    
    Увидев, что $t$-статистика при переменной $\text{Coffee}_i$ мала, исследователь Стёпчик решил выкинуть из модели эту переменную.
    
    Оценив парную регрессию, Стёпчик получил результат:
    \[
    \widehat{\text{Time}}_i = \underset{(3.8)}{24} - \underset{(5.9)}{2.2} \text{Exams}_i.
    \]


    \begin{enumerate}
        \item Обычно невключение существенных переменных приводит к несостоятельности оценок коэффициентов при оставшихся в модели переменных. 
        Будет ли наблюдаться смещение в нашем случае? 
        Если нет, докажите, если да, найдите величину смещения, зная, что истинная зависимость имеет вид:
        \[
        \text{Time}i = 15 - 2.5 \text{Exams}_i + 1.5  \text{Coffee}_i + \varepsilon_i.
        \]
        \item Стёпчик зачем-то ещё оценил регрессию времени только на объём выпитого кофе. 
        Найдите математическое ожидание коэффициента при переменной $\text{Coffee}_i$.
    \end{enumerate}
    
    \item Известны результаты оценки множественной регрессии:
    \[
    \hat{y}_i = 157.3 + \underset{(0.31)}{4.05} x_{i} - \underset{(0.21)}{2.79} z_i + \underset{(0.27)}{1.95} w_i + \underset{(2.5)}{6.95} d_{i}
    \]
    и $\hCov(\hb_x, \hb_z)=-0.04$, $RSS=57907$, $TSS=203400$, $n=988$.
    \begin{enumerate}
        \item Протестируйте гипотезу, что $\beta_x=-2 \beta_z$.
        \item Проверьте значимость регрессии в целом.
    \end{enumerate}

    Далее исследователь оценил другую регрессию
    \[
    \hat{y}_{i} = \underset{(6.47)}{156.27}+\underset{(0.128)}{2.06} v_{i}+\underset{(0.27)}{1.95} w_{i},
    \]
    где $v_{i}=x_{i}+2 z_{i}$ и $RSS = 57921$.

    \begin{enumerate}[resume]
        \item Протестируйте гипотезу об одновременном равенстве $\{\beta_{x}=2 \beta_{z}, \beta_{d} = 0\}$.
    \end{enumerate}

    В скобках указаны $t$-статистики.

    \item Исследователь Януарий хочет оценить регрессию $Y_i$ на константу, $X_i$ и $Z_i$ с регуляризацией LASSO и потому рассматривает штрафную функцию 
    \[
        Q(\hb) = \sum_{i=1}^n (Y_i - \hat Y_i)^2 + \lambda (\abs{\hb_x} + \abs{\hb_z}). 
    \]
    По ошибке Януарий не стандартизирует переменные.
    Прогнозы Януарий строит по формуле $\hat Y_i = \hb_1 + \hb_x X_i + \hb_z Z_i$.

    \begin{enumerate}
        \item Верно ли, что в данной модели сумма модулей остатков будет равна нулю при любом $\lambda$?
        \item Верно ли, что в данной модели остатки будут ортогональны вектору из единиц при любом $\lambda$?
        \item Верно ли, что в данной модели остатки будут ортогональны прогнозам?
        \item При некотором $\lambda$ оказалось, что оценки всех коэффициентов отличны от нуля. 
        Януарий утверждает, что он может занулить оценку $\hat{\beta}_{z}$ просто сменив масштаб переменной $Z$. 
        Прав ли он?
    \end{enumerate}

    Аргументируйте ответы.
\end{enumerate}

\subsection{Промежуточный экзамен, 2020-12-23, решения}

\begin{enumerate}
    \item 
    \item 
    \item 
    \item Раз есть константа в модели, то остатки в сумме дают ноль. Это и есть ортогональность вектору из единиц. 
    Обоснование: берем производную и приравниваем к нулю, видим, что условие первого порядка есть равенство суммы остатков нулю. 
    Достаточно устной аргументации про наличие константы.

    
    Сумма модулей остатков конечно не ноль. Сумма модулей остатков может быть ноль только если данные идеально ложаться на линию регрессии, 
    да и то при несильной регуляризации. 
    
    При некотором промежуточном лямбда, одна из оценок бет будет нулевой, другая нет. 
    Стало быть ортогональности предиктору при ненулевой оценке не будет. 
    Стало быть не будет и ортогональности остатков прогнозам. 

    Выбирая масштаб переменной я влияю на относительную важность двух слагаемых в целевой функции LASSO. 
    Поэтому выбрав масштаб переменной очень мелким можно добиться нулевой оценки коэффициента, 
    из-за того, что весь вес целевой функции будет перенесен на слагаемое с модулем.
\end{enumerate}


\subsection{Контрольная 3, 2021-03-30}

Правила: 120 минут, онлайн, самодельный прокторинг в зуме. Можно использовать лист А4.


\begin{enumerate}
    \item Исследователи Стефано и Сергей уже 60 дней играют в Overwatch. 
    Каждый день они записывают количество побед $Wins_i$, температуру на улице $Temp_i$ и число прочитанных статей по эконометрике $Papers_i$. 
    Оценив регрессию $Wins_i$ на константу, $Temp_i$ и $Papers_i$, ребята решили проверить наличие в модели гетероскедастичности. 
    
    Случайные ошибки независимы и нормально распределены. 
    
    \begin{enumerate}
        \item (2 балла) Для тестирования гетероскедастичности Стефано оценил две вспомогательные регрессии по 20 самым тёплым и 20 самым холодным дням со всеми исходными регрессорами. 
        В результате оказалось $RSS_{Warm} = 320$ и $RSS_{Cold} = 140$ соответственно. 
        Сформулируйте нулевую и альтернативную гипотезы подходящего теста и проведите его.
        % 2 балла
        % BB: прописать H_a
        \item (3 балла) Для тестирования гетероскедастичности Серёжа оценил вспомогательную регрессию на константу, 
        исходные переменные, их квадраты и попарные произведения по всем наблюдениям. 
        И получил в ней $R^2_{aux} = 0.3$. 
        Сформулируйте нулевую и альтернативную гипотезы подходящего теста и проведите его.
        % 3 балла
    \end{enumerate}
    
    
    \item Рассмотрим линейную регрессионную модель 
    $Y  = X\beta + \varepsilon$
    с диагональной ковариационной матрицей ошибок регрессии, где $\Var(\varepsilon_i) = c \cdot i$.
   
   Проверьте, выполнено ли для оценок модели свойство ортогональности $\hat Y$  и  $e = Y - \hat Y$, если
   \begin{enumerate}
       \item (1 балл) коэффициенты $\beta$ оценивают с помощью МНК;
       % 1 балл
       \item (3 балла) коэффициенты $\beta$ оценивают с помощью взвешенного МНК.
       % 3 балла
   \end{enumerate}
   
   Ответ обоснуйте.
    
    \item Величины $X_1$, \ldots, $X_n$ независимы и равномерно распределены на отрезке $[2a, 3a]$, где $a > 0$.
    
    \begin{enumerate}
        \item (2 балла) Найти оценку максимального правдоподобия параметра $a$.
        % 2 балла
        \item (3 балла) Является ли эта оценка несмещенной? Ответ обоснуйте.
    Если оценка является смещенной, то найдите величину смещения.
        % 3 балла
    % \item Посчитайте значение оценки для выборки $X_1 = 5$, $X_2 = 10$, $X_3 = 10$ и $X_4 = 7$.
    \end{enumerate}
% В разных вариантах эти 2a, 3a можно менять
    
    \item Исследователь Михаил 100 дней решал задачки по эконометрике вместе с гостящими у него друзьями. % подругами?
    Михаил предполагает, что число решённых задачек $Problems_i$ зависит от числа выпитых банок газировки $Drinks_i$ и количества съеденных пачек чипсов $Chips_i$. 
    Оценив соответствующую регрессию, он получил результат:
    
    \[
        \widehat{Problems}_i = 2.43 + 10.26 Drinks_i + 0.47 Chips_i
    \]
    
    Помимо этого, зная что ошибки в модели нормальны, он получил логарифм функции правдоподобия для этой модели: $\ell_{UR} = -535.82$. 
    Оценка ковариационной матрицы оценок коэффициентов имеет вид: 
    
    \[
        \begin{pmatrix}
        61.30 & -8.51 & 0 \\
        -8.51 & 2.13  & 0 \\
        0     & 0     & 1.20 \\
        \end{pmatrix}
    \]
    
    Из исследований других умных студентов известно, что очень часто влияние газировки и чипсов на эффективность учёбы оказывается одинаковым по величине, то есть $\beta_{Drinks} = \beta_{Chips}$. 
    Для проверки этой гипотезы Миша даже оценил вспомогательную регрессию и получил:
    
    \[
        \widehat{Problems}_i = 27.51 + 3.99 (Drinks_i + Chips_i)
    \]
    
    Логарифм правдоподобия для этой модели $\ell_R = -548.84$.

    Помогите Мише проверить гипотезу при помощи
    
    \begin{enumerate}
        \item (2 балла) $t$-теста; % 2 балла
        \item (2 балла) теста отношения правдоподобия; % 2 балла
        \item (2 балла) теста Вальда. % 2 балла
    \end{enumerate}
    % тест Вальда соответствует $t$-тесту. Это задумка?
    
    \item Маркетолог-аналитик Евгений пытается оценить зависимость предполагаемого объёма продаж, $y_i$, от размера инвестиций $x_i$,
\[    
y_i = \beta_1 + \beta_2 x_i + \varepsilon_i.
\]
Оказалось, что предполагаемый объем продаж измеряется с ошибкой $w_i$, то есть наблюдаемый объём продаж равен $z_i = y_i + w_i$.   

Поэтому маркетолог оценивает следующее уравнение регрессии:
\[ 
\hat z_i = \hat\beta_1 + \hat\beta_2 x_i
\]
Известно, что $\E(w_i)=\E(\varepsilon_i)=0$ и $\Var(w_i)=\sigma^2_w$, $\Var(\varepsilon_i)=\sigma^2_\varepsilon$. 
Ошибка $w_i$ не коррелирует с $x_i$ и не коррелирует с $y_i$. Ошибка $\varepsilon_i$ не коррелирует с $x_i$.
Наблюдения представляют собой случайную выборку.

\begin{enumerate}
    \item (2 балла) Найдите предел по вероятности для $\hat{\beta}_2$. 
    \item (1 балл) Является ли оценка состоятельной? Поясните.
\end{enumerate}


\item Амбидекстр Леонардо пишет левой и правой рукой с одинаковой ожидаемой скоростью $a$ слов в минуту. 
    
    Величина $L_i$ — количество слов, написанных левой рукой за минуту $i$, 
    величина $R_i$ — количество слов, написанных правой рукой за минуту $i$. 
    Величины, относящиеся к разным минутам, независимы. 
    
    Обратная ковариационная матрица вектора $(L_i, R_i)$ равна
    \[
        \Var^{-1} \begin{pmatrix}
        L_i \\
        R_i \\
        \end{pmatrix} = 
        \begin{pmatrix}
        4a^2 & a \\
        a & a^2+4 \\
        \end{pmatrix}.
    \]
    
    Пронаблюдав за Леонардо 100 минут я обнаружил, что $\bar L = 20$, а $\bar R = 30$.
    
    \begin{enumerate}
        \item (2 балла) Найдите оценку параметра $a$, используя обобщённый метод моментов с единичной взвешивающей матрицей. Обозначим эту оценку $\hat a$. % 2 балла 
        \item (1 балл) Оцените оптимальную взвешивающую матрицу, используя полученную предварительную оценку $\hat a$. % 1 балл
        \item (2 балла) Найдите финальную оценку параметра $a$, используя обобщённый метод моментов с оценённой взвешивающей матрицей. % 2 балла
        
    \end{enumerate}

    Уточнение: в качестве моментных условий используйте уравнения на $\E(L_i)$ и $\E(R_i)$.

    \item 
Для вероятности высоких кассовых сборов фильма в США за первые выходные (выше 40 млн. долл.) была оценена логистическая модель. Зависимая переменная  $Y$ равна $1$, если кассовые сборы больше 40 млн. долл. В модель включены регрессоры: 

$\text{Bud}$ — бюджет фильма в млн. долларов, 

$G$ — рейтинг кинокритиков в процентах (от 0 до 100), 

$W$ — жанр фильма: $W=1$ для фантастики или мультфильмов, $W=0$ в остальных случаях.

\[
\hat \P(y_i = 1) = \Lambda(-8.68 + 0.027 \text{Bud}_i + 0.052 G_i + 1.91 W_i)    
\]

\begin{enumerate}
% \item  Выпишите логарифмическую функцию правдоподобия, используемую для оценивания представленной линейной регрессионной модели. Выпишите процедуру получения оценок коэффициентов регрессии.
\item (2 балла) Найдите предельный  эффект влияния бюджета фильма при условии жанра $W=1$, рейтинга кинокритиков $G = 70\%$ и бюджета $50$ млн. долл. Проинтерпретируйте полученное значение. 
% \item (2 балла) Вариант 2. Определите предельный эффект влияния жанра на вероятность получить более высокие кассовые сборы при рейтинге кинокритиков $80\%$ и бюджете в $100$ млн. долл. Проинтерпретируйте полученное значение. 

\item (2 балла) Найдите максимальный предельный эффект влияния бюджета при $W=0$ и рейтинге кинокритиков $G=70\%$. Проинтерпретируйте полученное значение. 

\item (1 балл) Найдите значение чувствительности этой модели в точке отсечения $0.23$ по таблице ниже:

\begin{tabular}{@{}cccc@{}}
\toprule
           & $Y=1$ & $Y=0$ & Всего\\ \midrule
$\hat Y=1$ & 30 & 15 &  \\
$\hat Y=0$ & 5 & 80 &  \\
Всего &  &  & 130   \\ \bottomrule
\end{tabular}

Как изменится значение чувствительности при увеличении границы отсечения с $0.23$ до $0.5$? 
\end{enumerate}


\item Рассмотрим систему уравнений:
\[
\begin{cases}
y_1 = \beta_1 y_3 + \gamma_1 x_2 + u_1 \\
y_2 = \beta_2 y_1 + \beta_3 y_3 + \gamma_2 x_1 + \gamma_3 x_2 + u_2 \\
y_3 = \gamma_4 x_1 + u_3 \\
\end{cases}
\]

Наблюдения представляют собой случайную выборку. Переменные $y_i$ — эндогенные, переменные $x_i$ — экзогенные.

Для каждого из уравнений системы:

\begin{enumerate}
\item (2 балла) Проверьте выполнение условий порядка и условие ранга для каждого уравнения. 
\item  (1 балл) Проверьте идентифицируемость системы в целом. 
\item  (2 балла) Опишите схему получения оценок коэффициентов идентифицируемых уравнений. Укажите все регрессии, которые вы будете использовать. 

% \item  Укажите количество оцененных параметров при применении к этой модели методов 2SLS и 3SLS и распишите процедуру оценивания в каждом из случаев.
\end{enumerate}


Выберите \textbf{одну из двух} задач по выбору:

        
\item[9A.] Дополнительная задача 1. (2 балла) Приведите содержательный экономический пример, в котором следует использовать мультиномиальную логит модель.
 
\item[9B.] Дополнительная задача 2. (2 балла) Чем отличается тобит модель от модели Хекмана?
\end{enumerate}



\subsection{Контрольная 3, 2021-03-30, решения}

\begin{enumerate}
    \item 
    \begin{enumerate}
        \item Тест Голдфельда-Квандта, $F=\frac{320/(20-3)}{140/(20-3)}$.
        \item Тест Уайта.
    \end{enumerate}
    \item Важно! Фактическая ковариационная матрица здесь никак не влияет на ответ. 
    Эта задача про нестатистические свойства метода. 
    \begin{enumerate}
        \item Для МНК всегда $\hat Y \perp \hat Y - Y$. Это свойство метода следует из условий первого порядка. 
        \item Для взвешенного МНК с неравеными весами прогнозы не ортогональны остаткам. 
    \end{enumerate}
    \item 
    \begin{enumerate}
        \item $\hat a = \max \{X_1, \ldots, X_n\}$.
        \item 
    \end{enumerate}
    \item 
    \begin{enumerate}
        \item x
        \item $LR = 2(\ell_{UR} - \ell_R)$;
        \item Тест Вальда — это квадрат $t$-теста.
    \end{enumerate}
    \item 
    \begin{enumerate}
        \item $\plim \hat\beta_2 = \beta$;
        \item Оценка $\hat\beta_2$ состоятельная.
    \end{enumerate}
    \item 
    \begin{enumerate}
        \item Целевая функция
        \[
        \begin{pmatrix} a - 20 & a - 30 \end{pmatrix} \cdot I \cdot \begin{pmatrix} a - 20 \\ a - 30 \end{pmatrix} \to \min_a.
        \]
        Результат: $\hat a = 25$.
        \item Подставляем $\hat a$ в матрицу, обратную к ковариационной матрице вектора $(L_i, R_i)$, получаем $\hat W$.
        Можно общие множители выносить за матрицу, так как на точку экстремума общий множитель не влияет.
        \item Целевая функция
        \[
        \begin{pmatrix} a - 20 & a - 30 \end{pmatrix} \cdot \hat W \cdot \begin{pmatrix} a - 20 \\ a - 30 \end{pmatrix} \to \min_a.
        \]
    \end{enumerate}
    \item x
    \item x
    \item x
\end{enumerate}



\subsection{Предварительный экзамен, 2021-06-15}


\begin{enumerate}
    \item 
    Рассмотрим два уравнения на $Y_t$: 
    \[
    \text{Уравнение } A: Y_t=4 + 0.6 Y_{t-1} + \varepsilon_t,
    \]
    \[
    \text{Уравнение } B: Y_t=2 + 0.8 Y_{t-1}+1.5 X_{t-1} + \varepsilon_t.
    \]
    
    \begin{enumerate}
        % \item Есть ли у уравнения $A$ стационарное решение вида $MA(\infty)$ относительно $\varepsilon_t$? % (является ли модель $A$ стационарной)?
        \item (3 балла) Для стационарного процесса описываемого моделью $A$ найдите 
        $\Var(Y_t)$, $\Cov(Y_t, Y_{t-k})$ при $k>0$, $\Cov(Y_t, \varepsilon_{t-k})$ при $k>0$.
        \item (2 балла) Предположим, что $\varepsilon_t$ нормально распределены $\cN(0;16)$, независимы между собой, и не зависят от прошлых значений $Y_{t-k}$ и $X_{t-k}$ при $k>0$. Известно, что $Y_{100}=10$, $X_{100} = 2$ и $X_{101}=1$.
        
        Постройте 95\%-й предиктивный интервал для $Y_{102}$ в модели B. 
    \end{enumerate}
    

    \item  
    Процесс $X_t$ стационарен.
    \begin{enumerate}
        \item (2 балла) Верно ли, что стационарен процесс $Y_t = X_t - 5X_{t-1}$? Аргументируйте ответ. 
        
        \item (2 балла) Приведите пример такого стационарного процесса $Z_t$, что процесс $Y'_t = X_t - 5Z_t$ не стационарен. 
        
    \end{enumerate}
    
    \item  
    По 540 наблюдениям оценили зависимость почасовой заработной
    платы американцев (переменная EARNINGS) от длительности обучения (S), опыта работы (EXP), результатов теста (ASVABC), пола индивида (MALE = 1 для мужчин и 0 для женщин) по разным выборкам.

\begin{tabular}{@{}llrrr@{}}
\toprule
Выборка & Регрессия для $\text{EARNINGS}$ & $R^2$ & $n$ & $RSS$ \\ 
\midrule
все наблюдения & 
\begin{small} $-30.15 + 2.11 \text{S}_i + 0.38 \text{EXP}_i + 0.22 \text{ASVABC}_i + 6.15\text{MALE}_i$
\end{small} & $0.26$ & 540 & 85201 \\
белые американцы & \begin{small} $-32.49 + 2.24 \text{S}_i + 0.42 \text{EXP}_i + 0.22 \text{ASVABC}_i + 6.88\text{MALE}_i$
\end{small} & $0.25$ & 467 &  81096 \\
афроамериканцы & \begin{small} $-18.63 + 1.19 \text{S}_i + 0.21 \text{EXP}_i + 0.30 \text{ASVABC}_i + 2.33\text{MALE}_i$
\end{small} & $0.54$ & 39 & 763 \\ 
латиноамериканцы & \begin{small} $-17.77 + 2.54 \text{S}_i + 0.53 \text{EXP}_i - 0.19 \text{ASVABC}_i + 2.87\text{MALE}_i$
\end{small} & $0.34$ & 34 & 2210 \\ 
все кроме белых & \begin{small} $-17.42 + 1.58 \text{S}_i + 0.30 \text{EXP}_i + 0.15 \text{ASVABC}_i + 1.86\text{MALE}_i$
\end{small} & $0.36$ & 73 & 3277 \\ 

\bottomrule
\end{tabular}
    
    \begin{enumerate}
        \item (2 балла) Можно ли считать зависимость единой для афроамериканцев и латиноамериканцев?
        \item (3 балла) Можно ли считать зависимость единой для всех трех выделенных групп?
    \end{enumerate}
Обоснуйте каждый ответ подходящим тестом, аккуратно сформулировав основную и альтернативную гипотезу, выписав формулу для тестовой статистики.


\item 
Имеются панельные данные 100 фирм за 2000-2010 годы. Рассмотрим модель:
\[ 
y_{it} = \alpha_i + \beta x_{it} + \gamma z_i+ \varepsilon_{it}
\]

\begin{enumerate}
    \item (2 балла) Как устроено «within»-преобразование данных? Возможно ли получить «within»-оценку параметра $\gamma$? 
    \item (2 балла) Как устроено «between»-преобразование данных? Возможно ли получить «between»-оценку параметра $\gamma$? 
\end{enumerate}


\item 
По данным за 2005-2016 годы для 80 российских регионов была оценена зависимость уровня безработицы (unempl) от валового регионального продукта (grp) с помощью динамической модели:
\[
\ln \text{unempl}_{it}= \alpha_i + \delta \ln \text{unempl}_{it-1} + \beta \ln \text{grp}_{it} + \varepsilon_{it}.
\]
Результаты оценки модели с помощью обобщенного метода моментов в рамках подхода Ареллано и Бонда следующие: 
$\hat\delta = 0.52$, $\hat \beta = -0.09$.



 \begin{enumerate}
\item 	(3 балла) Объясните, почему в предложенной динамической модели возникает проблема эндогенности? 
Как эту проблему решают в рамках подхода Ареллано и Бонда? 
Какие переменные используют в качестве инструментов? 
Какие моменты?
\item (2 балла) Согласно оцененным результатам, как изменение ВРП влияет на изменение безработицы в краткосрочном и в долгосрочном периоде?
\end{enumerate}



\item 
Исследователь Михаил все лето (92 дня!) ходил гулять в парк и записывал цены на мороженое и температуру на улице. 
% Из-за того, что Миша довольно ленив, он записывал цену только в одном случайно выбранном из множества киосков с мороженым. 
Цена мороженого $Price_i$ в рублях зависит от температуры $Temp_i$ в градусах и дня недели: по выходным одна цена, по будням — другая (дамми-переменная $Weekend_i$ равна единице в выходные).

\[
Price_i = \beta_0 + \beta_{T} Temp_i + \beta_{W} Weekends_i + \varepsilon_i
\]

Дисперсия цен на мороженое тоже зависит от температуры и дня недели! 
По будням дисперсия всегда одинаковая и равна $\Var(\varepsilon_i) = 2\sigma^2$, а по выходным она растёт с ростом температуры: $\Var(\varepsilon_i) = 2 \sigma^2 Temp_i$. 

\begin{enumerate} 

% \item Каким образом нужно трансформировать данные, чтобы получить несмещённые оценки коэффициентов модели с помощью МНК?
% тут смело сказать, что нужно съесть мороженое и расслабиться

\item (2 балла) Каким образом нужно трансформировать данные, чтобы обычная оценка ковариационной матрицы оценок коэффициентов была состоятельной?


\end{enumerate}

Ленивый Миша решил, что бороться с гетероскедастичностью — слишком сложно, и гораздо проще выкинуть из рассмотрения выходные и оценивать модель только на данных по будням. 

\begin{enumerate}[resume]
\item (3 балла) Поможет ли это решить проблему с гетероскедастичностью? Какие другие проблемы могут появиться при таком подходе?



\end{enumerate}



\item 
Ответственная исследовательница Маша оценила логистическую регрессию:

\[
\hat \P(Y_i = 1) = \Lambda( \underset{(2.4)}{0.6} + \underset{(0.3)}{1.1} X_i + \underset{(0.07)}{0.5} D_i - \underset{(0.03)}{0.2} D_i X_i),
\]
где:
\begin{itemize}
\item $Y_i$ — дамми-переменная, равная единице, если студент успешно сдал экзамен, и нулю иначе; 
\item $D_i$ — дамми-переменная, равная единице, если студент ходил накануне в бар, и нулю иначе;
\item $X_i$ — количество часов подготовки к экзамену. 
\end{itemize}

В скобках указаны стандартные отклонения, число наблюдений велико.

\begin{enumerate} 

\item (2 балла) Проверьте гипотезу о том, что поход в бар не влияет на эффективность подготовки к экзамену.

\item (2 балла) Найдите предельный эффект дополнительного часа Машиной подготовки к экзамену. 
Маша готовилась один час, а оставшееся время провела в баре.

\end{enumerate}


\item 
Винни-Пух с опилками в голове изучает модель парной регрессии $Y_i = \beta_1 + \beta_x X_i + u_i$.
Ошибки $u_t$ строго экзогенны, независимы и имеют нормальное распределение $u_t \sim \cN(0;16)$.

Всего есть $n=100$ наблюдений, $\bar X = 1$, $\bar Y= 5$, $\hat \beta_x^{ols} = 3$, $\sum_i X_i^2=200$, $\sum_i Y_i^2=10000$.


Винни-Пух хочет проверить гипотезу $H_0$: $\beta_x = 2$ против альтернативной $H_0$: $\beta_x \neq 2$ с помощью теста отношения правдоподобия и никак иначе. 


(4 балла) Помогите ему это сделать!


\end{enumerate}



\subsection{Предварительный экзамен, 2021-06-15, подсказки}

\begin{enumerate}
\item
\item 
\begin{enumerate}
    \item Да верно, просто выписывается ковариация и видно, что она не зависит от $t$.
    \item Например, можно определить $Z_1 = X_1$, а остальные $Z_t$ определить независимыми величинами с такой же дисперсией.
\end{enumerate}
\item
\item

\item Краткосрочный эффект равен $\hat \beta = -0.09$, долгосрочный — $\hat \beta/(1 - \hat\delta) = -0.09 / 0.52$, 
\item \begin{enumerate}
    \item Просто описать WLS, только с учётом хитрой структуры дисперсии.
    \item Сказать что гетероскедастичности не останется, эффективность оценок будет потеряна, так как потеряли наблюдения. Эндогенности не будет.
\end{enumerate}
\item
\item Выписываем разность лог-плотностей многомерного нормального распределение, домноженную на два.
\end{enumerate}


\subsection{Экзамен, 2021-06-21}

Можно использовать шпаргалку А4 с двух сторон с любым содержимым. Продолжительность 120 минут плюс 10 минут на загрузку работ.
Таблицы доступны по ссылке

\url{https://github.com/bdemeshev/probability_hse_exams/raw/master/tables/tables_all.pdf}.




\begin{enumerate}

\item % автор: Мила вариант 1
Стационарный процесс $Y_t$ описывается системой 
 \[
 \begin{cases}
   Y_t=0.3 + 0.8 Y_{t-2} + u_t, \\
   u_t = \varepsilon_t - 0.5 \varepsilon_{t-1},   \\
   \varepsilon_t \sim \cN(0; 4) 
  \end{cases}
\]

Величины $\varepsilon_t$ независимы. 

\begin{enumerate}
% \item (1 балл) Найдите $\E(Y_t)$.
\item (1 балл) Найдите дисперсию $\Var(u_t)$. Являются ли ошибки $u_t$ гомоскедастичными? 
% да, являются
\item (2 балла) Найдите ковариацию $\Cov(Y_9, u_2)$.
% находим явную формулу для Y_9
% \[
% Y_9 = const + u_9 + 0.8 u_7 + 0.8^2 u_5 + 0.8^3 u_3 + 0.8^5 u_1 + \ldots
% \]
% отсюда находим ковариацию помня, что $u_2$ зависимо с $u_1$ и $u_3$

\item (2 балла) Постройте 95\%-й предиктивный интервал для $Y_{101}$, 
 если $Y_{100} = 2$, $Y_{99} = 3$, $\varepsilon_{99}=-0.1$.
% вытаскиваем $\varepsilon_{100}$ из имеющейся информации
% строим предиктивный интервал пользуясь нормальностью ошибок
\end{enumerate}
    

\item  % автор: Ольга Анатольевна Вариант 1

     По 540 наблюдениям оценили зависимость почасовой заработной
    платы американцев (переменная EARNINGS) от длительности обучения (S) и пола индивида (MALE = 1 для мужчин и 0 для женщин) по разным выборкам.

\begin{tabular}{@{}llrrr@{}}
\toprule
Выборка & Регрессия для $\text{EARNINGS}$ & $R^2$ & $n$ & $RSS$ \\ 
\midrule
все наблюдения & 
\begin{small} $-12.61 + 2.36 \text{S}_i$
\end{small} & $0.17$ & 540 & 95222 \\
все наблюдения & \begin{small} $-8.24 + 1.78 \text{S}_i - 6.34 \text{MALE}_i + 0.97 \text{S}_i\cdot \text{MALE}_i$
\end{small} & $0.24$ & 540 &  87793 \\
проживающие на западе & \begin{small} $-1.05 + 1.69 \text{S}_i$
\end{small} & $0.07$ & 91 & 20415 \\ 
проживающие на севере & \begin{small} $-19.96 + 2.86 \text{S}_i$
\end{small} & $0.26$ & 260 & 34579 \\ 
проживающие на юге & \begin{small} $-8.41 + 2.04 \text{S}_i$
\end{small} & $0.13$ & 189 & 38989 \\ 

\bottomrule
\end{tabular}
    
    \begin{enumerate}
        \item (2 балла) Предположим, что модель едина для всех штатов. Можно ли считать зависимость единой для мужчин и женщин?
       
        \item (2 балла) Предположим, что модель едина для мужчин и женщин. 
        Можно ли считать зависимость единой для проживающих в западных, северных и южных штатах?
    \end{enumerate}
Обоснуйте каждый ответ подходящим тестом, аккуратно сформулировав основную и альтернативную гипотезу, выписав формулу для тестовой статистики.

\newpage
\item % автор: Ольга Анатольевна Вариант 1 

Для оценки зависимости прямых иностранных инвестиций, нормированных на ВВП (invest), от уровня  безработицы (unempl) была использована модель:
\[
\text{invest}_{t}= \beta_0 + \delta \text{invest}_{t-1} + \beta_1 \text{unempl}_{t} + \beta_2 \text{unempl}_{t-1} + \varepsilon_{t},
\]
где переменная $unempl$ рассматривается как экзогенная, а ошибки $\varepsilon_t$ удовлетворяют марковской схеме первого порядка.

 \begin{enumerate}
\item 	(3 балла) Можно ли получить состоятельные оценки параметров с помощью МНК? Если нет, то опишите способ получения состоятельных оценок.
% Нельзя. IV. ИЛИ ML, если сделать доп предположение о нормальном распределении ошибок.
\item (2 балла) 
Состоятельные оценки параметров, полученные с использованием российских данных за 1990-2020, следующие: 
$\hat\delta = 0.3$,
$\hat\beta_0 = -2.1$,
$\hat\beta_1 = -0.08$, $\hat \beta_2 = -0.06$.
% в скобках указаны стандартные ошибки оценок коэффициентов.

Оцените краткосрочный и долгосрочный эффект от изменения безработицы на один процентный пункт (unempl) на прямые инвестиции, нормированные на ВВП (invest). 

% Согласно оцененным результатам, как изменение безработицы влияет на изменение прямых иностранных инвестиций, нормированных на ВВП, в краткосрочном и в долгосрочном периоде?

\end{enumerate}



\item % автор: Лена Вариант 1

Марина решила оценить влияние экологических инноваций на прибыль фирм. Она скачала информацию о доле затрат на экологические инновации ($a$), количестве сотрудников ($b$), затратам на R\&D ($c$)  и прибыли ($y$) по выборке из 100 фирм за 5 последних лет. 
\[ 
y_{it} = \beta_0 + \beta_1 a_{it} +  \beta_2 b_{it} + \beta_3 c_{it} + \varepsilon_{it}
\]
Оценив модель с помощью МНК, она задумалась о том, что фирмы в ее выборке очень разные, и что прибыль может различаться по множеству различных причин. 
А собрать всю возможную информацию об этих фирмах для Марины невыполнимая задача.
\begin{enumerate}
    \item (2 балла) Предложите Марине модель, благодаря которой она сможет учесть недостающую информацию, не обращаясь к базам данных. Запишите спецификацию модели. Какие характеристики фирм учитывает данная модель? Приведите пример. 

    \item (2 балла) Марина также провела тест Хаусмана для сравнения RE и FE оценок и получила значение статистики 1.43.  
    Как распределена статистика и сколько имеет степеней свободы? 
    Какую модель Марине необходимо выбрать в соответствии с результатами теста?
    Запишите необходимые предпосылки выбранной модели.
\end{enumerate}



\item %автор: Ваня

Исследователь Михаил собрал данные по успеваемости студентов двух групп, в первой учится 5 студентов, во второй — 10. Михаил предполагает, что оценка по метрике $\text{Metrics}_i$ должна зависеть от оценки по матстату $\text{Stat}_i$. Беда в том, что в первую группу отобрали только отличников по статистике, а во вторую брали всех подряд, из-за чего, как кажется Михаилу, дисперсия ошибок в двух группах может отличаться. 

\begin{enumerate}
    \item (4 балла) Предложите какой-нибудь тест, с помощью которого можно было бы проверить наличие гетероскедастичности в таких условиях. Укажите необходимые предпосылки, нулевую и альтернативную гипотезы, расчётную статистику и её распределение. 
    
    \item (1 балла) Предположим, что дисперсии ошибок в двух группах известны и Михаил может оценить модель при помощи обобщённого МНК. Какой вид будет в таком случае иметь ковариационная матрица ошибок $\Omega$?
    
\end{enumerate}


\newpage
\item %автор: Ваня

Вероятность того, что трудолюбивая исследовательница Лена пойдёт гулять с друзьями, зависит от температуры на улице $Temp_i$ и количества экзаменов в ближайшие три дня $Exams_i$. Пронаблюдав за Леной 100 дней, исследователь Николай смог оценить две логистические регрессии:

\[
\hat \P(Y_i = 1) = \Lambda( \underset{(1.3)}{0.9} + \underset{(5.8)}{11.2} Temp_i - \underset{(0.03)}{0.2} Temp_i^2 - \underset{(0.03)}{0.2} Exams_i ),
\]

\[
\hat \P(Y_i = 1) = \Lambda( \underset{(2.3)}{0.7} - \underset{(0.09)}{0.4} Exams_i ).
\]

Известно также, что логарифм функции правдоподобия для первой модели оказался равен $\ln L_1 = -400$, для второй модели $\ln L_2 = -420$. 
В скобках указаны стандартные ошибки.

\begin{enumerate}
    \item (2 балла) Помогите Николаю проверить гипотезу о том, что температура не оказывает никакого влияния на вероятность того, что Лена пойдёт гулять.
    
    \item (2 балла) При каких условиях предельный эффект температуры в первой модели будет нулевым?
    
\end{enumerate}



\item  % автор: Боря

Иногда сезонность моделируют с помощью тригонометрических функций. Например, для месячной сезонной составляющей $s_t$ можно использовать модель
\[
\begin{pmatrix}
s_t \\
s_t^*
\end{pmatrix} = 
\begin{pmatrix}
\cos \lambda & \sin \lambda \\
-\sin \lambda & \cos \lambda \\
\end{pmatrix} \cdot 
\begin{pmatrix}
s_{t-1} \\
s_{t-1}^*
\end{pmatrix} +
\begin{pmatrix}
w_t \\
w_t^*
\end{pmatrix},   
\]
где $\lambda = 2\pi/12$, белые шумы $w_t$ и $w_t^*$ некоррелированы и одинаково распределены $\cN(0;4)$. Шумы $w_t$ и $w_t^*$ не зависят от предыдущих значение $s_{t-k}$ и $s_{t-k}^*$ при $k>0$.

Процессы $s_t$ и $s_t^*$ имеют постоянное математическое ожидание\footnote{было ошибочно написано, что процесс стационарный, это неверно из-за непостоянной дисперсии}. 

\begin{enumerate}
    \item (2 балла) Найдите ожидание $\E (s_t)$.
    \item (3 балла) Найдите условное ожидание $\E(s_t \mid s_{t-12})$.
\end{enumerate}


\item % автор: Боря
Для борьбы с инфляцией оценок всех студентов отправили на комиссию\footnote{Неплохая идея, а?}.

Поэтому за каждую работу есть три оценки: $x_i$ — оценка первого проверяющего, $a_i$ и $b_i$ — оценки второго и третьего проверяющих. 

Наблюдения представляют собой случайную выборку и хорошо описываются моделью:
\[
\begin{cases}
a_i = \beta x_i + u_i, u_i \sim \cN(0; \sigma^2) \\
b_i = \beta x_i + \varepsilon_i, \varepsilon_i \sim \cN(0; \sigma^2) \\
\end{cases}
\]

\begin{enumerate}
    \item (1 балл) Найдите оценку метода максимального правдоподобия для $\beta$,
    предполагая второго и третьего проверяющего независимыми.
    
    
    \item (3 балла) Найдите оценку метода максимального правдоподобия для $\beta$,
    предполагая что корреляция между $u_i$ и $\varepsilon_i$ равна $\rho = 0.5$, а вектор ошибок $(u_i, \varepsilon_i)$ имеет двумерное нормальное распределение. 
    
\end{enumerate}

\end{enumerate}




\subsection{Экзамен, 2021-06-21, решения}

\begin{enumerate}
    \item \begin{enumerate}
    \item да, являются
    \item Находим явную формулу для $Y_9$:
    \[
    Y_9 = const + u_9 + 0.8 u_7 + 0.8^2 u_5 + 0.8^3 u_3 + 0.8^5 u_1 + \ldots
    \]
    Отсюда находим ковариацию помня, что $u_2$ зависимо с $u_1$ и $u_3$.

    \item Обозначим:
    \[
    x = \begin{pmatrix}
        Y_{100} \\
        Y_{99} \\
        \varepsilon_{99}
    \end{pmatrix}    
    \]
    Считаем условное ожидание:
    \[
    \E(Y_{101} \mid x) = 0.3 + 0.8 \cdot 3 + 0 -0.5 \E(\varepsilon_{100} \mid x)    
    \]
    Ба, да это же теоретическая регрессия!
    \[
    \E(\varepsilon_{100} \mid x) =  \Cov(\varepsilon_{100}, x) \Var^{-1}(x) (x - \E(x)),
    \]
    где
    \[
        \Cov(\varepsilon_{100}, x) = \begin{pmatrix}
            4 & 0 & 0
        \end{pmatrix}, \quad \Var(x) = \begin{pmatrix}
            500/36 & -10 & -2 \\
            -10 & 500/36 & 4 \\
            -2 & 4 & 4 \\
        \end{pmatrix}, \quad
        \E(x) = \begin{pmatrix}
            1.5 \\
            1.5 \\
            0 
        \end{pmatrix}
    \]

    С дисперсией:
    \[
    \Var(Y_{101} \mid x) = 4 + 0.25 \Var(\varepsilon_{100} \mid x) = 
    4  + 0.25 (\Var(\varepsilon_{100}) - \Cov(\varepsilon_{100}, x) \Var^{-1}(x) \Cov(x, \varepsilon_{100}))   
    \]


    Строим предиктивный интервал пользуясь нормальностью ошибок.
    \end{enumerate}
    \item \begin{enumerate}
            \item Проверить, что коэффициенты при переменных $MALE_i$ и $S_i\cdot MALE_i$ одновременно равны 0. 

            \item Надо провести тест Чоу для 3-х групп.
    \end{enumerate}

\item \begin{enumerate}
\item  Нельзя. IV или ML, если сделать доп предположение о нормальном распределении ошибок.
\item xxxx
\end{enumerate}
    
\item \begin{enumerate}
    \item Нужно включить в регрессию индивидуальные эффекты — RE или FE. 
    В такой регрессии буду учитываться ненаблюдаемые характеристики, постоянные во времени. 
    Примеры: форма собственности фирмы, расположение фирмы (регион) и другие характеристики, не меняющиеся во времени.

\item  Тест Хаусмана.  
Статистика теста имеет хи-квадрат распределение с 3 степенями свободы.
$\chi^2_{crit, 5\%}=7.8$. 

При $\chi^2 (3)=1.43$  гипотеза об отсутствии разницы между оценками не отвергается. RE оценки более эффективны.
Предпосылки RE модели: $\mu_i \sim (0,\sigma^2_{\mu})$, $\varepsilon_{it} \sim (0, \sigma^2_{\varepsilon})$, некоррелированность  индивидуальных эффектов и регрессоров.

При $\chi^2 (3)=14.3$  гипотеза об отсутствии разницы между оценками отвергается. Только FE оценки состоятельны.
Предпосылки FE модели: $\varepsilon_{it} \sim (0, \sigma^2_{\varepsilon})$.
\end{enumerate}

\item \begin{enumerate}
    \item Описываем аналог теста Голдфелда-Куандта, только с делением по группе и выборки будут разного размера, 
    RSS надо делить правильно.
    
    \item Матрица, где по диагонали сперва одно число, потом — другое.
    
\end{enumerate}

\item \begin{enumerate}
    \item LR тест, всё для него есть.
    
    \item Берём производную, получаем, что предельный эффект равен плотности в точке домножить на $(11.2 - 0.4Temp)$. 
    Плотность ненулевая везде, значит, нужно, чтобы $11.2 - 0.4Temp=0$. Отсюда находим температуру.
    
\end{enumerate}

\item \begin{enumerate}
    \item Решаем систему с двумя нулевыми неизвестными. Система имеет единственное решение, так как определитель равен единице.
    \[
    \begin{pmatrix}
    \E(s_t) \\
    \E(s_t^*)
    \end{pmatrix} = 
    \begin{pmatrix}
    \cos \lambda & \sin \lambda \\
    -\sin \lambda & \cos \lambda \\
    \end{pmatrix} \cdot 
    \begin{pmatrix}
    \E(s_{t-1}) \\
    \E(s_{t-1}^*)
    \end{pmatrix} +
    \begin{pmatrix}
    0 \\
    0
    \end{pmatrix},   
    \]
    По условию нас интересует решение с $\E(s_t) = \E(s_{t-1})$ и $\E(s_t^*) = \E(s_{t-1}^*)$.
    Решение единственно, $\E(s_t) =0$.

    \item При возведении матрицы в 12-ю степень получим единичную, $A^{12}=I$, поэтому ответ $s_{t-12}$, подробнее:
    
    Запишем в векторной форме:
    \[
    y_t = A y_{t-1} + u_t,    
    \]
    где $y_t = \begin{pmatrix}
        s_t \\ s_t^*
    \end{pmatrix}$, $u_t = \begin{pmatrix}
        w_t \\ w_t^*
    \end{pmatrix}$.

    Отсюда:
    \[
    y_t = u_t + Au_{t-1} + A^2 u_{t-2} + \ldots + A^{11} u_{t-11} + A^{12} y_{t-12}
    \]

    Замечаем, что $\E(y \mid y_{t-12}) = A^{12} y_{t-12}$. При этом $A^{12}=I$.

\end{enumerate}

\item \begin{enumerate}
    \item Можно просто написать данные по первому и второму проверяющему друг под другом, считая их отметки игреком. затем применить МНК.
    
    \item Потребуется выписать многомерную нормальную плотность.
    Поскольку $\rho$ фиксирована, то максимум правдоподобия совпадает с минимумом функции
    \[
       \sum \begin{pmatrix} a_i - \beta x_i & b_i - \beta x_i \end{pmatrix} V^{-1} \begin{pmatrix} a_i - \beta x_i \\ b_i - \beta x_i \end{pmatrix}
    \]
    Относительно $\beta$ целевая функция — парабола.
\end{enumerate}

        
\end{enumerate}
