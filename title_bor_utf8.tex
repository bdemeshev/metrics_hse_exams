% !TEX root = metrics_hse_exams.tex

\usepackage{libertine}

\usepackage{fontspec}
\usepackage{polyglossia}
\usepackage{csquotes}

\setmainlanguage{russian}
\setotherlanguages{english}

% download "Linux Libertine" OTF-fonts:
% http://www.linuxlibertine.org/index.php?id=91&L=1
% \setmainfont{Linux Libertine O} % or Helvetica, Arial, Cambria
% why do we need \newfontfamily:
% http://tex.stackexchange.com/questions/91507/
% \newfontfamily{\cyrillicfonttt}{Linux Libertine O}
% \newfontfamily{\cyrillicfont}{Linux Libertine O}
% \newfontfamily{\cyrillicfontsf}{Linux Libertine O}

\usepackage{etoolbox} % to use ifdef, must be after babel


\usepackage[paper=a4paper, top=13.5mm, bottom=13.5mm, left=16.5mm, right=13.5mm, includefoot]{geometry}

% \usepackage{etex} % расширение классического tex
% в частности позволяет подгружать гораздо больше пакетов, чем мы и займёмся далее

\usepackage{floatrow} % сильно вниз если сдвинуть - ругается!


\usepackage{makeidx} % для создания предметных указателей
\usepackage{verbatim} % для многострочных комментариев
%\usepackage[pdftex]{graphicx} % для вставки графики
% omit pdftex option if not using pdflatex


%\usepackage{dsfont} % шрифт для единички с двойной палочкой (для индикатора события)
\usepackage{bbm} % шрифт - двойные буквы


\usepackage[usenames, dvipsnames, svgnames, table, rgb]{xcolor}

\usepackage{colortbl}

\usepackage{comment} % для комментирования кусков

% пакет для тестов:
\usepackage[box, % запрет на перенос вопросов
nopage, % убираем колонтитулы страницы
insidebox, % ставим буквы в квадратики
separateanswersheet, % добавляем бланк ответов
nowatermark, % отсутствие надписи "Черновик"
indivanswers,  % показываем верные ответы
%answers,
lang=RU, % локализация слов "нет верных ответов", "вопрос" и тд
completemulti % добавлять "нет правильного ответа" во всех вопросах множественного выбора
]{automultiplechoice}


\usepackage{multicol}
\usepackage{multirow} % Слияние строк в таблице


\usepackage[colorlinks, hyperindex, unicode, breaklinks]{hyperref} % гиперссылки в pdf

\usepackage{dcolumn}

\usepackage{amssymb}
\usepackage{amsmath}
\usepackage{amsthm}
\usepackage{epsfig}
\usepackage{bm}
\usepackage{color}

\usepackage{todonotes} % для вставки в документ заметок о том, что осталось сделать
% \todo{Здесь надо коэффициенты исправить}
% \missingfigure{Здесь будет картина Последний день Помпеи}
% команда \listoftodos — печатает все поставленные \todo'шки


\usepackage{textcomp}  % Чтобы в формулах можно было русские буквы писать через \text{}

%\usepackage{embedfile} % Чтобы код LaTeXа включился как приложение в PDF-файл

\usepackage{subfigure} % для создания нескольких рисунков внутри одного

\usepackage{tikz, pgfplots} % язык для рисования графики из latex'a
\usetikzlibrary{trees} % прибамбас в нем для рисовки деревьев
\usetikzlibrary{arrows} % прибамбас в нем для рисовки стрелочек подлиннее
\usepackage{tikz-qtree} % прибамбас в нем для рисовки деревьев




\usepackage{enumitem}


%\embedfile[desc={Исходный LaTeX файл}]{\jobname.tex} % Включение кода в выходной файл
%\embedfile[desc={Стилевой файл}]{title_bor_utf8.tex}



% вместо горизонтальной делаем косую черточку в нестрогих неравенствах
\renewcommand{\le}{\leqslant}
\renewcommand{\ge}{\geqslant}
\renewcommand{\leq}{\leqslant}
\renewcommand{\geq}{\geqslant}

% делаем короче интервал в списках
\setlength{\itemsep}{0pt}
\setlength{\parskip}{0pt}
\setlength{\parsep}{0pt}

% свешиваем пунктуацию (т.е. знаки пунктуации могут вылезать за правую границу текста, при этом текст выглядит ровнее)
\usepackage{microtype}

% \usepackage{physics}
\usepackage{mathtools}
\DeclarePairedDelimiter{\abs}{\lvert}{\rvert}
\DeclarePairedDelimiter{\norm}{\lVert}{\rVert}

% более красивые таблицы
\usepackage{booktabs}
% заповеди из докупентации:
% 1. Не используйте вертикальные линни
% 2. Не используйте двойные линии
% 3. Единицы измерения - в шапку таблицы
% 4. Не сокращайте .1 вместо 0.1
% 5. Повторяющееся значение повторяйте, а не говорите "то же"

\DeclareMathOperator*{\argmin}{arg\,min}
\DeclareMathOperator*{\argmax}{arg\,max}
\DeclareMathOperator*{\amn}{arg\,min}
\DeclareMathOperator*{\amx}{arg\,max}
\DeclareMathOperator{\Var}{Var}
\DeclareMathOperator{\Cov}{Cov}
\DeclareMathOperator{\Corr}{Corr}

\DeclareMathOperator{\card}{card}
\DeclareMathOperator{\trace}{trace}
\DeclareMathOperator{\tr}{trace}
\DeclareMathOperator{\rank}{rank}

\DeclareMathOperator{\dist}{dist}
\DeclareMathOperator{\sign}{sign}
\DeclareMathOperator{\sgn}{sign}

\DeclareMathOperator{\col}{col}
\DeclareMathOperator{\row}{row}




\let \P\relax
\DeclareMathOperator{\P}{\mathbb{P}}


\newcommand \R{\mathbb R}
\newcommand \N{\mathbb N}
\newcommand \Z{\mathbb Z}

\newcommand \RR{\mathbb R}
\newcommand \NN{\mathbb N}
\newcommand \ZZ{\mathbb Z}


\newcommand{\SSR}{SS^{\text{res}}}
\newcommand{\SSE}{SS^{\text{expl}}}
\newcommand{\SST}{SST}

%на всякий случай пока есть
%теоремы без нумерации и имени
\newtheorem*{theorem}{Теорема}

%"Определения","Замечания"
%и "Гипотезы" не нумеруются
\newtheorem*{definition}{Определение}
%\newtheorem*{rem}{Замечание}
%\newtheorem*{conj}{Гипотеза}

%"Теоремы" и "Леммы" нумеруются
%по главам и согласованно м/у собой
%\newtheorem{theorem}{Теорема}
%\newtheorem{lemma}[theorem]{Лемма}

% Утверждения нумеруются по главам
% независимо от Лемм и Теорем
%\newtheorem{prop}{Утверждение}
%\newtheorem{cor}{Следствие}




\newcommand \useR{$[$R$]$ }

%% эконометрические сокращения
\DeclareMathOperator{\sVar}{sVar}
\DeclareMathOperator{\E}{\mathbb{E}}
\DeclareMathOperator{\sCov}{sCov}
\DeclareMathOperator{\sCorr}{sCorr}
\DeclareMathOperator \hVar{\widehat{\Var}}
\DeclareMathOperator \hCorr{\widehat{\Corr}}
\DeclareMathOperator \hCov{\widehat{\Cov}}

\DeclareMathOperator*{\plim}{plim}
\DeclareMathOperator{\Lin}{Lin}


\newcommand \hb{\hat{\beta}}
\newcommand \hs{\hat{s}}
\newcommand \hy{\hat{y}}
\newcommand \hY{\hat{Y}}
\newcommand \he{\hat{\varepsilon}}
\newcommand \vone{\vec{1}}
\newcommand \cN{\mathcal{N}}
\newcommand \e{\varepsilon}
\newcommand \z{z}


% \DeclareMathOperator{\tr}{tr}

%% лаг
\renewcommand{\L}{\mathrm{L}}

%% алая и белая розы
%% запускается так: \WhiteRose[масштаб], например, \WhiteRose[0.5]
\newcommand{\WhiteRose}[1]{\begingroup
\setbox0=\hbox{\includegraphics[scale=#1]{figures/Yorkshire_rose.pdf}}%
\parbox{\wd0}{\box0}\endgroup}

\newcommand{\RedRose}[1]{\begingroup
\setbox0=\hbox{\includegraphics[scale=#1]{figures/Lancashire_rose.pdf}}%
\parbox{\wd0}{\box0}\endgroup}

\newcommand{\WhiteRoseLine}{
\begin{center}
\WhiteRose{0.3} Версия Белой Розы \WhiteRose{0.3}
\end{center}}

\newcommand{\RedRoseLine}{
\begin{center}
\RedRose{0.3} Версия Алой Розы \RedRose{0.3}
\end{center}}
