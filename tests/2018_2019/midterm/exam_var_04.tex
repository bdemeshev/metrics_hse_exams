\documentclass[12pt]{article}

\usepackage{tikz} % картинки в tikz
\usepackage{microtype} % свешивание пунктуации

\usepackage{array} % для столбцов фиксированной ширины

\usepackage{indentfirst} % отступ в первом параграфе

\usepackage{sectsty} % для центрирования названий частей
\allsectionsfont{\centering}

\usepackage{amsmath, amsthm} % куча стандартных математических плюшек

\usepackage{comment}
\usepackage{amsfonts}

\usepackage[top=2cm, left=1cm, right=1cm, bottom=2cm]{geometry} % размер текста на странице

\usepackage{lastpage} % чтобы узнать номер последней страницы

\usepackage{enumitem} % дополнительные плюшки для списков
%  например \begin{enumerate}[resume] позволяет продолжить нумерацию в новом списке
\usepackage{caption}

\usepackage{hyperref} % гиперссылки

\usepackage{multicol} % текст в несколько столбцов


\usepackage{fancyhdr} % весёлые колонтитулы
\pagestyle{fancy}
\lhead{Эконометрика}
\chead{2018-12-27}
\rhead{Промежуточный экзамен}
\lfoot{Вариант $\delta$}
\cfoot{}
\rfoot{}
\renewcommand{\headrulewidth}{0.4pt}
\renewcommand{\footrulewidth}{0.4pt}



\usepackage{todonotes} % для вставки в документ заметок о том, что осталось сделать
% \todo{Здесь надо коэффициенты исправить}
% \missingfigure{Здесь будет Последний день Помпеи}
% \listoftodos --- печатает все поставленные \todo'шки


% более красивые таблицы
\usepackage{booktabs}
% заповеди из докупентации:
% 1. Не используйте вертикальные линни
% 2. Не используйте двойные линии
% 3. Единицы измерения - в шапку таблицы
% 4. Не сокращайте .1 вместо 0.1
% 5. Повторяющееся значение повторяйте, а не говорите "то же"


\usepackage{fontspec}
\usepackage{polyglossia}

\setmainlanguage{russian}
\setotherlanguages{english}

% download "Linux Libertine" fonts:
% http://www.linuxlibertine.org/index.php?id=91&L=1
\setmainfont{Linux Libertine O} % or Helvetica, Arial, Cambria
% why do we need \newfontfamily:
% http://tex.stackexchange.com/questions/91507/
\newfontfamily{\cyrillicfonttt}{Linux Libertine O}

\AddEnumerateCounter{\asbuk}{\russian@alph}{щ} % для списков с русскими буквами
\setlist[enumerate, 2]{label=\asbuk*),ref=\asbuk*}

%% эконометрические сокращения
\DeclareMathOperator{\Cov}{Cov}
\DeclareMathOperator{\Corr}{Corr}
\DeclareMathOperator{\Var}{Var}
\DeclareMathOperator{\E}{E}
\def \hb{\hat{\beta}}
\def \hs{\hat{\sigma}}
\def \htheta{\hat{\theta}}
\def \s{\sigma}
\def \hy{\hat{y}}
\def \hY{\hat{Y}}
\def \v1{\vec{1}}
\def \e{\varepsilon}
\def \he{\hat{\e}}
\def \z{z}
\def \hVar{\widehat{\Var}}
\def \hCorr{\widehat{\Corr}}
\def \hCov{\widehat{\Cov}}
\def \cN{\mathcal{N}}
\def \P{\mathbb{P}}

%%%%%%%%%%%
% блок для тестов
%%%%%%%%%%%
% [1][3] 1 = one argument, 3 = value if missing
% эта магия создаёт окружение answerlist
% именно в окружении answerlist записаны варианты ответов в подключаемых exerciseXX
% просто \begin{answerlist} сделает ответы в три столбца
% если ответы длинные, то надо в них руками сделать
% \begin{answerlist}[1] чтобы они шли в один столбец
\newenvironment{answerlist}[1][3]{
\begin{multicols}{#1}
\begin{enumerate}[label=\fbox{\emph{\Alph*}},ref=\emph{\alph*}]
}
{
\end{enumerate}
\end{multicols}
}


\excludecomment{solution} % without solutions

\theoremstyle{definition}

% опция [subsection] для сброса счётчика вопросов после каждой subsection
\newtheorem{question}{Вопрос}[subsection]

% чтобы номер вопроса был без номера секции:
\renewcommand{\thequestion}{\arabic{question}}
% конец блока для тестов
%%%%%%%%%%%%


\begin{document}

\fbox{
  \begin{minipage}{42em}
    Имя, фамилия и номер группы:\vspace*{3ex}\par
    \noindent\dotfill\vspace{2mm}
  \end{minipage}
}

\vspace{30pt}

Ответы на тест внесите в таблицу:

\begin{tabular}{|m{3cm}|m{1cm}|m{1cm}|m{1cm}|m{1cm}|m{1cm}|m{1cm}|m{1cm}|m{1cm}|m{1cm}|m{1cm}|}
\toprule
Вопрос теста & 1 &  2 & 3 & 4 & 5 & 6 & 7 & 8 & 9 & 10 \\
\hline
Ответ &  &  & & & & & & & & \\
  &  &  & & & & & & & & \\
\bottomrule
\end{tabular}

\vspace{30pt}

Удачи! :)

\vspace{30pt}


Таблица заполняется проверяющим работу:

\begin{tabular}{|m{2cm}|m{2cm}|m{2cm}|m{2cm}|m{2cm}|m{2cm}|m{2cm}|}
\toprule
Тест & 1 &  2 & 3 & 4 & 5 & Итого \\
\midrule
&  &  & & & & \\
 &  &  & & & & \\
\bottomrule
\end{tabular}



\newpage

\fbox{
  \begin{minipage}{42em}
    Имя, фамилия и номер группы:\vspace*{3ex}\par
    \noindent\dotfill\vspace{2mm}
  \end{minipage}
}


\section*{Тест}



\begin{question}
Стьюдентизированные остатки регрессии используются
\begin{answerlist}
  \item в тесте Саргана
  \item на первом шаге двухшагового МНК
  \item на первом шаге при проведении теста Годфельда-Квандта
  \item в методе главных компонент
  \item для выявления выбросов
\end{answerlist}
\end{question}




\begin{question}
Тест Саргана для проверки валидности инструментов можно использовать
только в том случае, если число инструментов
\begin{answerlist}
  \item меньше числа эндогенных переменных
  \item больше числа эндогенных переменных
  \item совпадает с числом эндогенных переменных
  \item совпадает с числом экзогенных переменных
  \item меньше числа экзогенных переменных
\end{answerlist}
\end{question}




\begin{question}
(1 балл) Какое условие НЕ требуется в теореме Гаусса-Маркова?
\begin{answerlist}[2]
  \item матрица регрессоров \(X\) имеет полный ранг
  \item модель \(Y=X\beta + \varepsilon\) правильно специфицирована
  \item случайные ошибки \(\varepsilon_i\) не коррелированы
  \item случайные ошибки \(\varepsilon_i\) имеют одинаковые дисперсии
  \item случайные ошибки \(\varepsilon_i\) нормально распределены
  \item нет верного ответа
\end{answerlist}
\end{question}

\begin{solution}
\begin{answerlist}
  \item Bad answer :(
  \item Bad answer :(
  \item Bad answer :(
  \item Bad answer :(
  \item Good answer :)
\end{answerlist}
\end{solution}


\begin{question}
(1 балл) Выборочная корреляция между регрессорами \(X\) и \(Z\) равна \(0.5\). В
регрессии \(\hat Y_i = \hat\beta_0 + \hat\beta_1 X_i + \hat\beta_2 Z_i\)
показатель \(VIF\) для регрессора \(X\) равен
\begin{answerlist}
  \item \(1/4\)
  \item \(4/3\)
  \item \(1/2\)
  \item \(2\)
  \item \(3/4\)
  \item нет верного ответа
\end{answerlist}
\end{question}

\begin{solution}
\begin{answerlist}
  \item Bad answer :(
  \item Good answer :)
  \item Bad answer :(
  \item Bad answer :(
  \item Bad answer :(
\end{answerlist}
\end{solution}

\newpage

\begin{question}
Стьюдентизированные остатки регрессии используются
\begin{answerlist}
  \item в методе главных компонент
  \item на первом шаге двухшагового МНК
  \item в тесте Саргана
  \item для выявления выбросов
  \item на первом шаге при проведении теста Годфельда-Квандта
\end{answerlist}
\end{question}




\begin{question}
По 52 наблюдениям студент построил две регрессии,
\(\hat Y_i = 3.1 + 0.8X_i\) и \(\hat X_i = -0.3 + 0.2Y_i\). Коэффициент
\(R^2_{adj}\) для первой регрессии примерно равен
\begin{answerlist}
  \item \(0.14\)
  \item \(0.16\)
  \item \(0.40\)
  \item \(0.32\)
  \item \(0.37\)
\end{answerlist}
\end{question}




\begin{question}
Использование скорректированных стандартных ошибок Уайта при
гомоскедастичности приведет к
\begin{answerlist}
  \item смещённости МНК оценок коэффициентов
  \item повышению эффективности МНК оценок коэффициентов
  \item получению состоятельной оценки дисперсии случайной ошибки
  \item понижению эффективности МНК оценок коэффициентов
  \item несостоятельности МНК оценок коэффициентов
\end{answerlist}
\end{question}

\begin{solution}
========
\end{solution}



\begin{question}
Рассмотрим модель множественной регрессии \(Y=X\beta+\varepsilon\), где
\(\hat Y = X\hat\beta\), \(e=Y-\hat Y\). Величина \(RSS\) --- это
квадрат длины вектора
\begin{answerlist}
  \item \(\hat Y - \bar Y\)
  \item \(\varepsilon\)
  \item \(\hat Y\)
  \item \(Y-\bar Y\)
  \item \(e\)
\end{answerlist}
\end{question}




\begin{question}
Чебурашка оценил модель \(Y_i = \beta_0 + \beta_1 X_i + \varepsilon_i\),
а Крокодил Гена --- модель \(X_i = \gamma_0 + \gamma_1 Y_i + u_i\).
Оказалось, что \(\hat\gamma_1 = 0.25/\hat\beta_1\). Величина \(R^2\) в
регрессии Чебурашки равна
\begin{answerlist}
  \item \(1\)
  \item \(0.5\)
  \item \(0\)
  \item \(0.75\)
  \item \(0.25\)
\end{answerlist}
\end{question}

\begin{solution}
\(R^2 = \hat\beta_1 \cdot \hat\gamma_1\)
\end{solution}


\newpage

\begin{question}
Рассмотрим модель
\(Y_i= \beta_0 + \beta_z Z_{i} + \beta_w W_{i} + \varepsilon\) при
гетероскедастичности. Стандартная ошибка МНК-оценки, рассчитываемая по
формуле \(se(\hat\beta_w)=\sqrt{RSS \cdot (X'X)^{-1}_{33}/(n-3)}\),
является
\begin{answerlist}
  \item смещённой
  \item несмещённой
  \item состоятельной
  \item смещённой вниз
  \item смещённой вверх
\end{answerlist}
\end{question}




\newpage

\section*{Задачи}

\begin{enumerate}
\item
(5 баллов) Рассмотрим алгоритм LASSO с параметром регуляризации $\lambda$ для модели $Y=X\beta + \varepsilon$, где все переменные центрированы.
\begin{enumerate}
    \item Выпишите целевую функцию алгоритма.
    \item Что произойдет с оценками $\hat\beta_{LASSO}$ при $\lambda \to \infty$?
    \item Что произойдет с оценками $\hat\beta_{LASSO}$ при $\lambda \to 0$?
\end{enumerate}
\newpage

\item
(5 баллов)
По 200 фирмам была оценена зависимость выпуска $Y$ от труда $L$ и капитала $K$ с помощью двух моделей:

Модель Кобба-Дугласа: $\ln{Y_i} = \beta_0 + \beta_1 \ln{L_i} + \beta_2 \ln{K_i} + \varepsilon_i$

Транслоговая модель: $\ln{Y_i} = \gamma_0 + \gamma_1 \ln{L_i} + \gamma_2 \ln{K_i} + \gamma_3 (0.5 \ln^2{L_i}) + \gamma_4 (0.5 \ln^2{K_i}) + \gamma_5 \ln{K_i} \ln{L_i} + \varepsilon_i$

Оценки коэффициентов обеих моделей (в скобках приведены стандартные ошибки):

\begin{tabular}{lcc}
\toprule
Переменная & Модель Кобба-Дугласа & Транслоговая модель \\
\midrule
константа & 1.1706 (0.326) & 0.9441 (2.911)   \\
$\ln L$ & 0.6029 (0.125) & 3.613 (1.548)  \\
$\ln K$ & 0.375 (0.085) & -1.893 (1.016)  \\
$0.5 \ln^2 L$ &  & -0.964 (0.707)  \\
$0.5 \ln^2 K$ & & 0.0852 (0.2922) \\
$\ln L \ln K$ & & 0.3123 (0.4389)  \\
$R^2$ & 0.9 & 0.954  \\
\bottomrule
\end{tabular}

В модели Кобба-Дугласа $\hCov(\hb_1, \hb_2)= -0.0096$.

%Оценка ковариационной матрицы коэффициентов модели Кобба-Дугласа:

%\begin{tabular}{cccc}
%\toprule
% & константа & $\ln{L}$ & $\ln{K}$  \\
%\midrule
%константа & 0.1068 & &   \\
%$\ln{L}$ & -0.0198 & 0.01586 &  \\
%$\ln{K}$ & 0.00189 & -0.0096 & 0.00728 \\
%\bottomrule
%\end{tabular}

На уровне значимости $\alpha = 0.05$ проверьте следующие гипотезы:
\begin{enumerate}
\item В модели Кобба-Дугласа эластичность выпуска по капиталу равна единице.
\item В модели Кобба-Дугласа эластичности выпуска по труду и капиталу одинаковы.
\item В транслоговой модели $\gamma_4=0$.
\item В транслоговой модели $\gamma_3 = \gamma_4 = \gamma_5 = 0$.
\end{enumerate}

\newpage
\item
(4 балла)
Исследователь оценил зависимость продолжительности жизни $Y$ от концентрации  промышленных выбросов в атмосфере $X$ и ежегодных частных расходов на медицинскую помощь $Z$.

Для 300 жителей индустриальных центров, $\hat{Y}_i = \underset{(10.43)}{65.91} - \underset{(0.0001)}{0.03}X_i - \underset{(0.019)}{0.036}Z_i, \; RSS = 100$.

Для 200 сельских жителей, $\hat{Y}_i = \underset{(15.3)}{58.4} - \underset{(0.006)}{0.017}X_i - \underset{(0.007)}{0.024}Z_i, \; RSS = 200$.

А также по общей выборке, $\hat{Y}_i = \underset{(12.4)}{63.2} - \underset{(0.005)}{0.02}X_i - \underset{(0.001)}{0.031}Z_i, \; RSS = 500$.

В скобках приведены стандартные ошибки.

Можно ли считать, что зависимость едина для городских и сельских жителей?
Ответ обоснуйте подходящим тестом, аккуратно выписав тестируемую гипотезу.
\newpage

\item
(5 баллов)
%Теоретическая зависимость имеет вид:
%\[
%Y_i = \beta_0 + \beta_1 X_{i1} + \beta_2 X_{i2} + \beta_3 X_{i3} + \varepsilon_i
%\]
Исследователь Д'Артаньян стандартизировал (центрировал и нормировал) все имеющиеся регрессоры и поместил их в столбцы матрицы $\tilde X$. Выборочная корреляционная матрица регрессоров равна:
\[
\begin{pmatrix}
1 & 0.80 & 0  \\
0.80 & 1 & 0  \\
0 & 0 & 1 \\
\end{pmatrix}.
\]
\begin{enumerate}
    \item Найдите параметр обусловленности (condition number) матрицы  $\tilde X^T \tilde X$.
    \item Вычислите одну или две главные компоненты, объясняющие не менее 70\% суммарной дисперсии стандартизированных регрессоров. Выпишите найденные компоненты как линейные комбинации столбцов матрицы $\tilde X$.
\end{enumerate}
\newpage

\item
(6 баллов)
Для 400 голландских магазинов модной одежды с помощью трёх моделей оценили зависимость продаж в расчете на квадратный метр в гульденах, $Sales$, от:
\begin{itemize}
\item общей площади магазина, $Size$, в м$^2$;
\item количества сотрудников, работающих целый день, $Nfull$;
\item количества временных рабочих, $Ntemp$;
\item дамми-переменной $Owner$, равной единице, если собственник один, и нулю иначе.
\end{itemize}

$\widehat{Sales}_i = \underset{(718)}{6083} - \underset{(1.59)}{15.25}Size_i + \underset{(171)}{1452.8} Nfull_i + \underset{(423)}{420.15} Ntemp_i - \underset{(361)}{1464.1} Owner_i$

$\ln \widehat{Sales}_i = \underset{(0.11)}{8.59} - \underset{(0.00024)}{0.0024}Size_i + \underset{(0.026)}{0.183} Nfull_i + \underset{(0.066)}{0.102} Ntemp_i - \underset{(0.056)}{0.209} Owner_i$

$\ln \widehat{Sales}_i = \underset{(0.21)}{10.08} - \underset{(0.043)}{0.31}\ln Size_i + \underset{(0.061)}{0.22} \ln Nfull_i + \underset{(0.118)}{0.066} \ln Ntemp_i - \underset{(0.059)}{0.19} \ln Owner_i$

В скобках приведены стандартные ошибки.

\begin{enumerate}
    \item Дайте интерпретацию коэффициента при переменной $Size$ в каждой из трёх моделей;
    \item Подробно опишите, как выбрать наилучшую из этих моделей.
\end{enumerate}

\end{enumerate}

\end{document}
